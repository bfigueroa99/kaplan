\chapter{Meeting Language and Professional Communication}

\section{Lesson Objectives}
In this chapter, you will learn:
\begin{itemize}
    \item Professional vocabulary for meetings
    \item How to open, manage, and close meetings
    \item Expressions for agreeing, disagreeing, and making suggestions
    \item Formal vs. informal register in business contexts
\end{itemize}

\section{Reading Context}
\begin{readingbox}[title=Dialogue: The Project Kick-off]
\textbf{Chair:} \textbf{Right, let's get started.} The purpose of this meeting is to plan the new marketing campaign.\\
\textbf{Mark:} \textbf{How about we start with} the budget? It's the most critical item.\\
\textbf{Sarah:} \textbf{I see your point, but} I think we should define our goals first.\\
\textbf{Chair:} \textbf{I agree with Sarah.} Let's set the goals, then look at the budget.\\
\textbf{Mark:} Fair enough. \textbf{I suggest that we} aim for a 20\% increase in sales.\\
\textbf{Sarah:} \textbf{That sounds good to me.} It's ambitious but achievable.\\
\textbf{Chair:} Great. \textbf{Let's move on to} the timeline.
\end{readingbox}

\section{Key Concepts: Meeting Vocabulary}

\begin{vocabbox}[title=Essential Terms]
\begin{itemize}
    \item \keyterm{Agenda} (noun): A list of items to be discussed at a formal meeting.
    \begin{itemize}
        \item \trans{Orden del día}
    \end{itemize}
    \item \keyterm{Minutes} (noun): The written record of what was said at a meeting.
    \begin{itemize}
        \item \trans{Acta de la reunión}
    \end{itemize}
    \item \keyterm{Chair} (verb/noun): To lead a meeting; the person leading.
    \begin{itemize}
        \item \trans{Presidir / Presidente}
    \end{itemize}
    \item \keyterm{Action Item} (noun): A specific task assigned to someone during a meeting.
    \begin{itemize}
        \item \trans{Tarea asignada}
    \end{itemize}
    \item \keyterm{AOB} (acronym): Any Other Business (discussed at the end).
    \begin{itemize}
        \item \trans{Otros asuntos / Ruegos y preguntas}
    \end{itemize}
\end{itemize}
\end{vocabbox}

\section{Functional Language: Managing Meetings}

\begin{grammarbox}[title=Opening and Closing]
\textbf{Opening:}
\begin{itemize}
    \item "Right, let's get started." \trans{Empecemos}
    \item "The purpose of this meeting is to..." \trans{El propósito es...}
    \item "Thank you all for coming." \trans{Gracias por venir}
\end{itemize}

\textbf{Closing:}
\begin{itemize}
    \item "Let's wrap this up." \trans{Terminemos esto}
    \item "To summarize what we've decided..." \trans{Para resumir...}
    \item "The meeting is adjourned." \trans{Se levanta la sesión}
\end{itemize}
\end{grammarbox}

\begin{grammarbox}[title=Discussion Phrases]
\textbf{Making Suggestions:}
\begin{itemize}
    \item "How about we...?" \trans{¿Qué tal si...?}
    \item "I suggest that we..." \trans{Sugiero que...}
\end{itemize}

\textbf{Agreeing:}
\begin{itemize}
    \item "I completely agree." \trans{Totalmente de acuerdo}
    \item "That sounds good to me." \trans{Me parece bien}
\end{itemize}

\textbf{Disagreeing (Politely):}
\begin{itemize}
    \item "I see your point, but..." \trans{Entiendo tu punto, pero...}
    \item "I'm not sure about that because..." \trans{No estoy seguro porque...}
\end{itemize}
\end{grammarbox}

\section{Formal vs. Informal Register}

\begin{table}[h]
\centering
\begin{tabular}{|p{6cm}|p{6cm}|}
\hline
\textbf{Informal (Colleagues)} & \textbf{Formal (Clients/Superiors)} \\
\hline
Let's start. & Shall we begin? \\
\hline
What do you think? & What is your opinion on this? \\
\hline
I don't agree. & I'm afraid I have to disagree. \\
\hline
Can you say that again? & Could you please repeat that? \\
\hline
That's a bad idea. & I have some concerns about that. \\
\hline
\end{tabular}
\caption{Register comparison}
\end{table}

\section{Practice Exercises}

\subsection{Exercise 1: Categorize the Expressions}
Put each expression in the correct category: Opening, Suggesting, Agreeing, Disagreeing, or Closing.

\begin{enumerate}
    \item "I see your point, but..." $\rightarrow$ \underline{\hspace{3cm}}
    \item "Shall we begin?" $\rightarrow$ \underline{\hspace{3cm}}
    \item "That sounds good to me." $\rightarrow$ \underline{\hspace{3cm}}
    \item "Why don't we...?" $\rightarrow$ \underline{\hspace{3cm}}
    \item "To summarize what we've decided..." $\rightarrow$ \underline{\hspace{3cm}}
\end{enumerate}

\subsection{Exercise 2: Make it More Formal}
Rewrite these informal expressions in a more formal way:

\begin{enumerate}
    \item "That's wrong." $\rightarrow$ \underline{\hspace{5cm}}
    \item "Let's talk about the budget." $\rightarrow$ \underline{\hspace{5cm}}
    \item "I like that." $\rightarrow$ \underline{\hspace{5cm}}
\end{enumerate}

\subsection{Exercise 3: Role Play Script}
Write a short dialogue for a meeting where you suggest a new idea (e.g., "Casual Fridays") and a colleague disagrees politely.

\begin{tcolorbox}[colback=white,height=5cm]
% Write your dialogue here
\end{tcolorbox}

\section{Key Takeaways}
\begin{itemize}
    \item Meetings have a clear structure: open, discuss, summarize, close.
    \item Use polite phrases when disagreeing to maintain professional relationships.
    \item Adjust your formality based on who is in the meeting.
    \item "Minutes" are the notes, not the time!
\end{itemize}

