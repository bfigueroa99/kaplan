\chapter{Advanced Verb Patterns}

\section{Lesson Objectives}
In this chapter, you will learn:
\begin{itemize}
    \item Using Gerunds as the subject of a sentence
    \item Verbs that change meaning with Gerund vs. Infinitive (Stop, Remember, Try)
    \item Expressing purpose with the Infinitive
\end{itemize}

\section{Reading Context}
\begin{readingbox}[title=Dialogue: Changing Habits]
\textbf{Mike:} I need to get fit. \textbf{Running} is too hard for me.\
\textbf{Lisa:} Have you tried \textbf{swimming}? It's great exercise.\
\textbf{Mike:} I remember \textbf{going} swimming as a child. I loved it.\
\textbf{Lisa:} You should start again. But remember \textbf{to bring} goggles.\
\textbf{Mike:} Good idea. I stopped \textbf{exercising} last year because of work.\
\textbf{Lisa:} Well, stop \textbf{to think} about your health. It's important.
\end{readingbox}

\section{Key Concepts: Gerund as Subject}

We often use the Gerund (-ing) as the subject of a sentence. It acts like a noun.

\begin{vocabbox}[title=Gerund Subjects]
\begin{itemize}
    \item \textbf{Swimming} is good for you. \trans{Nadar es bueno...}
    \item \textbf{Learning} English takes time. \trans{Aprender inglés...}
    \item \textbf{Smoking} is forbidden here. \trans{Fumar está prohibido...}
\end{itemize}
\end{vocabbox}

\section{Grammar Focus: Meaning Changes}

Some verbs can take both forms, but the meaning changes completely.

\begin{grammarbox}[title=1. Stop]
\begin{itemize}
    \item \textbf{Stop doing:} Quit an action forever/temporarily.
    \begin{itemize}
        \item He \textbf{stopped smoking}. (He doesn't smoke anymore)
    \end{itemize}
    \item \textbf{Stop to do:} Pause one action to do another.
    \begin{itemize}
        \item He \textbf{stopped to smoke}. (He paused walking to have a cigarette)
    \end{itemize}
\end{itemize}
\end{grammarbox}

\begin{grammarbox}[title=2. Remember]
\begin{itemize}
    \item \textbf{Remember doing:} Recall a memory from the past.
    \begin{itemize}
        \item I \textbf{remember locking} the door. (I have a memory of it)
    \end{itemize}
    \item \textbf{Remember to do:} Don't forget to do a task.
    \begin{itemize}
        \item Please \textbf{remember to lock} the door. (Don't forget!)
    \end{itemize}
\end{itemize}
\end{grammarbox}

\begin{grammarbox}[title=3. Try]
\begin{itemize}
    \item \textbf{Try doing:} Experiment with something to see if it works.
    \begin{itemize}
        \item \textbf{Try restarting} your computer. (See if it fixes the problem)
    \end{itemize}
    \item \textbf{Try to do:} Make an effort to do something difficult.
    \begin{itemize}
        \item I \textbf{tried to lift} the box, but it was too heavy.
    \end{itemize}
\end{itemize}
\end{grammarbox}

\section{Practice Exercises}

\subsection{Exercise 1: Gerund Subjects}
Complete the sentences with a gerund.

\begin{enumerate}
    \item \underline{\hspace{3cm}} (eat) vegetables is healthy.
    \item \underline{\hspace{3cm}} (drive) fast is dangerous.
    \item \underline{\hspace{3cm}} (read) books helps you learn.
\end{enumerate}

\subsection{Exercise 2: Stop, Remember, Try}
Choose the correct form based on the meaning.

\begin{enumerate}
    \item I stopped (buying / to buy) coffee because it was too expensive.
    \item On my way home, I stopped (buying / to buy) some milk.
    \item Remember (calling / to call) your mother today.
    \item I remember (playing / to play) in this park when I was young.
    \item The window is stuck. Try (pushing / to push) it harder.
\end{enumerate}

\subsection{Exercise 3: Writing Task}
Write 3 sentences about things you:
\begin{itemize}
    \item Stopped doing recently.
    \item Remember doing as a child.
    \item Try to do every day.
\end{itemize}

\begin{tcolorbox}[colback=white,height=5cm]
% Write here...
\end{tcolorbox}

\section{Key Takeaways}
\begin{itemize}
    \item Gerunds (-ing) can be the subject of a sentence (e.g., \textbf{Running} is fun).
    \item \textbf{Stop + -ing} = Quit. \textbf{Stop + to} = Pause to do something else.
    \item \textbf{Remember + -ing} = Memory. \textbf{Remember + to} = Task.
\end{itemize}
