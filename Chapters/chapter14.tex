\chapter{Advanced Verb Patterns}

\section{Lesson Objectives}
In this chapter, you will learn:
\begin{itemize}
    \item Using Gerunds as the subject of a sentence
    \item Verbs that change meaning with Gerund vs. Infinitive (Stop, Remember, Try)
    \item Expressing purpose with the Infinitive
\end{itemize}

\section{Reading Context}
\begin{readingbox}[title=Dialogue: Changing Habits]
\textbf{Mike:} I need to get fit. \textbf{Running} is too hard for me.\

\textbf{Lisa:} Have you tried \textbf{swimming}? It's great exercise.\

\textbf{Mike:} I remember \textbf{going} swimming as a child. I loved it.\

\textbf{Lisa:} You should start again. But remember \textbf{to bring} goggles.\

\textbf{Mike:} Good idea. I stopped \textbf{exercising} last year because of work.\

\textbf{Lisa:} Well, stop \textbf{to think} about your health. It's important.
\end{readingbox}

\section{Key Concepts: Gerund as Subject}

We often use the Gerund (-ing) as the subject of a sentence. It acts like a noun.

\begin{vocabbox}[title=Gerund Subjects]
\begin{itemize}
    \item \textbf{Swimming} is good for you. \trans{Nadar es bueno...}
    \item \textbf{Learning} English takes time. \trans{Aprender inglés...}
    \item \textbf{Smoking} is forbidden here. \trans{Fumar está prohibido...}
\end{itemize}
\end{vocabbox}

\section{Grammar Focus: Meaning Changes}

Some verbs can take both forms, but the meaning changes completely.

\begin{grammarbox}[title=1. Stop]
\begin{itemize}
    \item \textbf{Stop doing:} Quit an action forever/temporarily.
    \begin{itemize}
        \item He \textbf{stopped smoking}. (He doesn't smoke anymore)
    \end{itemize}
    \item \textbf{Stop to do:} Pause one action to do another.
    \begin{itemize}
        \item He \textbf{stopped to smoke}. (He paused walking to have a cigarette)
    \end{itemize}
\end{itemize}
\end{grammarbox}

\begin{grammarbox}[title=2. Remember]
\begin{itemize}
    \item \textbf{Remember doing:} Recall a memory from the past.
    \begin{itemize}
        \item I \textbf{remember locking} the door. (I have a memory of it)
    \end{itemize}
    \item \textbf{Remember to do:} Don't forget to do a task.
    \begin{itemize}
        \item Please \textbf{remember to lock} the door. (Don't forget!)
    \end{itemize}
\end{itemize}
\end{grammarbox}

\begin{grammarbox}[title=3. Try]
\begin{itemize}
    \item \textbf{Try doing:} Experiment with something to see if it works.
    \begin{itemize}
        \item \textbf{Try restarting} your computer. (See if it fixes the problem)
    \end{itemize}
    \item \textbf{Try to do:} Make an effort to do something difficult.
    \begin{itemize}
        \item I \textbf{tried to lift} the box, but it was too heavy.
    \end{itemize}
\end{itemize}
\end{grammarbox}

\section{Practice Exercises}

\subsection{Exercise 1: Gerund Subjects}
Complete the sentences with a gerund.

\begin{enumerate}
    \item \underline{\hspace{3cm}} (eat) vegetables is healthy.
    \item \underline{\hspace{3cm}} (drive) fast is dangerous.
    \item \underline{\hspace{3cm}} (read) books helps you learn.
\end{enumerate}

\subsection{Exercise 2: Stop, Remember, Try}
Choose the correct form based on the meaning.

\begin{enumerate}
    \item I stopped (buying / to buy) coffee because it was too expensive.
    \item On my way home, I stopped (buying / to buy) some milk.
    \item Remember (calling / to call) your mother today.
    \item I remember (playing / to play) in this park when I was young.
    \item The window is stuck. Try (pushing / to push) it harder.
\end{enumerate}

\subsection{Exercise 3: Writing Task}
Write 3 sentences about things you:
\begin{itemize}
    \item Stopped doing recently.
    \item Remember doing as a child.
    \item Try to do every day.
\end{itemize}

\begin{tcolorbox}[colback=white,height=5cm]
% Write here...
\end{tcolorbox}

\section{Key Takeaways}
\begin{itemize}
    \item Gerunds (-ing) can be the subject of a sentence (e.g., \textbf{Running} is fun).
    \item \textbf{Stop + -ing} = Quit. \textbf{Stop + to} = Pause to do something else.
    \item \textbf{Remember + -ing} = Memory. \textbf{Remember + to} = Task.
\end{itemize}


\section{Summary Chart}

This comprehensive chart summarizes the key patterns discussed throughout the chapter, helping you understand which verbs accept gerunds, infinitives, or both.

\subsection{Verbs with Same Meaning (Gerund or Infinitive)}

When these verbs are followed by either a gerund or infinitive, the meaning remains essentially the same:

\begin{tcolorbox}[colback=blue!5, colframe=blue!30, title=Same Meaning Pattern]
\begin{itemize}
    \item \textbf{Recommend + Gerund/Infinitive}
    \begin{itemize}
        \item I recommend \textbf{studying} English daily. (or: I recommend \textbf{to study} English daily.)
    \end{itemize}
    \item \textbf{Avoid + Gerund} (Note: avoid typically takes gerund only)
    \begin{itemize}
        \item You should avoid \textbf{eating} too much sugar.
    \end{itemize}
    \item \textbf{Want + Infinitive} (Note: want typically takes infinitive only)
    \begin{itemize}
        \item I want \textbf{to learn} French next year.
    \end{itemize}
    \item \textbf{Love + Both Gerund and Infinitive} (Both are correct)
    \begin{itemize}
        \item She loves \textbf{swimming}. / She loves \textbf{to swim}.
        \item I love \textbf{reading} novels. / I love \textbf{to read} novels.
    \end{itemize}
\end{itemize}
\end{tcolorbox}

\subsection{Verbs with Different Meaning (Gerund vs. Infinitive)}

These verbs change their meaning dramatically depending on whether they are followed by a gerund or infinitive:

\begin{tcolorbox}[colback=red!5, colframe=red!30, title=Different Meaning Pattern]
    \textbf{Stop + Doing vs. Stop + to Do}
    \begin{itemize}
        \item \textbf{Stop doing:} Cease or quit an activity permanently.
        \begin{itemize}
            \item I stopped \textbf{smoking} five years ago. (I no longer smoke)
        \end{itemize}
        \item \textbf{Stop to do:} Pause one activity in order to do something else.
        \begin{itemize}
            \item We stopped \textbf{to eat} lunch during our road trip. (We paused our journey to eat)
        \end{itemize}
    \end{itemize}
\end{tcolorbox}


\begin{tcolorbox}[colback=green!5, colframe=green!30, title=Different Meaning Pattern Continued]
    \textbf{Forget + Doing vs. Forget + to Do}
    \begin{itemize}
        \item \textbf{Forget doing:} Fail to recall a past action or event. (it did happens)
        \begin{itemize}
            \item I forgot \textbf{closing} the windows this morning. (I can't remember if I closed them)
        \end{itemize}
        \item \textbf{Forget to do:} Neglect or fail to perform a necessary task. In other words, you didn't do it.
        \begin{itemize}
            \item Don't forget \textbf{to lock} the door when you leave. (Remember to do this)
        \end{itemize}
    \end{itemize}
\end{tcolorbox}



\begin{tcolorbox}[colback=yellow!5, colframe=yellow!30, title=Different Meaning Pattern Continued]
    \textbf{Remember + Doing vs. Remember + to Do}
    \begin{itemize}
        \item \textbf{Remember doing:} Recall a past action or event.
        \begin{itemize}
            \item I remember \textbf{visiting} Paris last summer. (I have a memory of it)
        \end{itemize}
        \item \textbf{Remember to do:} Recall a task that needs to be performed.
        \begin{itemize}
            \item Please remember \textbf{to send} the email before noon. (Don't forget!)
        \end{itemize}
    \end{itemize}
\end{tcolorbox}


\begin{tcolorbox}[colback=orange!5, colframe=orange!30, title=Different Meaning Pattern Continued]
    \textbf{Try + Doing vs. Try + to Do}
    \begin{itemize}
        \item \textbf{Try doing:} Experiment with something as a possible solution.
        \begin{itemize}
            \item If it doesn't work, try \textbf{turning} it off and on again. (Experiment with this solution)
        \end{itemize}
        \item \textbf{Try to do:} Make an effort or attempt to accomplish something difficult.
        \begin{itemize}
            \item I tried \textbf{to open} the jar, but it was sealed tight. (I made an effort)
        \end{itemize}
    \end{itemize}
\end{tcolorbox}


\begin{tcolorbox}[colback=purple!5, colframe=purple!30, title=Different Meaning Pattern Continued]
    \textbf{Effort + Doing vs. Effort + to Do}
    \begin{itemize}
        \item \textbf{Effort doing:} Engage in an activity as a way to achieve a goal.
        \begin{itemize}
            \item She made an effort \textbf{studying} every day to improve her grades. (Engaging in the activity)
        \end{itemize}
        \item \textbf{Effort to do:} Make a conscious attempt to accomplish something.
        \begin{itemize}
            \item He made an effort \textbf{to finish} the project on time. (Conscious attempt)
        \end{itemize}
    \end{itemize}
\end{tcolorbox}


\begin{tcolorbox}[colback=red!5, colframe=red!30, title=Different Meaning Pattern Continued]
    \textbf{Regret + Doing vs. Regret + to Do}
    \begin{itemize}
        \item \textbf{Regret doing:} Feel sorry about a past action.
        \begin{itemize}
            \item I regret \textbf{telling} him the truth. (I feel sorry about it)
        \end{itemize}
        \item \textbf{Regret to do:} Feel sorry about having to do something.
        \begin{itemize}
            \item We regret \textbf{to inform} you that your application was unsuccessful. (We are sorry to have to tell you this)
        \end{itemize}
    \end{itemize}
\end{tcolorbox}



\subsection{Comprehensive Verb Pattern Chart}   

Verbs that are followed by gerunds, infinitives, or both are summarized in the table below:

\begin{table}[h!]
    \centering
    \begin{tabular}{|l|l|l|l|}
        \hline
        \textbf{Gerund} & \textbf{Infinitive} & \textbf{Both (Same Meaning)} & \textbf{Both (Different Meaning)} \\
        \hline
        Admit & Expect & Like & Stop \\
        Admission & Want & Hate & Forget \\
        Consider & Wish & Love & Try \\
        Enjoy & Decide & Prefer & Remember \\
        Imagine & Afford & Start & Effort \\
        Involve & Learn & Begin & \\
        Recommend & Arrange & Continue & \\
        Suggest & Fail & & \\
        & Expect & & \\
        & Hesitate & & \\
        & Need & & \\
        & Manage & & \\
        & Plan &  & \\
        & Seem &  & \\
        & Teach &  & \\
        \hline
    \end{tabular}
    \caption{Table showing verbs followed by gerunds, infinitives, or both. }
    \label{tab:gerund_infinitive}
\end{table}

\section{Where to Practice Gerunds and Infinitives}

To reinforce your understanding of gerund and infinitive patterns, practice with interactive exercises and quizzes online. Here are some recommended resources:

\begin{itemize}
    \item \href{https://test-english.com/grammar-points/b1-b2/gerund-or-infinitive/}{Test-English.com: Gerund or Infinitive} \\
    Practice with clear explanations and interactive quizzes for B1-B2 level learners.
    
    \item \href{https://www.perfect-english-grammar.com/gerunds-and-infinitives-exercises.html}{Perfect English Grammar: Gerunds and Infinitives Exercises} \\
    A variety of exercises to test your knowledge and improve your skills.
    
    \item \href{https://www.englishpage.com/gerunds/gerund_or_infinitive_1.htm}{EnglishPage.com: Gerund or Infinitive Quiz} \\
    Multiple quizzes with instant feedback and explanations.
    
    \item \href{https://learnenglish.britishcouncil.org/grammar/english-grammar-reference/verbs-followed-by-ing-or-to-infinitive}{British Council: Verbs followed by -ing or to + infinitive} \\
    Detailed grammar reference and practice activities.
    
    \item \href{https://agendaweb.org/verbs/infinitive_gerund-exercises.html}{Agenda Web: Gerund and Infinitive Exercises} \\
    Multiple exercises organized by difficulty level with interactive practice.
    
    \item \href{https://english.lingolia.com/en/grammar/verbs/infinitive-gerund/exercises}{Lingolia: Infinitive and Gerund Free Exercise} \\
    Exercises with automatic correction and instant feedback.
    
    \item \href{https://www.english-grammar.at/online_exercises/gerund-infinitive/gerund-infinitive-index.htm}{English Grammar Online: Gerund-Infinitive Index} \\
    Wide variety of online exercises for different proficiency levels.
    
    \item \href{https://www.grammarism.com/infinitive-gerund-exercises/}{Grammarism: 2080 Infinitive Gerund Exercises} \\
    101 different tests with over 2000 exercises for extensive practice.
    
    \item \href{https://www.grammarbank.com/gerunds-infinitives-exercises.html}{GrammarBank: Gerunds and Infinitives Exercises} \\
    8 different exercises with downloadable PDFs for offline practice.
    
    \item \href{https://learnenglishteens.britishcouncil.org/grammar/a1-a2-grammar/verb-ing-or-verb-infinitive}{British Council Teens: Verb + ing or infinitive} \\
    Videos with subtitles and interactive exercises for teenage learners.
    
    \item \href{https://premierskillsenglish.britishcouncil.org/podcasts/understanding-grammar/understanding-grammar-gerunds-and-infinitives}{Premier Skills English: Understanding Grammar Podcast} \\
    Educational podcast with listening activities and practical exercises.
    
    \item \href{https://www.englisch-hilfen.de/en/exercises/structures/gerund_infinitive.htm}{Englisch-Hilfen: Gerund and Infinitive Online Exercise} \\
    Fill-in-the-blank exercises with immediate correction and scoring.
\end{itemize}

\begin{tcolorbox}[colback=blue!5, colframe=blue!30, title=Tip]
Regular practice with these resources will help you master the difference between gerunds and infinitives in real-life contexts.
\end{tcolorbox}  


---------------------------------------------------------------------------------------------------------

\subsection{Create Christmas Story}

Write a Christmas story and have a dark twist.

\begin{tcolorbox}[colback=white,height=6cm]
% Write your story here...

A long time ago, in a small village, there was a tradition of giving gifts on Christmas Eve. Homewer, one day, a mysterious stranger arrived, offering a special gift to each villager. The gifts were beautiful, but they came with a curse. Each person who accepted a gift found themselves trapped in a never-ending nightmare. But one child, named Clara, discovered the truth behind the gifts. She bravely confronted the stranger, only to find out that he was a dark spirit feeding on their fears. In the end, Clara sacrificed her own freedom to save the village, becoming a guardian spirit herself, forever watching over the villagers on Christmas Eve.


\end{tcolorbox}



---------------------------------------------------------------------------------------------------------


\subsection{What are your 3 biggest dreams for the future?}

\begin{tcolorbox}[colback=white,height=4cm]
% Write your dreams here...

1. Live in a forest, surrounded by nature and wildlife.

2. Have cows and hourses roaming freely on my property.

3. Find a partner of life

\end{tcolorbox}

\subsection{ a) why is it sometimes hard to follow your dreams?}

\begin{tcolorbox}[colback=white,height=4cm]

Sometimes it is hard to follow your dreams because of fear of failure and uncertainty about the future, and be realistic about what is achievable.

\end{tcolorbox}

\subsection{ b) do your family/ friends support your dreams?}

\begin{tcolorbox}[colback=white,height=4cm]



\end{tcolorbox}


\subsection{ c) have you ever felt pressured to follow someone else's expectations}


\begin{tcolorbox}[colback=white,height=4cm]



\end{tcolorbox}

\section{Key Takeaways}
\begin{itemize}
    \item Gerunds can be subjects of sentences (Swimming is fun).
    \item Some verbs change meaning: Stop -ing (quit) vs. Stop to do (pause).
    \item Remember -ing (previous action) vs. to do (next action).
    \item Use infinitives to express purpose (I went to buy...).
\end{itemize}

\section{Online Practice}
\begin{tcolorbox}[colback=blue!5,colframe=blue!40!black,title=Resources to Practice]
Online PracticeReinforce what you have learned with these interactive exercises:
\begin{itemize}
    \item \textbf{Advanced Verb Patterns:} \url{https://www.perfect-english-grammar.com/gerunds-and-infinitives-exercise-2.html}
    \item \textbf{Stop/Remember/Try:} \url{https://test-english.com/grammar-points/b2/verb-patterns-gerund-infinitive-different-meaning/}
    \item \textbf{Purpose Infinitives:} \url{https://learnenglish.britishcouncil.org/grammar/english-grammar-reference/infinitives-with-and-without-to}
    \item \textbf{Complete Practice:} \url{https://www.englishpage.com/gerunds/advanced_gerund_infinitive.htm}
\end{itemize}
\end{tcolorbox}


