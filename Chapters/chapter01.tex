\chapter{Non-verbal Communication}

\section{Lesson Objectives}
In this chapter, you will learn:
\begin{itemize}
    \item Vocabulary related to body language and non-verbal communication
    \item How to form adverbs from adjectives using suffixes (-ly, -ed)
    \item How to describe personality traits based on body language
\end{itemize}

\section{Reading Context}
\begin{readingbox}[title=Dialogue: The Interview]
\textbf{Interviewer:} Welcome, please sit down. I noticed you were waiting \textbf{quietly} in the lobby.\\
\textbf{Candidate:} Thank you. I was a bit nervous, so I tried to stay calm.\\
\textbf{Interviewer:} I see. Your \textbf{posture} is very confident now. That's good.\\
\textbf{Candidate:} I try to be aware of my body language. I don't want to \textbf{gesticulate} too much.\\
\textbf{Interviewer:} It's okay to be \textbf{expressive}. We are looking for someone who can communicate effectively.\\
\textbf{Candidate:} That's a relief! I was worried I might seem too \textbf{imposing} if I used big gestures.\\
\textbf{Interviewer:} Not at all. Just speak \textbf{naturally}.
\end{readingbox}

\section{Key Concepts: Body Language}

Body language is a form of non-verbal communication where physical behaviors, as opposed to words, are used to express or convey information.

\begin{vocabbox}[title=Key Vocabulary]
\begin{itemize}
    \item \keyterm{Gesticulate} (verb): To use gestures, especially dramatic ones, instead of speaking or to emphasize one's words.
    \begin{itemize}
        \item Example: "You gesticulate a lot, which means you're expressive."
        \item \trans{Gesticular mucho}
    \end{itemize}
    \item \keyterm{Posture} (noun): The position in which someone holds their body when standing or sitting.
    \begin{itemize}
        \item Example: "Your posture can be pretty imposing."
        \item \trans{Postura}
    \end{itemize}
    \item \keyterm{Expressive} (adjective): Effectively conveying thought or feeling.
    \begin{itemize}
        \item \trans{Expresivo/a}
    \end{itemize}
    \item \keyterm{Imposing} (adjective): Grand and impressive in appearance.
    \begin{itemize}
        \item \trans{Imponente}
    \end{itemize}
\end{itemize}
\end{vocabbox}

\section{Grammar Focus: Suffixes (-ly and -ed)}

Suffixes are added to the end of words to change their meaning or grammatical function.

\begin{grammarbox}[title=The -ly Suffix (Adverbs)]
Adding \textbf{-ly} to an adjective creates an adverb that describes \textit{how} something is done.

\begin{center}
\Large Adjective + \textbf{-ly} = Adverb
\end{center}

\textbf{Examples:}
\begin{itemize}
    \item Quiet + ly $\rightarrow$ \textbf{Quietly} \trans{Silenciosamente}
    \item Sudden + ly $\rightarrow$ \textbf{Suddenly} \trans{De repente}
\end{itemize}
\end{grammarbox}

\begin{grammarbox}[title=The -ed Suffix (Adjectives from Verbs)]
Adding \textbf{-ed} to some verbs creates adjectives that describe feelings or states.

\begin{center}
\Large Verb + \textbf{-ed} = Adjective (feeling/state)
\end{center}

\textbf{Examples:}
\begin{itemize}
    \item Confuse + ed $\rightarrow$ \textbf{Confused} \trans{Confundido/a}
    \item Relieve + ed $\rightarrow$ \textbf{Relieved} \trans{Aliviado/a}
\end{itemize}
\end{grammarbox}

\section{Vocabulary Reference}

\begin{table}[h]
\centering
\begin{tabular}{|l|l|l|p{5cm}|}
\hline
\textbf{Word with suffix} & \textbf{Root word} & \textbf{Meaning in Spanish} & \textbf{Example Situation} \\
\hline
quietly & quiet & silenciosamente & Monica quietly asks for a cup of water \\
\hline
grateful & gratitude & agradecido/a & Monica feels grateful because Heather opened door \\
\hline
curiously & curious & con curiosidad & Heather looks at Monica curiously \\
\hline
suddenly & sudden & de repente & Monica gets up suddenly leaving her glass on the table \\
\hline
seriously & serious & en serio & He looked at me seriously and said nothing \\
\hline
relieved & relieve & aliviado/a & I felt relieved when the exam was over \\
\hline
casually & casual & casualmente & She casually mentioned her new job \\
\hline
confused & confuse & confundido/a & He looked confused by the instructions \\
\hline
shyly & shy & tímidamente & She shyly introduced herself \\
\hline
troubled & trouble & preocupado/a & He seemed troubled by the news \\
\hline
cautiously & caution & con cautela & She cautiously opened the door \\
\hline
immediately & immediate & inmediatamente & He immediately called for help \\
\hline
\end{tabular}
\caption{Words with suffixes analysis}
\end{table}

\section{Practice Exercises}

\subsection{Exercise 1: Identify the Root Word}
Write the root word for each of the following:
\begin{enumerate}
    \item Nervously $\rightarrow$ \underline{\hspace{3cm}}
    \item Excited $\rightarrow$ \underline{\hspace{3cm}}
    \item Happily $\rightarrow$ \underline{\hspace{3cm}}
    \item Worried $\rightarrow$ \underline{\hspace{3cm}}
\end{enumerate}

\subsection{Exercise 2: Complete the Sentences}
Use the correct form of the word in parentheses:

\begin{enumerate}
    \item She spoke \underline{\hspace{2cm}} (quiet) during the meeting.
    \item I was \underline{\hspace{2cm}} (confuse) by his explanation.
    \item He \underline{\hspace{2cm}} (sudden) stood up and left.
    \item They looked \underline{\hspace{2cm}} (trouble) about something.
\end{enumerate}

\subsection{Exercise 3: Describe Body Language}
Write 3 sentences describing someone's body language using the vocabulary from this chapter.

\begin{tcolorbox}[colback=white,height=4cm]
% Write your sentences here
\end{tcolorbox}

\section{Key Takeaways}
\begin{itemize}
    \item Non-verbal communication includes gestures, posture, and facial expressions.
    \item The suffix \textbf{-ly} transforms adjectives into adverbs.
    \item The suffix \textbf{-ed} can transform verbs into adjectives describing feelings.
    \item Body language can reveal personality traits like being expressive or imposing.
\end{itemize}

\section{Online Practice}
\begin{tcolorbox}[colback=blue!5,colframe=blue!40!black,title=Resources to Practice]
Online PracticeReinforce what you have learned with these interactive exercises:
\begin{itemize}
    \item \textbf{Body Language Quiz:} \url{https://www.englishclub.com/business-english/body-language.php}
    \item \textbf{Adverbs with -ly:} \url{https://www.perfect-english-grammar.com/adverbs-of-manner.html}
    \item \textbf{-ed/-ing Adjectives:} \url{https://test-english.com/grammar-points/a2/ed-ing-adjectives/}
    \item \textbf{Non-verbal Communication:} \url{https://learnenglish.britishcouncil.org/vocabulary/b1-b2-vocabulary/body-parts-2}
\end{itemize}
\end{tcolorbox}
