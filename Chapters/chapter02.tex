\chapter{Business English and Adverbs}

\section{Lesson Objectives}
In this chapter, you will learn:
\begin{itemize}
    \item Business vocabulary related to partnerships and entrepreneurship
    \item Adverbs of comment and viewpoint
    \item How to express opinions in professional contexts
\end{itemize}

\section{Reading Context}
\begin{readingbox}[title=Dialogue: The Business Proposal]
\textbf{Entrepreneur:} \textbf{Frankly}, I think this new app idea is going to be a game-changer.\\
\textbf{Investor:} It sounds interesting. \textbf{Obviously}, the market is competitive right now.\\
\textbf{Entrepreneur:} That's true. But \textbf{fortunately}, we have a unique feature that no one else has.\\
\textbf{Investor:} \textbf{Personally}, I like the concept. But I need to see a solid business plan.\\
\textbf{Entrepreneur:} \textbf{Clearly}, we need to work on the financial details. \textbf{Ideally}, we can present it next week.\\
\textbf{Investor:} \textbf{Surprisingly}, I'm free next Tuesday. Let's meet then.
\end{readingbox}

\section{Key Concepts: Business Fundamentals}

\begin{vocabbox}[title=Business Vocabulary]
\begin{itemize}
    \item \keyterm{Partnership} (noun): A business relationship between two or more people.
    \begin{itemize}
        \item \trans{Sociedad / Asociación}
    \end{itemize}
    \item \keyterm{Entrepreneur} (noun): A person who starts a business, taking on financial risks in the hope of profit.
    \begin{itemize}
        \item \trans{Emprendedor}
    \end{itemize}
    \item \keyterm{Investor} (noun): A person who puts money into a business with the expectation of achieving a profit.
    \begin{itemize}
        \item \trans{Inversionista}
    \end{itemize}
    \item \keyterm{Stakeholder} (noun): A person with an interest or concern in a business.
    \begin{itemize}
        \item \trans{Parte interesada}
    \end{itemize}
    \item \keyterm{Revenue} (noun): Income, especially when of a company or organization and of a substantial nature.
    \begin{itemize}
        \item \trans{Ingresos}
    \end{itemize}
\end{itemize}
\end{vocabbox}

\section{Grammar Focus: Adverbs of Comment}

These adverbs express the speaker's opinion or attitude about what they are saying. They usually come at the beginning of a sentence.

\begin{grammarbox}[title=Adverbs of Comment and Viewpoint]
\textbf{Structure:}
\begin{center}
\Large \textbf{Adverb} + , + Sentence
\end{center}

\textbf{Examples:}
\begin{itemize}
    \item \textbf{Frankly}, I don't think this will work. \trans{Francamente...}
    \item \textbf{Unfortunately}, the meeting was cancelled. \trans{Desafortunadamente...}
    \item \textbf{Obviously}, we need a new strategy. \trans{Obviamente...}
\end{itemize}
\end{grammarbox}

\section{Vocabulary Reference}

\begin{table}[h]
\centering
\begin{tabular}{|l|l|p{6cm}|}
\hline
\textbf{Adverb} & \textbf{Spanish} & \textbf{Example Sentence} \\
\hline
Frankly & Francamente & Frankly, I don't think this will work. \\
\hline
Obviously & Obviamente & Obviously, we need to change our strategy. \\
\hline
Unfortunately & Desafortunadamente & Unfortunately, the meeting was cancelled. \\
\hline
Fortunately & Afortunadamente & Fortunately, we finished on time. \\
\hline
Honestly & Honestamente & Honestly, I prefer the first option. \\
\hline
Clearly & Claramente & Clearly, there's been a misunderstanding. \\
\hline
Surprisingly & Sorprendentemente & Surprisingly, the project was a success. \\
\hline
Apparently & Aparentemente & Apparently, they're closing the office. \\
\hline
Personally & Personalmente & Personally, I think we should wait. \\
\hline
Ideally & Idealmente & Ideally, we should start next week. \\
\hline
\end{tabular}
\caption{Adverbs of comment and viewpoint}
\end{table}

\section{Practice Exercises}

\subsection{Exercise 1: Complete with an Adverb}
Choose the correct adverb of comment:

\begin{enumerate}
    \item \underline{\hspace{2cm}}, the weather was perfect for our event. (Fortunately/Frankly)
    \item \underline{\hspace{2cm}}, I don't understand why they made that decision. (Obviously/Honestly)
    \item The results were, \underline{\hspace{2cm}}, better than expected. (surprisingly/unfortunately)
\end{enumerate}

\subsection{Exercise 2: Business Partnership Discussion}
Write 3 sentences about what makes a good business partnership using adverbs of comment.

\begin{tcolorbox}[colback=white,height=4cm]
% Example: Honestly, I believe trust is the most important element.
\end{tcolorbox}

\section{Key Takeaways}
\begin{itemize}
    \item Business partnerships require trust, communication, and clear agreements.
    \item Adverbs of comment express the speaker's opinion or attitude.
    \item These adverbs usually come at the beginning of a sentence followed by a comma.
\end{itemize}

\section{Online Practice}
\begin{tcolorbox}[colback=blue!5,colframe=blue!40!black,title=Resources to Practice]
Online PracticeReinforce what you have learned with these interactive exercises:
\begin{itemize}
    \item \textbf{Business English Vocabulary:} \url{https://www.businessenglishpod.com/category/business-english-vocabulary/}
    \item \textbf{Adverbs of Comment:} \url{https://test-english.com/grammar-points/b1/adverbs-of-comment/}
    \item \textbf{Business Discussions Practice:} \url{https://www.eslfast.com/robot/topics/business/business.htm}
    \item \textbf{Entrepreneurship Vocabulary:} \url{https://www.englishclub.com/business-english/vocabulary.htm}
\end{itemize}
\end{tcolorbox}


