\chapter{Conditionals: If Clauses}

\section{Lesson Objectives}
In this chapter, you will learn:
\begin{itemize}
    \item The four types of conditional sentences
    \item Zero Conditional (facts and general truths)
    \item First Conditional (real future possibilities)
    \item Second Conditional (hypothetical present/future)
    \item Third Conditional (hypothetical past)
    \item Mixed conditionals
\end{itemize}

\section{Reading Context}
\begin{readingbox}[title=Dialogue: Planning and Regrets]
\textbf{Alice:} What will you do \textbf{if} it \textbf{rains} tomorrow?\\
\textbf{Bob:} \textbf{If} it \textbf{rains}, I\textbf{'ll stay} home and watch TV.\\
\textbf{Alice:} That's boring! \textbf{If} I \textbf{were} you, I\textbf{'d go} to the cinema instead.\\
\textbf{Bob:} Good idea. I wish I had more money though. \textbf{If} I \textbf{had} \$1 million, I\textbf{'d travel} the world!\\
\textbf{Alice:} Me too! By the way, did you pass your exam?\\
\textbf{Bob:} No, I failed. \textbf{If} I \textbf{had studied} harder, I \textbf{would have passed}.\\
\textbf{Alice:} That's a shame. You'll do better next time.
\end{readingbox}

\section{Conditional Overview}

Conditional sentences have two parts:
\begin{itemize}
    \item \textbf{IF-clause:} The condition
    \item \textbf{Main clause:} The result
\end{itemize}

\begin{center}
\Large \textbf{If} + condition, result \quad OR \quad result \textbf{if} + condition
\end{center}

\section{Zero Conditional: General Facts and Truths}

Use for things that are \textbf{always true} (scientific facts, habits, instructions).

\begin{grammarbox}[title=Zero Conditional Structure]
\textbf{Structure:} If + \textbf{present simple}, \textbf{present simple}

\textbf{Examples:}
\begin{itemize}
    \item \textbf{If} you \textbf{heat} water to 100°C, it \textbf{boils}. \trans{Si calientas agua...}
    \item \textbf{If} I \textbf{don't sleep} well, I \textbf{feel} tired. \trans{Si no duermo bien...}
    \item You \textbf{get} wet \textbf{if} it \textbf{rains}. \trans{Te mojas si llueve}
\end{itemize}

\textbf{Use:} Scientific facts, general truths, automatic results
\end{grammarbox}

\section{First Conditional: Real Future Possibilities}

Use for \textbf{real situations} that are likely to happen in the future.

\begin{grammarbox}[title=First Conditional Structure]
\textbf{Structure:} If + \textbf{present simple}, \textbf{will} + base verb

\textbf{Examples:}
\begin{itemize}
    \item \textbf{If} it \textbf{rains} tomorrow, I\textbf{'ll stay} home. \trans{Si llueve mañana...}
    \item \textbf{If} she \textbf{studies} hard, she\textbf{'ll pass} the exam. \trans{Si estudia mucho...}
    \item You\textbf{'ll be} late \textbf{if} you \textbf{don't hurry}. \trans{Llegarás tarde si...}
\end{itemize}

\textbf{Use:} Real future possibilities, predictions, warnings, promises

\textbf{Other modals possible:} can, may, might, should
\begin{itemize}
    \item \textbf{If} you \textbf{finish} early, you \textbf{can} leave. \trans{Si terminas temprano, puedes irte}
\end{itemize}
\end{grammarbox}

\section{Second Conditional: Hypothetical Present/Future}

Use for \textbf{imaginary or unlikely situations} in present or future.

\begin{grammarbox}[title=Second Conditional Structure]
\textbf{Structure:} If + \textbf{past simple}, \textbf{would} + base verb

\textbf{Examples:}
\begin{itemize}
    \item \textbf{If} I \textbf{had} \$1 million, I\textbf{'d buy} a house. \trans{Si tuviera \$1 millón...}
    \item \textbf{If} I \textbf{were} you, I\textbf{'d talk} to her. \trans{Si yo fuera tú...}
    \item She\textbf{'d travel} more \textbf{if} she \textbf{had} time. \trans{Ella viajaría más si...}
\end{itemize}

\textbf{Use:} Unlikely or impossible situations, advice, dreams

\textbf{Important:} Use \textbf{were} for all persons with "be" (formal):
\begin{itemize}
    \item If I \textbf{were} rich... (NOT: If I was rich - informal)
\end{itemize}

\textbf{Other modals possible:} could, might
\begin{itemize}
    \item If I won the lottery, I \textbf{could} retire early.
\end{itemize}
\end{grammarbox}

\section{Third Conditional: Hypothetical Past}

Use for \textbf{imaginary situations in the past} (things that didn't happen). Often expresses regret.

\begin{grammarbox}[title=Third Conditional Structure]
\textbf{Structure:} If + \textbf{past perfect}, \textbf{would have} + past participle

\textbf{Examples:}
\begin{itemize}
    \item \textbf{If} I \textbf{had studied}, I \textbf{would have passed}. \trans{Si hubiera estudiado, habría aprobado}
    \item \textbf{If} she \textbf{had left} earlier, she \textbf{wouldn't have missed} the train. \trans{Si hubiera salido antes...}
    \item I \textbf{would have helped} you \textbf{if} you \textbf{had asked}. \trans{Te habría ayudado si hubieras preguntado}
\end{itemize}

\textbf{Use:} Regrets, criticisms, imagining different past outcomes

\textbf{Other modals possible:} could have, might have
\begin{itemize}
    \item If I had known, I \textbf{could have} helped. \trans{Si hubiera sabido, podría haber ayudado}
\end{itemize}
\end{grammarbox}

\section{Comparison Table}

\begin{table}[h]
\centering
\begin{tabular}{|l|p{4cm}|p{5cm}|}
\hline
\textbf{Type} & \textbf{Structure} & \textbf{Use / Example} \\
\hline
\textbf{Zero} & If + present, present & Facts: If you heat ice, it melts. \\
\hline
\textbf{First} & If + present, will + verb & Real future: If it rains, I'll stay home. \\
\hline
\textbf{Second} & If + past, would + verb & Unlikely/imaginary: If I were rich, I'd travel. \\
\hline
\textbf{Third} & If + past perfect, would have + V3 & Past regret: If I had known, I would have come. \\
\hline
\end{tabular}
\caption{Conditional types summary}
\end{table}

\section{First vs. Second Conditional}

\begin{examplebox}[title=Understanding the Difference]
\textbf{First Conditional (Real/Likely):}
\begin{itemize}
    \item If it \textbf{rains} tomorrow, I\textbf{'ll take} an umbrella. \trans{Si llueve mañana...}
    \item (I think it might really rain - it's possible)
\end{itemize}

\textbf{Second Conditional (Unreal/Unlikely):}
\begin{itemize}
    \item If it \textbf{rained} in the Sahara, plants \textbf{would grow}. \trans{Si lloviera en el Sahara...}
    \item (It's very unlikely or imaginary)
\end{itemize}
\end{examplebox}

\section{Mixed Conditionals}

Sometimes we mix conditional types when the time in each clause is different.

\begin{grammarbox}[title=Common Mixed Conditional]
\textbf{Past condition + Present result}

\textbf{Structure:} If + past perfect, would + base verb

\textbf{Example:}
\begin{itemize}
    \item \textbf{If} I \textbf{had studied} medicine (past), I \textbf{would be} a doctor now (present).
    \item \trans{Si hubiera estudiado medicina, ahora sería doctor}
\end{itemize}
\end{grammarbox}

\section{Other Conditional Words}

\begin{vocabbox}[title=Alternatives to IF]
\begin{itemize}
    \item \textbf{Unless} = if not
    \begin{itemize}
        \item I'll go \textbf{unless} it rains. (= if it doesn't rain)
    \end{itemize}
    \item \textbf{As long as / Provided (that)} = only if
    \begin{itemize}
        \item You can go \textbf{as long as} you're back by 10.
    \end{itemize}
    \item \textbf{In case} = because something might happen
    \begin{itemize}
        \item Take an umbrella \textbf{in case} it rains.
    \end{itemize}
\end{itemize}
\end{vocabbox}

\section{Practice Exercises}

\subsection{Exercise 1: Identify the Conditional Type}
Write 0, 1, 2, or 3 for each conditional type.

\begin{enumerate}
    \item If I see her, I'll tell her. (\_\_\_)
    \item If I were rich, I'd buy a yacht. (\_\_\_)
    \item If you heat water, it boils. (\_\_\_)
    \item If I had known, I would have come. (\_\_\_)
\end{enumerate}

\subsection{Exercise 2: Complete the Conditionals}
Fill in the correct form of the verb.

\begin{enumerate}
    \item If it \underline{\hspace{3cm}} (rain) tomorrow, I'll stay home.
    \item If I \underline{\hspace{3cm}} (be) you, I'd apologize.
    \item If you \underline{\hspace{3cm}} (heat) ice, it melts.
    \item If she \underline{\hspace{3cm}} (study) harder, she would have passed.
    \item If I \underline{\hspace{3cm}} (have) time, I'll call you.
\end{enumerate}

\subsection{Exercise 3: First or Second Conditional?}
Choose the correct form.

\begin{enumerate}
    \item If I (win / won) the lottery, I'd travel the world.
    \item If it (rains / rained) tomorrow, we'll cancel the picnic.
    \item If I (am / were) taller, I could be a basketball player.
    \item If you (study / studied) hard, you'll pass the exam.
\end{enumerate}

\subsection{Exercise 4: Rewrite with the Correct Conditional}
\begin{enumerate}
    \item I don't have money. I can't buy it. (2nd conditional)\\
    $\rightarrow$ If \underline{\hspace{8cm}}
    \item I didn't know. I didn't come. (3rd conditional)\\
    $\rightarrow$ If \underline{\hspace{8cm}}
    \item You don't ask. You don't receive. (Zero conditional)\\
    $\rightarrow$ If \underline{\hspace{8cm}}
\end{enumerate}

\subsection{Exercise 5: Correct the Errors}
\begin{enumerate}
    \item If I will see her, I'll tell her.
    \item If I would be rich, I'd travel.
    \item If I had money, I will buy it.
    \item If I would have known, I would have come.
\end{enumerate}

\subsection{Exercise 6: Personal Situations}
Complete these sentences about yourself.

\begin{enumerate}
    \item If I have time this weekend, I'll \underline{\hspace{6cm}}
    \item If I won \$1 million, I would \underline{\hspace{6cm}}
    \item If I had studied harder at school, I would have \underline{\hspace{6cm}}
\end{enumerate}

\subsection{Exercise 7: Writing Task}
Write 3 sentences for each conditional type (0, 1, 2, 3) about your life.

\begin{tcolorbox}[colback=white,height=8cm]
% Zero: If...
% First: If...
% Second: If...
% Third: If...
\end{tcolorbox}

\section{Key Takeaways}
\begin{itemize}
    \item \textbf{Zero:} If + present, present (facts)
    \item \textbf{First:} If + present, will (real future)
    \item \textbf{Second:} If + past, would (unlikely/imaginary)
    \item \textbf{Third:} If + past perfect, would have + V3 (past regrets)
    \item Never use "will" or "would" in the IF-clause
    \item Use "were" (not "was") in Second Conditional with "be"
    \item Third Conditional often expresses regret or criticism
\end{itemize}
