\chapter{Superstitions, Travel, and Prepositions}

\section{Lesson Objectives}
In this chapter, you will learn:
\begin{itemize}
    \item How to discuss superstitions and cultural beliefs
    \item Prepositions of place: in, at, and on
    \item Travel vocabulary and expressions
    \item How to form questions about past experiences
\end{itemize}

\section{Speaking Activity: Find Someone Who...}

This activity helps practice forming past tense questions and having conversations about recent experiences.

\begin{table}[h]
\centering
\begin{tabular}{|p{3.5cm}|p{4.5cm}|p{2cm}|p{3cm}|}
\hline
\textbf{Activity} & \textbf{Question} & \textbf{Person} & \textbf{Details} \\
\hline
Went to a different country & Did you go to a different country? & Jason & Went to Edinburgh  \\
\hline
Tried some new food & Did you try some new food? & Rose & Tried Japanese food  \\
\hline
Slept at a fancy hotel & Did you sleep at a fancy hotel? & Rose & Never did  \\
\hline
Had a party & Did you have a party? & Cinar & Christmas party \\
\hline
Visited another part of England & Did you visit another part of England? & Valentina & Edinburgh \\
\hline
Saw an amazing monument & Did you see an amazing monument? & Jason & None \\
\hline
Bought something expensive & Did you buy something expensive? & Jason & Shoes \\
\hline
Didn't leave Oxford & Did you leave Oxford? & Adrien & Left Oxford and went to Paris  \\
\hline
Went to the theatre/opera & Did you go to the theatre/opera? & Adrien & No \\
\hline
Visited a famous city & Did you visit a famous city? & Jason & London  \\
\hline
\end{tabular}
\caption{Find someone who... activity results}
\end{table}
\section{Grammar Focus: Prepositions IN, AT, and ON}

Prepositions \textit{in}, \textit{at}, and \textit{on} are used to indicate place and time, but each has specific rules.

\begin{grammarbox}[title=Preposition IN]
\textbf{Usage:}
\begin{itemize}
    \item Large or general spaces: cities, countries, rooms
    \item Enclosed vehicles: cars, helicopters, taxis
    \item Periods of time: months, years, centuries
\end{itemize}
\textbf{Examples:}
\begin{itemize}
    \item I live \textbf{in} London. \trans{Vivo en Londres}
    \item The keys are \textbf{in} the drawer. \trans{Las llaves están en el cajón}
    \item She arrived \textbf{in} 2020. \trans{Llegó en 2020}
\end{itemize}
\end{grammarbox}

\begin{grammarbox}[title=Preposition AT]
\textbf{Usage:}
\begin{itemize}
    \item Specific locations or points
    \item Events and gatherings
    \item Specific times
\end{itemize}
\textbf{Examples:}
\begin{itemize}
    \item She is \textbf{at} the bus stop. \trans{Ella está en la parada de autobús}
    \item We met \textbf{at} the entrance. \trans{Nos encontramos en la entrada}
    \item The meeting is \textbf{at} 3 PM. \trans{La reunión es a las 3 PM}
\end{itemize}
\end{grammarbox}

\begin{grammarbox}[title=Preposition ON]
\textbf{Usage:}
\begin{itemize}
    \item Surfaces
    \item Public transportation where you can walk (except motorcycles)
    \item Days and dates
\end{itemize}
\textbf{Examples:}
\begin{itemize}
    \item The book is \textbf{on} the table. \trans{El libro está sobre la mesa}
    \item He is \textbf{on} the train. \trans{Él está en el tren}
    \item I'm riding \textbf{on} a motorcycle. \trans{Voy en motocicleta}
    \item See you \textbf{on} Monday. \trans{Nos vemos el lunes}
\end{itemize}
\end{grammarbox}

\begin{table}[h]
\centering
\begin{tabular}{|l|l|l|}
\hline
\textbf{Preposition} & \textbf{Use} & \textbf{Example} \\
\hline
IN & Inside an area/space & in a room, in a car, in London \\
\hline
AT & Specific point & at the door, at 5 PM, at home \\
\hline
ON & On a surface & on the table, on the bus, on Monday \\
\hline
\end{tabular}
\caption{Quick reference for prepositions}
\end{table}


\section{Common Mistakes}

\begin{grammarbox}[title=Word Order in Embedded Questions]
When asking an indirect question, the word order changes:
\begin{itemize}
    \item \textcolor{red}{\textbf{Incorrect:}} I don't know what is the name.
    \item \textcolor{green!60!black}{\textbf{Correct:}} I don't know what the name is.
    \item \textcolor{green!60!black}{\textbf{Correct:}} I don't know the name.
\end{itemize}
\textbf{Rule:} In embedded questions, use statement word order (subject + verb), not question word order.
\end{grammarbox}


\section{Travel Vocabulary and Expressions}

\subsection{Travel Idioms and Expressions}

\begin{vocabbox}[title=Common Travel Expressions]
\begin{itemize}
    \item \keyterm{See how the mood takes me} \trans{Decidir según cómo me sienta}
    \begin{itemize}
        \item Make decisions depending on how you're feeling in the moment
        \item Example: "I don't have fixed plans; I'll just see how the mood takes me."
    \end{itemize}
    \item \keyterm{Off the beaten track} \trans{Fuera de lo común / Poco turístico}
    \begin{itemize}
        \item A place without many tourists; unexplored areas
        \item Example: "We prefer traveling to places off the beaten track."
    \end{itemize}
    \item \keyterm{Culture shock} \trans{Choque cultural}
    \begin{itemize}
        \item Feeling uncomfortable or disoriented in a new place or culture
        \item Example: "I experienced culture shock when I first arrived in Tokyo."
    \end{itemize}
    \item \keyterm{Watch our backs} \trans{Tener cuidado / Estar alerta}
    \begin{itemize}
        \item Be careful and aware of potential dangers
        \item Example: "We need to watch our backs in crowded tourist areas."
    \end{itemize}
    \item \keyterm{Travel light} \trans{Viajar con poco equipaje}
    \begin{itemize}
        \item Travel without much luggage or belongings
        \item Example: "I always travel light with just a backpack."
    \end{itemize}
    \item \keyterm{Travel on a shoestring} \trans{Viajar con presupuesto limitado}
    \begin{itemize}
        \item Not spend much money while traveling; budget travel
        \item Example: "As a student, I had to travel on a shoestring."
    \end{itemize}
    \item \keyterm{Got the travel bug} \trans{Adicto a viajar}
    \begin{itemize}
        \item Be addicted to traveling; have a strong desire to travel
        \item Example: "After my first trip abroad, I got the travel bug."
    \end{itemize}
    \item \keyterm{Culture vulture} \trans{Aficionado a la cultura}
    \begin{itemize}
        \item Someone who likes museums, art galleries, and cultural activities
        \item Example: "She's a real culture vulture; she visits every museum in town."
    \end{itemize}
    \item \keyterm{Feel right at home} \trans{Sentirse como en casa}
    \begin{itemize}
        \item Feel very comfortable in a new place
        \item Example: "The host family was so welcoming; I felt right at home."
    \end{itemize}
    \item \keyterm{Get up at the crack of dawn} \trans{Levantarse muy temprano}
    \begin{itemize}
        \item Wake up very early in the morning
        \item Example: "We got up at the crack of dawn to catch the sunrise."
    \end{itemize}
    \item \keyterm{Savour the local delicacies} \trans{Saborear las delicias locales}
    \begin{itemize}
        \item Enjoy and appreciate local food
        \item Example: "Don't forget to savour the local delicacies when you visit Italy."
    \end{itemize}
    \item \keyterm{Sit and watch the world go by} \trans{Sentarse y relajarse}
    \begin{itemize}
        \item Relax and observe without doing anything active
        \item Example: "I love to sit at a café and watch the world go by."
    \end{itemize}
\end{itemize}
\end{vocabbox}

\section{Superstitions and Cultural Beliefs}

\subsection{Key Vocabulary: Myth}

\begin{vocabbox}[title=Word Focus: Myth]
\textbf{Definition:} A traditional story, especially one concerning the early history of a people or explaining a natural or social phenomenon, and typically involving supernatural beings or events.
\trans{Mito}

\textbf{Examples:}
\begin{itemize}
    \item The myth of Atlantis tells the story of a lost civilization.
    \item Many cultures have myths about creation and the origins of the world.
    \item Greek myths often feature gods and heroes.
\end{itemize}

\textbf{Common Collocations:}
\begin{itemize}
    \item ancient myth \trans{mito antiguo}
    \item popular myth \trans{mito popular}
    \item urban myth \trans{mito urbano}
    \item read a myth \trans{leer un mito}
    \item tell a myth \trans{contar un mito}
\end{itemize}
\end{vocabbox}

\subsection{Superstition Vocabulary}

\begin{vocabbox}[title=Key Terms for Discussing Superstitions]
\begin{itemize}
    \item \keyterm{Old wives' tales} \trans{Cuentos de viejas / Supersticiones populares}
    \begin{itemize}
        \item Traditional beliefs or superstitions, often passed down through generations
        \item Example: "Breaking a mirror brings seven years of bad luck is an old wives' tale."
    \end{itemize}
    \item \keyterm{Association} \trans{Asociación}
    \begin{itemize}
        \item A mental connection between ideas or things
        \item Example: "Many superstitions are based on associations between unrelated events."
    \end{itemize}
    \item \keyterm{Blessing} \trans{Bendición}
    \begin{itemize}
        \item God's favor and protection; something that brings good fortune
        \item Example: "Some people see finding a penny as a blessing."
    \end{itemize}
    \item \keyterm{Coincidence} \trans{Coincidencia}
    \begin{itemize}
        \item A remarkable occurrence of events at the same time by chance
        \item Example: "It was just a coincidence that it rained on Friday the 13th."
    \end{itemize}
    \item \keyterm{Irrational} \trans{Irracional}
    \begin{itemize}
        \item Not logical or reasonable
        \item Example: "Believing in superstitions is often considered irrational."
    \end{itemize}
    \item \keyterm{Remnants} \trans{Remanentes / Restos}
    \begin{itemize}
        \item Small remaining parts of something
        \item Example: "These superstitions are remnants of ancient beliefs."
    \end{itemize}
    \item \keyterm{Illusion} \trans{Ilusión}
    \begin{itemize}
        \item A false perception or belief
        \item Example: "The idea that lucky charms work is just an illusion."
    \end{itemize}
    \item \keyterm{Lightning strike} \trans{Rayo / Caída de un rayo}
    \begin{itemize}
        \item When lightning hits something
        \item Example: "The chances of a lightning strike are very small."
    \end{itemize}
\end{itemize}
\end{vocabbox}

\subsection{Expressing Beliefs and Doubts about Superstitions}

\begin{examplebox}[title=Asking About Superstitions]
\textbf{Ways to ask:}
\begin{itemize}
    \item Do you \textbf{buy into} superstitions? \trans{¿Crees en las supersticiones?}
    \item Do you \textbf{fall for} superstitions? \trans{¿Te crees las supersticiones?}
    \item Do you \textbf{believe in} superstitions? \trans{¿Crees en las supersticiones?}
\end{itemize}
\end{examplebox}

\begin{grammarbox}[title=Expressing Doubt and Disbelief]
\textbf{Formal expressions:}
\begin{itemize}
    \item I have some \textbf{reservations}. \trans{Tengo algunas reservas/dudas}
    \item I have some \textbf{doubts}. \trans{Tengo algunas dudas}
    \item I \textbf{doubt that}... \trans{Dudo que...}
    \item I doubt that's true. \trans{Dudo que eso sea cierto}
\end{itemize}

\textbf{Informal expressions:}
\begin{itemize}
    \item You're \textbf{pulling my leg}! \trans{¡Me estás tomando el pelo!}
    \begin{itemize}
        \item Meaning: You're joking, right?
    \end{itemize}
    \item That's \textbf{ridiculous}! \trans{¡Eso es ridículo!}
    \begin{itemize}
        \item Meaning: That's absurd
    \end{itemize}
    \item Oh, \textbf{give me a break}! \trans{¡Venga ya! / ¡Por favor!}
    \begin{itemize}
        \item Used to express disbelief or annoyance
    \end{itemize}
    \item \textbf{Funny that}... \trans{Qué curioso que...}
    \begin{itemize}
        \item Meaning: That's interesting or ironic
    \end{itemize}
    \item That's \textbf{silly}! \trans{¡Eso es tonto!}
    \begin{itemize}
        \item Meaning: That's foolish or absurd
    \end{itemize}
\end{itemize}
\end{grammarbox}

\section{Grammar Note: Fewer vs Less}

\begin{grammarbox}[title=Countable vs Uncountable]
\textbf{Fewer} is used with \textit{countable} nouns:
\begin{itemize}
    \item There are \textbf{fewer} people here today.
    \item We have \textbf{fewer} chairs than we need.
    \item I made \textbf{fewer} mistakes this time.
\end{itemize}

\textbf{Less} is used with \textit{uncountable} nouns:
\begin{itemize}
    \item There is \textbf{less} water in the bottle.
    \item We have \textbf{less} time than before.
    \item She has \textbf{less} experience.
\end{itemize}

\textbf{Memory tip:} If you can count it, use \textbf{fewer}. If you can't count it, use \textbf{less}.
\end{grammarbox}

\section{Additional Useful Expressions}

\begin{examplebox}[title=General Expressions]
\begin{itemize}
    \item \textbf{Per se} \trans{En sí mismo / Intrinecamente}
    \begin{itemize}
        \item Meaning: Intrinsically, by itself
        \item Example: "The book isn't per se bad, but it's not my favorite genre."
    \end{itemize}
\end{itemize}
\end{examplebox}







