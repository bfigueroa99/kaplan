\chapter{Writing Skills and Common Errors}

\begin{center}
\begin{tabular}{ccc}
\cefrlevel{B2-C1} & \textbf{Study Time:} 3-4 hours & \textbf{Difficulty:} ⭐⭐⭐⭐☆
\end{tabular}
\end{center}

\section{Lesson Objectives}
In this chapter, you will learn:
\begin{itemize}
    \item How to plan, write, and check your written work
    \item Common errors in English writing and how to avoid them
    \item Linking words to connect ideas
    \item Common preposition collocations
\end{itemize}

\section{Reading Context}
\begin{readingbox}[title=Dialogue: Reviewing a Report]
\textbf{Editor:} I read your draft. It's good, but there are a few errors.\\
\textbf{Writer:} Really? I thought I checked it carefully.\\
\textbf{Editor:} You missed some articles. For example, you wrote "I have car" instead of "I have \textbf{a} car".\\
\textbf{Writer:} Oh, I see. I'm always worried \textbf{about} making mistakes with articles.\\
\textbf{Editor:} Don't worry. Also, use more linking words. \textbf{However}, the content is excellent.\\
\textbf{Writer:} Thanks. I'll give it another look. I'm interested \textbf{in} improving my writing.\\
\textbf{Editor:} Great. Remember to plan before you write next time.
\end{readingbox}

\section{Key Concepts: The Writing Process}

Good writing follows three key stages:

\begin{vocabbox}[title=The 3 Stages]
\begin{enumerate}
    \item \keyterm{Planning} (17\% of time): Brainstorm ideas, organize structure, make notes.
    \item \keyterm{Writing} (60\% of time): Write your first draft quickly, focus on content.
    \item \keyterm{Checking} (23\% of time): Proofread for errors, revise and improve.
\end{enumerate}
\end{vocabbox}

\section{Grammar Focus: Common Errors}

\begin{spanishbox}[title=Para Hispanohablantes: Common Spanish Speaker Errors]
⚠️ \textbf{Errores típicos de hispanohablantes en escritura:}

\begin{itemize}
    \item \textbf{Artículos olvidados:}
    \begin{itemize}
        \item ❌ I have car (español: "tengo coche")
        \item ✓ I have \textbf{a} car
    \end{itemize}
    \item \textbf{Preposiciones dependientes del verbo:}
    \begin{itemize}
        \item ❌ I'm worried OF mistakes (español: "preocupado DE")
        \item ✓ I'm worried \textbf{ABOUT} mistakes
        \item ❌ I'm interested ON writing (español: "interesado EN")
        \item ✓ I'm interested \textbf{IN} writing
    \end{itemize}
    \item \textbf{Conectores (Linking Words):}
    \begin{itemize}
        \item Español usa menos conectores
        \item Inglés REQUIERE conectores: However, Moreover, Furthermore, In addition
    \end{itemize}
\end{itemize}

💡 \textbf{Truco:} Cada verbo/adjetivo inglés tiene SU preposición específica. ¡Memorízalas juntas!
\end{spanishbox}

\begin{grammarbox}[title=Error Correction Guide]
\textbf{1. Missing Articles:}
\begin{itemize}
    \item Incorrect: I have car.
    \item Correct: I have \textbf{a} car.
\end{itemize}

\textbf{2. Wrong Prepositions:}
\begin{itemize}
    \item Incorrect: I'm interested for music.
    \item Correct: I'm interested \textbf{in} music.
\end{itemize}

\textbf{3. Run-on Sentences:}
\begin{itemize}
    \item Incorrect: I went home I was tired.
    \item Correct: I went home \textbf{because} I was tired.
\end{itemize}
\end{grammarbox}

\section{Vocabulary: Linking Words and Prepositions}

\begin{vocabbox}[title=Linking Words]
\begin{itemize}
    \item \textbf{Addition:} Furthermore, Moreover, Also \trans{Además}
    \item \textbf{Contrast:} However, Although, On the other hand \trans{Sin embargo}
    \item \textbf{Result:} Therefore, Consequently, As a result \trans{Por lo tanto}
    \item \textbf{Sequence:} First, Then, Next, Finally \trans{Primero, Luego...}
\end{itemize}
\end{vocabbox}

\begin{table}[h]
\centering
\begin{tabular}{|l|p{4cm}|p{5cm}|}
\hline
\textbf{Collocation} & \textbf{Spanish} & \textbf{Example} \\
\hline
Interested \textbf{in} & Interesado en & I'm interested \textbf{in} art. \\
\hline
Good \textbf{at} & Bueno en & She's good \textbf{at} math. \\
\hline
Responsible \textbf{for} & Responsable de & He's responsible \textbf{for} it. \\
\hline
Depend \textbf{on} & Depender de & It depends \textbf{on} you. \\
\hline
Afraid \textbf{of} & Miedo de & I'm afraid \textbf{of} spiders. \\
\hline
\end{tabular}
\caption{Common Preposition Collocations}
\end{table}

\section{Practice Exercises}

\subsection{Exercise 1: Find and Correct the Errors}
Each sentence has one error. Find and correct it.

\begin{enumerate}
    \item I have been living in London since 3 years.
    \item She is very good in cooking.
    \item I went to store and bought some milk.
    \item However I think we should wait.
\end{enumerate}

\subsection{Exercise 2: Complete with Linking Words}
Fill in the blanks with: \textit{However, Therefore, Furthermore, First}.

\begin{enumerate}
    \item I studied hard. \underline{\hspace{2cm}}, I passed the exam.
    \item The food was delicious. \underline{\hspace{2cm}}, it was expensive.
    \item \underline{\hspace{2cm}}, we need to buy tickets. Then, we can enter.
\end{enumerate}

\subsection{Exercise 3: Writing Practice}
Write a short paragraph (50-80 words) about your city using at least 2 linking words and 2 preposition collocations.

\begin{tcolorbox}[colback=white,height=5cm]
% Write your paragraph here
\end{tcolorbox}

\section{Key Takeaways}
\begin{itemize}
    \item Always plan before writing and leave time for checking.
    \item Watch out for missing articles and wrong prepositions.
    \item Use linking words (However, Therefore) to connect ideas.
    \item Memorize collocations: interested IN, good AT, depend ON.
\end{itemize}

\section{Advanced Writing Techniques: Rephrasal and Paraphrasal}

\subsection{Rephrasal}

Rephrasal is the act of expressing the same idea using different words or phrases. It is often used to avoid repetition, clarify meaning, or simplify complex ideas.

\textbf{When to use rephrasal:}
\begin{itemize}
    \item Explain with other words so someone understands better
    \item Avoid repeating the same words
    \item Make your writing more interesting
    \item Simplification of complex ideas
    \item Clarify meaning
\end{itemize}

\textbf{Example:}
\begin{itemize}
    \item Original: "My best friend is such a good laugh"
    \item Rephrasal: "My best friend is so funny"
\end{itemize}

\subsection{Paraphrasal}

Paraphrasal is the act of rephrasing or restating a sentence or passage using different words while maintaining the original meaning. It is often used in academic writing, summarizing, and note-taking.

\textbf{When to use paraphrasal:}
\begin{itemize}
    \item To summarize information
    \item To avoid plagiarism
    \item To clarify complex ideas
    \item To understand better
    \item Literature review
\end{itemize}

\textbf{Example:}
\begin{itemize}
    \item Original: "I'm always falling out with my best friend but usually we make up"
    \item Paraphrasal: "I'm always fighting with my best friend but usually we become friends again"
\end{itemize}

\section{Online Practice}
\begin{tcolorbox}[colback=blue!5,colframe=blue!40!black,title=Resources to Practice]
Online PracticeReinforce what you have learned with these interactive exercises:
\begin{itemize}
    \item \textbf{Writing Skills:} \url{https://learnenglish.britishcouncil.org/skills/writing}
    \item \textbf{Linking Words:} \url{https://www.englishclub.com/writing/linking-words.htm}
    \item \textbf{Common Errors:} \url{https://www.perfect-english-grammar.com/common-mistakes.html}
    \item \textbf{Preposition Collocations:} \url{https://test-english.com/grammar-points/a2-b1/dependent-prepositions/}
\end{itemize}
\end{tcolorbox}