\chapter{Relative Clauses: WHO, WHICH, THAT, WHERE, WHY}

\begin{center}
\begin{tabular}{ccc}
\cefrlevel{B2-C1} & \textbf{Study Time:} 3-4 hours & \textbf{Difficulty:} ⭐⭐⭐⭐☆
\end{tabular}
\end{center}

\section{Lesson Objectives}
In this chapter, you will learn:
\begin{itemize}
    \item How to form and use relative clauses
    \item The difference between WHO, WHICH, and THAT
    \item WHERE and WHY in relative clauses
    \item Defining vs. non-defining (restrictive vs. non-restrictive) clauses
    \item How to combine sentences using relative clauses
    \item Common punctuation rules for relative clauses
\end{itemize}

\section{Reading Context}

\begin{readingbox}[title=Dialogue: Finding the Perfect Office]
\textbf{Manager:} We need an office \textbf{that} is modern and affordable.\\
\textbf{Realtor:} I know a building \textbf{which} has excellent facilities. It's in the downtown area, \textbf{where} there's great public transport.\\
\textbf{Manager:} Great! Who is the landlord?\\
\textbf{Realtor:} Mr. Chen, \textbf{who} is very professional, owns it. He's someone \textbf{that} many companies trust.\\
\textbf{Manager:} That sounds promising. Is there a reason \textbf{why} the rent is so reasonable?\\
\textbf{Realtor:} Yes. The building is older, \textbf{which} is the main reason. But it's been recently renovated, \textbf{which} adds great value.\\
\textbf{Manager:} Let's schedule a visit next week. I want to see it before I make a decision.
\end{readingbox}

\begin{comparisonbox}[title=🇪🇸 Pronombres Relativos: "QUE" en Español vs WHO/WHICH/THAT en Inglés]
\textbf{⚠️ En español usamos "QUE" para todo, en inglés hay que elegir}

\textbf{1. Español - "QUE" es universal:}
\begin{itemize}
    \item "La persona \textit{que} me ayudó" (persona)
    \item "El libro \textit{que} leí" (cosa)
    \item "La ciudad \textit{que} visité" (lugar)
\end{itemize}

\textbf{2. Inglés - WHO para personas, WHICH para cosas:}
\begin{itemize}
    \item The person \textbf{WHO} helped me (persona)
    \item The book \textbf{WHICH} I read (cosa)
    \item The city \textbf{WHERE} I visited (mejor: "that I visited")
\end{itemize}

\textbf{3. THAT - puede reemplazar WHO o WHICH (informal):}
\begin{itemize}
    \item The person \textbf{THAT} helped me ✓ (menos formal que WHO)
    \item The book \textbf{THAT} I read ✓ (menos formal que WHICH)
\end{itemize}

\textbf{4. "Quien" español ≠ WHO inglés:}
\begin{itemize}
    \item Español: "\textit{Quien} estudia, aprende" (indefinido, sin antecedente)
    \item English: "\textbf{He who} studies, learns" (arcaico/formal)
    \item Modern English: "The person \textbf{who} studies learns" (necesita antecedente)
\end{itemize}

\textbf{5. Comas en cláusulas no-restrictivas:}
\begin{itemize}
    \item Sin comas (defining): "The teacher WHO wrote the book" (específico, esencial)
    \item Con comas (non-defining): "Sarah, WHO is my friend, lives here" (extra info)
    \item En español también usamos comas pero menos estrictamente
\end{itemize}
\end{comparisonbox}

\section{Grammar Focus: Types of Relative Clauses}

\begin{grammarbox}[title=What are Relative Clauses?]
A \textbf{relative clause} adds extra information about a noun. It starts with a relative pronoun (who, which, that, where, why) and contains a verb.

\textbf{Basic Structure:}
\begin{center}
Noun + Relative Pronoun + Additional Information
\end{center}

\textbf{Examples:}
\begin{itemize}
    \item The person \textbf{who} helped me is very kind.
    \item The book \textbf{that} I read was interesting.
    \item The city \textbf{where} I was born is beautiful.
\end{itemize}
\end{grammarbox}

\subsection{WHO - For People}

\begin{grammarbox}[title=Relative Pronoun: WHO]
Use WHO for \textbf{people} when the pronoun replaces the subject of the clause.

\textbf{Structure:} Person + WHO + verb

\textbf{Examples:}
\begin{itemize}
    \item The teacher \textbf{who} wrote the book is famous. \trans{El profesor que escribió el libro es famoso}
    \item Sarah, \textbf{who} is my colleague, speaks five languages. \trans{Sarah, quien es mi colega, habla cinco idiomas}
    \item The man \textbf{who} called yesterday will arrive tomorrow. \trans{El hombre que llamó ayer llegará mañana}
\end{itemize}

\textbf{Can WHO be omitted?} NO. WHO as a subject pronoun is essential.
\begin{itemize}
    \item \textbf{[INCORRECT]} "The teacher wrote the book is famous."
    \item ✅ "The teacher \textbf{who} wrote the book is famous."
\end{itemize}
\end{grammarbox}

\subsection{WHICH - For Things/Animals}

\begin{grammarbox}[title=Relative Pronoun: WHICH]
Use WHICH for \textbf{things or animals}, or to refer to the entire previous clause.

\textbf{Structure:} Thing + WHICH + verb

\textbf{Examples:}
\begin{itemize}
    \item The car \textbf{which} I bought last year is still working well. \trans{El coche que compré el año pasado sigue funcionando bien}
    \item The project, \textbf{which} started last month, is progressing well. \trans{El proyecto, que comenzó el mes pasado, progresa bien}
    \item The dog \textbf{which} belonged to Sarah was friendly. \trans{El perro que pertenecía a Sarah era amigable}
\end{itemize}

\textbf{WHICH for entire clause:}
\begin{itemize}
    \item She didn't pass the exam, \textbf{which} surprised everyone.
    \trans{No aprobó el examen, lo cual sorprendió a todos}
\end{itemize}

\textbf{Punctuation:} Use commas before WHICH when it provides extra (non-essential) information.
\end{grammarbox}

\subsection{THAT - For People or Things (More Informal)}

\begin{grammarbox}[title=Relative Pronoun: THAT]
Use THAT for \textbf{people or things}. It's more informal than WHO/WHICH but increasingly common.

\textbf{Structure:} Noun + THAT + verb

\textbf{Examples:}
\begin{itemize}
    \item The person \textbf{that} helped me is very kind. \trans{La persona que me ayudó es muy amable}
    \item The laptop \textbf{that} I use is old. \trans{La portátil que uso es vieja}
    \item The restaurant \textbf{that} we visited was excellent. \trans{El restaurante que visitamos era excelente}
\end{itemize}

\textbf{When to use THAT:}
\begin{itemize}
    \item In defining (essential) clauses
    \item In informal writing or speech
    \item After superlatives: "the best movie \textbf{that} I've seen"
    \item After words like "all," "only," "every": "all the books \textbf{that} we have"
\end{itemize}

\textbf{Note:} THAT is often omitted in informal English when it's the object:
\begin{itemize}
    \item The book \textbf{(that)} I read was great. ✅ (THAT can be omitted)
    \item The book \textbf{that} was on my desk is gone. \textbf{[INCORRECT]} (THAT cannot be omitted here — it's the subject)
\end{itemize}
\end{grammarbox}

\subsection{WHERE - For Places}

\begin{grammarbox}[title=Relative Pronoun: WHERE]
Use WHERE for \textbf{places or locations}.

\textbf{Structure:} Place + WHERE + verb

\textbf{Examples:}
\begin{itemize}
    \item The city \textbf{where} I grew up is in the south. \trans{La ciudad donde crecí está en el sur}
    \item That's the restaurant \textbf{where} we had dinner. \trans{Ese es el restaurante donde cenamos}
    \item The office \textbf{where} she works is very modern. \trans{La oficina donde trabaja es muy moderna}
\end{itemize}

\textbf{Equivalent forms:}
\begin{itemize}
    \item The city \textbf{where} I grew up = The city \textbf{that} I grew up in
    \item The restaurant \textbf{where} we dined = The restaurant \textbf{(that)} we dined in
\end{itemize}
\end{grammarbox}

\subsection{WHY - For Reasons}

\begin{grammarbox}[title=Relative Pronoun: WHY]
Use WHY for \textbf{reasons or causes}. It's less common than the others.

\textbf{Structure:} Reason + WHY + verb

\textbf{Examples:}
\begin{itemize}
    \item The reason \textbf{why} she was late is unknown. \trans{La razón por la que llegó tarde es desconocida}
    \item I don't understand the reason \textbf{why} he left. \trans{No entiendo la razón por la que se fue}
    \item That's why I came — I found the reason. \trans{Por eso vine — encontré la razón}
\end{itemize}
    \textbf{Alternative:} Can also use "the reason (that/which):
    \begin{itemize}
        \item The reason \textbf{(that)} she was late is unknown.
    \end{itemize}
\end{grammarbox}

\subsection{WHOM - For People (Object)}

\begin{grammarbox}[title=Relative Pronoun: WHOM]
Use WHOM for \textbf{people} when the pronoun is the \textbf{object} of the verb or preposition. It is more formal than WHO.

\textbf{Structure:} Person + WHOM + Subject + Verb

\textbf{Examples:}
\begin{itemize}
    \item The woman \textbf{whom} I met at the conference is a scientist. \trans{La mujer a quien conocí...}
    \item To \textbf{whom} it may concern. \trans{A quien corresponda.}
    \item The student \textbf{whom} the teacher praised was happy.
\end{itemize}

\textbf{Note:} In modern spoken English, WHO is often used instead of WHOM, or it is omitted completely.
\begin{itemize}
    \item Formal: The man \textbf{whom} I saw...
    \item Informal: The man \textbf{who} I saw... / The man I saw...
\end{itemize}
\end{grammarbox}

\subsection{WHOSE - For Possession}

\begin{grammarbox}[title=Relative Pronoun: WHOSE]
Use WHOSE to show \textbf{possession} or relationship. It replaces "his", "her", "their", or "its".

\textbf{Structure:} Person/Thing + WHOSE + Noun + Verb

\textbf{Examples:}
\begin{itemize}
    \item That is the girl \textbf{whose} brother is famous. \trans{Esa es la chica cuyo hermano es famoso.}
    \item I live in a house \textbf{whose} roof is red. \trans{Vivo en una casa cuyo techo es rojo.}
    \item The man \textbf{whose} phone rang was embarrassed.
\end{itemize}
\end{grammarbox}

\section{Comparison of Relative Pronouns}

\subsection{Who vs. That vs. Whom}

\begin{table}[h]
\centering
\begin{tabular}{|l|p{4cm}|p{6cm}|}
\hline
\textbf{Pronoun} & \textbf{Function} & \textbf{Example} \\
\hline
\textbf{WHO} & Subject for people & The man \textbf{who} is talking is my brother. \\
\hline
\textbf{THAT} & Subject/Object for people or things (Informal) & The book \textbf{that} you lent me was fascinating. \\
\hline
\textbf{WHOM} & Object for people (Formal) & The woman \textbf{whom} I met is a renowned scientist. \\
\hline
\end{tabular}
\end{table}

\subsection{That vs. Which}

\begin{table}[h]
\centering
\begin{tabular}{|l|p{4cm}|p{6cm}|}
\hline
\textbf{Pronoun} & \textbf{Usage} & \textbf{Example} \\
\hline
\textbf{THAT} & Restrictive (Essential info) & The car \textbf{that} I bought last year is very reliable. \\
\hline
\textbf{WHICH} & Non-restrictive (Extra info, with commas) & My car, \textbf{which} I bought last year, is very reliable. \\
\hline
\end{tabular}
\end{table}

\section{Defining vs. Non-Defining Relative Clauses}

\begin{table}[h]
\centering
\begin{tabular}{|p{3.2cm}|p{3.2cm}|p{3.2cm}|}
\hline
\textbf{Aspect} & \textbf{DEFINING (Essential)} & \textbf{NON-DEFINING (Extra Info)} \\
\hline
Purpose & Identifies which person/thing & Adds extra information \\
\hline
Punctuation & NO commas & Use commas \\
\hline
Can be omitted? & NO (meaning changes) & YES (still understood) \\
\hline
Example & The student \textbf{who} passed is happy. & My brother, \textbf{who} lives in London, is visiting. \\
\hline
Explanation & Specifies WHICH student & My brother's location is extra info \\
\hline
\end{tabular}
\caption{Defining vs. Non-Defining Relative Clauses}
\end{table}

\section{Practice Exercises}

\subsection{Exercise 1: Fill in the Relative Pronoun}
Complete the sentences with WHO, WHICH, THAT, WHERE, or WHY:

\begin{enumerate}
    \item The person \underline{\hspace{2cm}} called me yesterday is my boss.
    \item The book \underline{\hspace{2cm}} I'm reading is very interesting.
    \item I don't know the reason \underline{\hspace{2cm}} she left.
    \item The city \underline{\hspace{2cm}} we vacationed was beautiful.
    \item My father, \underline{\hspace{2cm}} is a doctor, works at the hospital.
\end{enumerate}

\subsection{Exercise 2: Combine Two Sentences}
Combine each pair using a relative clause:

\begin{enumerate}
    \item The car is red. It belongs to Tom.
    \begin{itemize}
        \item Answer: \underline{\hspace{6cm}}
    \end{itemize}
    \item The woman spoke English. She was from Australia.
    \begin{itemize}
        \item Answer: \underline{\hspace{6cm}}
    \end{itemize}
    \item London is a big city. I visited it last summer.
    \begin{itemize}
        \item Answer: \underline{\hspace{6cm}}
    \end{itemize}
\end{enumerate}


\subsection{Exercise 3: Identify Type of Clause}
Decide if the relative clause is DEFINING (D) or NON-DEFINING (ND):

\begin{enumerate}
    \item The laptop \textbf{that} costs £500 is on sale. \underline{\hspace{2cm}}
    \item My sister, \textbf{who} lives in Paris, is an artist. \underline{\hspace{2cm}}
    \item The people \textbf{who} arrived late missed the beginning. \underline{\hspace{2cm}}
\end{enumerate}

\subsection{Exercise 4: Write Your Own Sentences}
Write sentences about yourself using relative clauses:

\begin{enumerate}
    \item About a person you know (use WHO or THAT):
    \item About a place you like (use WHERE):
    \item About something you own (use WHICH):
    \item About a reason (use WHY):
\end{enumerate}

\subsection{Exercise 5: Speaking Practice}
Talk about or write answers for the following topics:

\begin{enumerate}
    \item Tell me about someone \textbf{whom} you look up to.
    \item Tell me about a place \textbf{where} you would love to live one day.
    \item Tell me about a song \textbf{which} you love and a song \textbf{that} you dislike.
    \item Tell me about a thing \textbf{that} you could never live without.
    \item Tell me about a time \textbf{when} you did something that you will remember forever.
\end{enumerate}

\section{\textbf{[INCORRECT]} Common Mistakes}

\begin{tcolorbox}[colback=red!5, colframe=red!30, title=Relative Clause Errors]

\textbf{Mistake 1:} Using WHO for things or WHICH for people
\begin{itemize}
    \item \textbf{[INCORRECT]} "The car which I drive is fast." (should be WHICH)
    \item ✅ "The car \textbf{which} I drive is fast." (correct for things) OR
    \item ✅ "The man \textbf{who} I met is kind." (correct for people)
\end{itemize}

\textbf{Mistake 2:} Forgetting to use a relative pronoun in formal contexts
\begin{itemize}
    \item \textbf{[INCORRECT]} "The book I read was good." (informal, but correct)
    \item ✅ "The book \textbf{that} I read was good." (more formal)
\end{itemize}

\textbf{Mistake 3:} Using commas incorrectly (confusing defining and non-defining)
\begin{itemize}
    \item \textbf{[INCORRECT]} "The person, who helped me, is very kind." (should be no commas if defining)
    \item ✅ "The person \textbf{who} helped me is very kind." (defining, no commas)
    \item ✅ "My mother, \textbf{who} helped me, is very kind." (non-defining, with commas)
\end{itemize}

\textbf{Mistake 4:} Double subjects
\begin{itemize}
    \item \textbf{[INCORRECT]} "The person who he met was nice."
    \item ✅ "The person \textbf{who} met was nice." (if WHO is subject)
    \item ✅ "The person \textbf{(that)} he met was nice." (if he is subject)
\end{itemize}

\end{tcolorbox}

\section{🇬🇧 British English Notes}

In British English, relative clauses have specific preferences:

\begin{itemize}
    \item \textbf{WHICH vs. THAT:} British English prefers WHICH for non-defining clauses and is more likely to use WHO/WHICH formally
    \begin{itemize}
        \item "The book, \textbf{which} is excellent, ..." (British)
        \item "The book \textbf{that} is excellent ..." (American — less formal)
    \end{itemize}
    \item \textbf{Relative pronoun omission:} British English is more formal and less likely to omit the relative pronoun
    \begin{itemize}
        \item "The person \textbf{(that)} I met" — British more likely to include
    \end{itemize}
    \item \textbf{Preposition placement:} British English often places the preposition at the end
    \begin{itemize}
        \item "The office I work in" (British) vs. "The office in which I work" (very formal)
    \end{itemize}
\end{itemize}

\section{Key Takeaways}

\begin{itemize}
    \item \textbf{WHO} = for people
    \item \textbf{WHICH} = for things/animals or entire clauses
    \item \textbf{THAT} = for people or things (informal, often in defining clauses)
    \item \textbf{WHERE} = for places
    \item \textbf{WHY} = for reasons (less common)
    \item \textbf{Defining clauses} = essential information, no commas
    \item \textbf{Non-defining clauses} = extra information, use commas
    \item Relative pronouns cannot be repeated (no "the person who he...")
    \item In informal English, THAT/WHO can sometimes be omitted when they're objects
\end{itemize}
