\chapter{Mishaps and Past Experiences}

\begin{center}
\begin{tabular}{ccc}
\cefrlevel{B1-B2} & \textbf{Study Time:} 2-3 hours & \textbf{Difficulty:} ⭐⭐⭐☆☆
\end{tabular}
\end{center}

\section{Lesson Objectives}
In this chapter, you will learn:
\begin{itemize}
    \item Vocabulary for describing accidents and mishaps
    \item How to use adjectives ending in -ed and -ing correctly
    \item Structures for narrating past experiences
    \item Collocations with common verbs like \textit{lose, miss, spill}
\end{itemize}

\section{Reading Context}
\begin{readingbox}[title=Dialogue: A Terrible Morning]
\textbf{Sarah:} You look stressed. What happened?\\
\textbf{Mark:} I had a terrible morning. First, I \textbf{overslept} because my alarm didn't go off.\\
\textbf{Sarah:} Oh no. Did you miss your bus?\\
\textbf{Mark:} Yes! I ran to the bus stop, but I \textbf{missed} it by one minute. It was so \textbf{frustrating}.\\
\textbf{Sarah:} So how did you get here?\\
\textbf{Mark:} I had to take a taxi. But then, I realized I had \textbf{left} my wallet at home.\\
\textbf{Sarah:} That's \textbf{embarrassing}. What did you do?\\
\textbf{Mark:} The driver was nice. I paid him with my phone app. But then I \textbf{spilled} coffee on my shirt!\\
\textbf{Sarah:} Wow. You really need a break.
\end{readingbox}

\begin{comparisonbox}[title=🇪🇸 Adjetivos -ED/-ING: Español NO distingue - Inglés SÍ]
\textbf{⚠️ Español usa el mismo adjetivo | Inglés cambia -ED vs -ING}

\textbf{1. Español - mismo adjetivo:}
\begin{itemize}
    \item "Estoy aburrido" (yo siento)
    \item "Es aburrido" (algo causa)
    \item Mismo adjetivo "aburrido" para ambos
\end{itemize}

\textbf{2. Inglés - DOS formas diferentes:}
\begin{itemize}
    \item \textbf{-ED} = La PERSONA siente la emoción
    \begin{itemize}
        \item "I'm bor\textbf{ed}" = Yo estoy aburrido
        \item "I'm frustrat\textbf{ed}" = Estoy frustrado
        \item "I'm interest\textbf{ed}" = Estoy interesado
    \end{itemize}
    \item \textbf{-ING} = La COSA/SITUACIÓN causa la emoción
    \begin{itemize}
        \item "It's bor\textbf{ing}" = Es aburrido (causa aburrimiento)
        \item "It's frustrat\textbf{ing}" = Es frustante (causa frustración)
        \item "It's interest\textbf{ing}" = Es interesante (causa interés)
    \end{itemize}
\end{itemize}

\textbf{3. Regla clave:}
\begin{center}
\textbf{-ED} → Cómo te SIENTES tú (persona)\\
\textbf{-ING} → Qué CAUSA el sentimiento (cosa/situación)
\end{center}

\textbf{4. Pares comunes:}
\begin{center}
\begin{tabular}{|l|l|l|}
\hline
\textbf{Español} & \textbf{-ED (persona)} & \textbf{-ING (causa)} \\
\hline
Aburrido & Bored & Boring \\
Confundido & Confused & Confusing \\
Emocionado & Excited & Exciting \\
Frustrado & Frustrated & Frustrating \\
Interesado & Interested & Interesting \\
Cansado & Tired & Tiring \\
\hline
\end{tabular}
\end{center}

\textbf{5. Errores típicos:}
\begin{itemize}
    \item ❌ "I'm boring" (= Soy aburrido/a como persona)
    \item ✓ "I'm bored" (= Estoy aburrido/a - sentimiento)
    \item ❌ "The class is interested" (= ¿La clase siente interés?)
    \item ✓ "The class is interesting" (= La clase es interesante)
\end{itemize}

\textbf{💡 Truco fácil:}
\begin{itemize}
    \item Si hablas de una PERSONA y cómo se SIENTE → -ED
    \item Si hablas de una COSA y qué EFECTO tiene → -ING
\end{itemize}
\end{comparisonbox}

\section{Key Concepts: Common Mishaps}

A \textbf{mishap} is an unlucky accident. Here are common collocations:

\begin{vocabbox}[title=Mishap Collocations]
\begin{itemize}
    \item \textbf{Lose} your keys / phone / wallet \trans{Perder llaves/teléfono/billetera}
    \item \textbf{Miss} a flight / bus / train / deadline \trans{Perder (transporte/plazo)}
    \item \textbf{Spill} coffee / water / wine \trans{Derramar café/agua/vino}
    \item \textbf{Slip} on ice / the floor \trans{Resbalarse en hielo/piso}
    \item \textbf{Drop} your phone / a glass \trans{Dejar caer...}
    \item \textbf{Break down} (car / machine) \trans{Averiarse}
    \item \textbf{Bang} your head / knee \trans{Golpearse la cabeza/rodilla}
\end{itemize}
\end{vocabbox}

\section{Grammar Focus: Adjectives and Narration}

\begin{grammarbox}[title=-ED vs -ING Adjectives]
\begin{itemize}
    \item \textbf{-ED adjectives} describe how \textbf{you feel}.
    \item \textbf{-ING adjectives} describe the \textbf{cause} of the feeling.
\end{itemize}
\textbf{Examples:}
\begin{itemize}
    \item I was \textbf{bored} (feeling). The movie was \textbf{boring} (cause).
    \item I was \textbf{embarrassed} (feeling). The situation was \textbf{embarrassing} (cause).
\end{itemize}
\end{grammarbox}

\begin{grammarbox}[title=Narrating Past Events]
Use these phrases to tell a story:
\begin{itemize}
    \item \textbf{Starting:} I once... / One time... / I remember when...
    \item \textbf{Sequencing:} First... / Then... / Suddenly... / In the end...
    \item \textbf{Background:} I was walking when... (Past Continuous + Past Simple)
\end{itemize}
\end{grammarbox}

\section{Practice Exercises}

\subsection{Exercise 1: Match the Verb and Noun}
Match the verb on the left with the noun on the right.

\begin{enumerate}
    \item Lose \hspace{3cm} a. coffee
    \item Miss \hspace{3cm} b. your head
    \item Spill \hspace{3cm} c. your keys
    \item Bang \hspace{3cm} d. the bus
\end{enumerate}

\subsection{Exercise 2: Choose the Correct Adjective}
Select the correct option (-ed or -ing).

\begin{enumerate}
    \item The news was (shocked / shocking).
    \item I was (annoyed / annoying) because he was late.
    \item It was a very (tired / tiring) journey.
    \item She was (disappointed / disappointing) with the result.
\end{enumerate}

\subsection{Exercise 3: Complete the Story}
Fill in the blanks with: \textit{First, Then, Finally, Unfortunately}.

\underline{\hspace{2cm}}, I woke up late. \underline{\hspace{2cm}}, I couldn't find my keys. \underline{\hspace{2cm}}, I found them under the sofa. \underline{\hspace{2cm}}, I arrived at work on time.

\subsection{Exercise 4: Writing Task}
Write a short paragraph (60-80 words) about a mishap you had. Use at least 3 mishap verbs and 2 feeling adjectives.

\begin{tcolorbox}[colback=white,height=5cm]
% Write your story here
\end{tcolorbox}

\section{Key Takeaways}
\begin{itemize}
    \item Use \textbf{miss} for transport/events and \textbf{lose} for objects.
    \item Remember: -ED for feelings (I am bored), -ING for things (It is boring).
    \item Use sequence words (First, Then, Finally) to structure your stories.
    \item Common mishaps: spill coffee, slip on ice, car broke down.
\end{itemize}

\section{Online Practice}
\begin{tcolorbox}[colback=blue!5,colframe=blue!40!black,title=Resources to Practice]
Online PracticeReinforce what you have learned with these interactive exercises:
\begin{itemize}
    \item \textbf{-ED vs -ING Adjectives:} \url{https://test-english.com/grammar-points/a2/ed-ing-adjectives/}
    \item \textbf{Mishap Vocabulary:} \url{https://www.englishclub.com/vocabulary/disasters.php}
    \item \textbf{Past Experiences:} \url{https://learnenglish.britishcouncil.org/grammar/english-grammar-reference/past-simple}
    \item \textbf{Story Telling Practice:} \url{https://www.perfect-english-grammar.com/narrative-tenses.html}
\end{itemize}
\end{tcolorbox}


