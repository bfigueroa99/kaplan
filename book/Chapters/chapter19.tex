\chapter{Conditionals: If Clauses}

\begin{center}
\begin{tabular}{ccc}
\cefrlevel{B1-B2} & \textbf{Study Time:} 4-5 hours & \textbf{Difficulty:} ⭐⭐⭐⭐⭐
\end{tabular}
\end{center}

\section{Lesson Objectives}
In this chapter, you will learn how to use conditional sentences to express:
\begin{itemize}
    \item \textbf{Facts and general truths} (Zero Conditional) — \textit{Verdades universales}
    \item \textbf{Real future possibilities} (First Conditional) — \textit{Posibilidades reales}
    \item \textbf{Hypothetical or imaginary situations} in the present (Second Conditional) — \textit{Situaciones hipotéticas}
    \item \textbf{Regrets about the past} (Third Conditional) — \textit{Arrepentimientos}
    \item \textbf{Mixed timeframes} (Mixed Conditionals) — \textit{Condicionales mixtos}
\end{itemize}

\textit{By the end of this chapter, you will be able to talk about possibilities, give advice, express regrets, and describe cause-and-effect relationships with confidence.}

\section{What is a Conditional Sentence?}
A conditional sentence talks about a \textbf{condition} (something that might happen) and a \textbf{result} (what happens because of it). In Spanish, these are called \textit{"oraciones condicionales"} and work similarly.

They always have two parts:
\begin{enumerate}
    \item The \textbf{If-clause} (the condition): \textit{If it rains...} \trans{Si llueve...}
    \item The \textbf{Main clause} (the result): \textit{...I will stay home.} \trans{...me quedaré en casa.}
\end{enumerate}

\begin{notebox}[title=Clause Order: Flexible!]
You can swap the order of the clauses without changing the meaning:
\begin{itemize}
    \item \textbf{If it rains, I will stay home.} (If-clause first → use comma)
    \item \textbf{I will stay home if it rains.} (Main clause first → no comma needed)
\end{itemize}
\textbf{Punctuation Rule:} Use a comma only when the sentence \textit{starts} with "If".
\end{notebox}

\subsection{Why Are Conditionals Important?}
Conditionals are essential for:
\begin{itemize}
    \item \textbf{Making plans:} \textit{If I finish early, I'll call you.}
    \item \textbf{Giving advice:} \textit{If I were you, I would accept the offer.}
    \item \textbf{Expressing regrets:} \textit{If I had known, I would have helped.}
    \item \textbf{Explaining rules:} \textit{If you press this button, the machine stops.}
\end{itemize}

\section{Reading Context}
\begin{readingbox}[title=Dialogue: Planning and Regrets]
\textbf{Alice:} What will you do \textbf{if it rains} tomorrow?\\
\textbf{Bob:} \textbf{If it rains}, I\textbf{'ll stay} home and watch TV.\\
\textbf{Alice:} That's boring! \textbf{If I were} you, I\textbf{'d go} to the cinema instead.\\
\textbf{Bob:} Good idea. I wish I had more money though. \textbf{If I had} \$1 million, I\textbf{'d travel} the world!\\
\textbf{Alice:} Me too! By the way, did you pass your exam?\\
\textbf{Bob:} No, I failed. \textbf{If I had studied} harder, I \textbf{would have passed}.\\
\textbf{Alice:} That's a shame. You'll do better next time.
\end{readingbox}

\begin{comparisonbox}[title=🇪🇸 Condicionales: Español vs Inglés - Tiempos Verbales Diferentes]
\textbf{⚠️ TRAMPA MAYOR: Los tiempos verbales NO coinciden entre idiomas}

\textbf{1. First Conditional (posibilidad real):}
\begin{itemize}
    \item Español: "Si llueve, \textit{me quedaré}" (futuro en ambas partes a veces)
    \item English: "If it \textbf{rains} (Present), I\textbf{'ll stay}" (Future)
    \item \textcolor{red}{❌} "If it will rain" (ERROR COMÚN - no usar WILL en IF-clause)
\end{itemize}

\textbf{2. Second Conditional (hipotético presente):}
\begin{itemize}
    \item Español: "Si \textit{tuviera} dinero, \textit{viajaría}" (imperfecto subjuntivo + condicional)
    \item English: "If I \textbf{had} (Past Simple), I\textbf{'d travel}" (would + base)
    \item Nota: En inglés usamos Past Simple, NO subjuntivo
\end{itemize}

\textbf{3. Third Conditional (pasado irreal):}
\begin{itemize}
    \item Español: "Si \textit{hubiera estudiado}, \textit{habría aprobado}" (pluscuamperfecto subjuntivo)
    \item English: "If I \textbf{had studied}, I \textbf{would have passed}" (Past Perfect + would have)
\end{itemize}

\textbf{💡 Error \#1 de hispanohablantes:}
\begin{itemize}
    \item \textcolor{red}{❌} "If I \textit{will have} time, I'll call" 
    \item \textcolor{green}{✓} "If I \textbf{have} time, I'll call"
    \item Regla: NUNCA "will" después de "if" en condicionales
\end{itemize}
\end{comparisonbox}

\section{Zero Conditional: Facts and Truths}
Use this for things that are \textbf{always true}, like scientific facts, habits, or rules. It is not about the future; it is about \textbf{general reality} — things that happen every time the condition is met.

\begin{grammarbox}[title=Zero Conditional Structure]
\textbf{Structure:} If + \textbf{Present Simple}, ... \textbf{Present Simple}

\textbf{Formula:} \textit{If + Subject + Verb (base/s), Subject + Verb (base/s)}
\end{grammarbox}

\subsection{When to Use the Zero Conditional}
\begin{itemize}
    \item \textbf{Scientific facts:} Things that are universally true.
    \item \textbf{General truths:} Common knowledge everyone agrees on.
    \item \textbf{Rules and instructions:} How things work.
    \item \textbf{Personal habits:} What always happens in your routine.
\end{itemize}

\begin{examplebox}[title=Zero Conditional Examples]
\textbf{Scientific Facts:}
\begin{itemize}
    \item \textbf{If} you \textbf{heat} water to 100°C, it \textbf{boils}. \trans{Si calientas agua a 100°C, hierve.}
    \item \textbf{If} you \textbf{freeze} water, it \textbf{becomes} ice. \trans{Si congelas agua, se convierte en hielo.}
    \item \textbf{If} you \textbf{mix} red and yellow, you \textbf{get} orange. \trans{Si mezclas rojo y amarillo, obtienes naranja.}
\end{itemize}

\textbf{General Truths and Habits:}
\begin{itemize}
    \item \textbf{If} I \textbf{don't sleep} well, I \textbf{feel} tired. \trans{Si no duermo bien, me siento cansado.}
    \item People \textbf{get} hungry \textbf{if} they \textbf{don't eat}. \trans{La gente tiene hambre si no come.}
    \item \textbf{If} my phone \textbf{rings} at night, I \textbf{don't answer}. \trans{Si mi teléfono suena de noche, no contesto.}
\end{itemize}

\textbf{Rules and Instructions:}
\begin{itemize}
    \item \textbf{If} you \textbf{press} this button, the machine \textbf{starts}. \trans{Si presionas este botón, la máquina arranca.}
    \item \textbf{If} students \textbf{arrive} late, they \textbf{can't enter} the classroom. \trans{Si los estudiantes llegan tarde, no pueden entrar al aula.}
\end{itemize}
\end{examplebox}

\begin{tipbox}[title=Zero Conditional Tip]
\textbf{"If" = "When":} In Zero Conditionals, you can usually replace "\textbf{if}" with "\textbf{when}" without changing the meaning, because the result \textit{always} happens.
\begin{itemize}
    \item \textit{When you heat water, it boils.} = \textit{If you heat water, it boils.}
    \item \textit{When I eat too much, I feel sick.} = \textit{If I eat too much, I feel sick.}
\end{itemize}
\end{tipbox}

\section{First Conditional: Real Future Possibilities}
Use this for \textbf{real situations} that are likely to happen in the future. The condition is \textbf{possible}, and the result is \textbf{probable}. Think of it as making predictions or plans based on realistic scenarios.

\begin{grammarbox}[title=First Conditional Structure]
\textbf{Structure:} If + \textbf{Present Simple}, ... \textbf{will} + verb

\textbf{Formula:} \textit{If + Subject + Verb (present), Subject + will + Verb (base)}

\textbf{Negative forms:}
\begin{itemize}
    \item If-clause: \textit{If + Subject + \textbf{don't/doesn't} + Verb...}
    \item Main clause: \textit{Subject + \textbf{won't} + Verb...}
\end{itemize}
\end{grammarbox}

\subsection{When to Use the First Conditional}
\begin{itemize}
    \item \textbf{Making predictions:} What will probably happen.
    \item \textbf{Making plans:} What you intend to do.
    \item \textbf{Warnings and threats:} Consequences of actions.
    \item \textbf{Promises and offers:} Commitments you will keep.
\end{itemize}

\begin{examplebox}[title=First Conditional Examples]
\textbf{Predictions and Plans:}
\begin{itemize}
    \item \textbf{If} it \textbf{rains} tomorrow, I \textbf{will stay} home. \trans{Si llueve mañana, me quedaré en casa.}
    \item \textbf{If} she \textbf{studies}, she \textbf{will pass} the exam. \trans{Si ella estudia, aprobará el examen.}
    \item \textbf{If} I \textbf{see} Maria, I \textbf{will tell} her the news. \trans{Si veo a María, le contaré las noticias.}
\end{itemize}

\textbf{Warnings and Threats:}
\begin{itemize}
    \item \textbf{If} you \textbf{touch} that, you \textbf{will get} burned! \trans{¡Si tocas eso, te quemarás!}
    \item \textbf{If} you \textbf{don't hurry}, we \textbf{will be} late. \trans{Si no te apuras, llegaremos tarde.}
    \item I \textbf{won't help} you \textbf{if} you \textbf{don't ask} nicely. \trans{No te ayudaré si no pides amablemente.}
\end{itemize}

\textbf{Promises and Offers:}
\begin{itemize}
    \item \textbf{If} you \textbf{need} help, I \textbf{will be} there for you. \trans{Si necesitas ayuda, estaré ahí para ti.}
    \item I \textbf{will buy} you a gift \textbf{if} you \textbf{pass} your exams. \trans{Te compraré un regalo si apruebas tus exámenes.}
\end{itemize}
\end{examplebox}

\begin{warningbox}[title=Crucial Rule: No "Will" in the If-Clause!]
This is the \textbf{most common mistake} English learners make. Never use \textbf{will} in the IF-clause, even though you're talking about the future.

\begin{itemize}
    \item \incorrect \textit{If I \textbf{will see} him, I will tell him.}
    \item \correct \textit{If I \textbf{see} him, I will tell him.}
    \item \incorrect \textit{If it \textbf{will be} sunny, we will go to the beach.}
    \item \correct \textit{If it \textbf{is} sunny, we will go to the beach.}
\end{itemize}

\textbf{Why?} In English, the Present Simple in the if-clause already implies future meaning. Adding "will" is redundant and grammatically incorrect.
\end{warningbox}

\begin{tipbox}[title=Using Other Modals]
You can replace \textit{will} with other modal verbs in the result clause to express different meanings:
\begin{itemize}
    \item \textbf{can} (ability/permission): \textit{If you finish early, you \textbf{can} leave.}
    \item \textbf{may/might} (less certain): \textit{If it rains, we \textbf{might} cancel the trip.}
    \item \textbf{should} (advice): \textit{If you feel sick, you \textbf{should} see a doctor.}
    \item \textbf{must} (obligation): \textit{If you want to pass, you \textbf{must} study harder.}
\end{itemize}
\end{tipbox}

\section{Second Conditional: Hypothetical Present/Future}
Use this for \textbf{imaginary, impossible, or unlikely situations} in the present or future. The situation is \textbf{not real} — you're dreaming, imagining, or talking about things that probably won't happen.

\begin{grammarbox}[title=Second Conditional Structure]
\textbf{Structure:} If + \textbf{Past Simple}, ... \textbf{would} + verb

\textbf{Formula:} \textit{If + Subject + Verb (past), Subject + would + Verb (base)}

\textbf{Important:} Even though we use the Past Simple, we are NOT talking about the past! The past tense here creates a sense of "unreality" or distance from actual facts.
\end{grammarbox}

\subsection{When to Use the Second Conditional}
\begin{itemize}
    \item \textbf{Imaginary situations:} Things that aren't true now.
    \item \textbf{Unlikely possibilities:} Things that probably won't happen.
    \item \textbf{Dreams and wishes:} What you would do in ideal circumstances.
    \item \textbf{Giving advice:} Using "If I were you..."
\end{itemize}

\begin{examplebox}[title=Second Conditional Examples]
\textbf{Imaginary Situations (Not True Now):}
\begin{itemize}
    \item \textbf{If} I \textbf{had} a million dollars, I \textbf{would buy} a house. \trans{Si tuviera un millón de dólares, compraría una casa.}\\ \textit{(Reality: I don't have a million dollars.)}
    \item \textbf{If} I \textbf{lived} in Japan, I \textbf{would learn} Japanese. \trans{Si viviera en Japón, aprendería japonés.}\\ \textit{(Reality: I don't live in Japan.)}
    \item \textbf{If} I \textbf{knew} the answer, I \textbf{would tell} you. \trans{Si supiera la respuesta, te la diría.}\\ \textit{(Reality: I don't know the answer.)}
\end{itemize}

\textbf{Dreams and Wishes:}
\begin{itemize}
    \item \textbf{If} I \textbf{won} the lottery, I \textbf{would travel} the world. \trans{Si ganara la lotería, viajaría por el mundo.}
    \item \textbf{If} I \textbf{could} fly, I \textbf{would visit} every country. \trans{Si pudiera volar, visitaría todos los países.}
    \item \textbf{If} I \textbf{were} famous, I \textbf{would} help poor people. \trans{Si fuera famoso, ayudaría a la gente pobre.}
\end{itemize}

\textbf{Giving Advice:}
\begin{itemize}
    \item \textbf{If I were} you, I \textbf{would apologize}. \trans{Si yo fuera tú, me disculparía.}
    \item \textbf{If I were} in your situation, I \textbf{would talk} to her. \trans{Si estuviera en tu situación, hablaría con ella.}
    \item What \textbf{would} you do \textbf{if} you \textbf{were} me? \trans{¿Qué harías si fueras yo?}
\end{itemize}
\end{examplebox}

\begin{notebox}[title=The Special "Were" Rule (Subjunctive Mood)]
In formal English and especially when giving advice, use \textbf{were} instead of \textit{was} for ALL subjects, including I, he, she, and it. This is called the \textbf{subjunctive mood}.

\begin{itemize}
    \item \textbf{Formal:} \textit{If I \textbf{were} rich...} / \textit{If she \textbf{were} here...}
    \item \textbf{Informal:} \textit{If I \textbf{was} rich...} / \textit{If she \textbf{was} here...}
\end{itemize}

\textbf{"If I were you"} is a fixed expression — always use \textbf{were}, never \textbf{was}:
\begin{itemize}
    \item \correct \textit{If I \textbf{were} you, I would take the job.}
    \item \incorrect \textit{If I \textbf{was} you, I would take the job.}
\end{itemize}
\end{notebox}

\begin{tipbox}[title=Using Other Modals in Second Conditional]
You can use \textbf{could} or \textbf{might} instead of \textbf{would}:
\begin{itemize}
    \item \textbf{could} (ability): \textit{If I spoke French, I \textbf{could} work in Paris.}
    \item \textbf{might} (possibility): \textit{If we left now, we \textbf{might} arrive on time.}
\end{itemize}
\end{tipbox}

\section{Third Conditional: Hypothetical Past}
Use this for \textbf{imaginary situations in the past}. It talks about things that \textbf{did not happen} and imagines a different result. It is often used to express \textbf{regret}, \textbf{criticism}, or to reflect on past decisions.

\begin{grammarbox}[title=Third Conditional Structure]
\textbf{Structure:} If + \textbf{Past Perfect}, ... \textbf{would have} + \textbf{Past Participle}

\textbf{Formula:} \textit{If + Subject + had + Past Participle, Subject + would have + Past Participle}

\textbf{Key Point:} Both parts of the sentence refer to the past. We are imagining how things \textbf{could have been different}.
\end{grammarbox}

\subsection{When to Use the Third Conditional}
\begin{itemize}
    \item \textbf{Expressing regret:} Wishing you had done something differently.
    \item \textbf{Criticizing past actions:} Pointing out what someone should have done.
    \item \textbf{Imagining different outcomes:} How life would be different if the past had changed.
    \item \textbf{Reflecting on missed opportunities:} What you could have achieved.
\end{itemize}

\begin{examplebox}[title=Third Conditional Examples]
\textbf{Expressing Regret:}
\begin{itemize}
    \item \textbf{If} I \textbf{had studied}, I \textbf{would have passed}. \trans{Si hubiera estudiado, habría aprobado.}\\ \textit{(Reality: I didn't study, so I didn't pass.)}
    \item \textbf{If} I \textbf{had saved} money, I \textbf{would have bought} a car. \trans{Si hubiera ahorrado dinero, habría comprado un carro.}\\ \textit{(Reality: I didn't save, so I couldn't buy it.)}
    \item \textbf{If} I \textbf{had known} about the party, I \textbf{would have gone}. \trans{Si hubiera sabido de la fiesta, habría ido.}
\end{itemize}

\textbf{Criticizing or Reflecting:}
\begin{itemize}
    \item \textbf{If} you \textbf{had listened} to me, you \textbf{wouldn't have made} that mistake. \trans{Si me hubieras escuchado, no habrías cometido ese error.}
    \item \textbf{If} we \textbf{had taken} a taxi, we \textbf{wouldn't have been} late. \trans{Si hubiéramos tomado un taxi, no habríamos llegado tarde.}
    \item \textbf{If} she \textbf{had told} me the truth, I \textbf{would have understood}. \trans{Si me hubiera dicho la verdad, habría entendido.}
\end{itemize}

\textbf{Imagining Different Outcomes:}
\begin{itemize}
    \item \textbf{If} I \textbf{had been born} in another country, my life \textbf{would have been} completely different. \trans{Si hubiera nacido en otro país, mi vida habría sido completamente diferente.}
    \item \textbf{If} the weather \textbf{had been} better, we \textbf{would have had} a great picnic. \trans{Si el clima hubiera sido mejor, habríamos tenido un gran picnic.}
\end{itemize}
\end{examplebox}

\begin{notebox}[title=Contractions in Speech]
In spoken English, contractions are very common with the Third Conditional:

\begin{itemize}
    \item \textbf{'d} = \textbf{had}: \textit{If I'd known...} = \textit{If I had known...}
    \item \textbf{would've} = \textbf{would have}: \textit{I would've come.} = \textit{I would have come.}
    \item \textbf{wouldn't have} = \textit{I wouldn't have said that.}
\end{itemize}

\textbf{Full sentence:} \textit{If I'd known you were coming, I would've baked a cake.}\\
\trans{Si hubiera sabido que venías, habría horneado un pastel.}
\end{notebox}

\begin{tipbox}[title=Using Other Modals in Third Conditional]
You can use \textbf{could have} or \textbf{might have} instead of \textbf{would have}:
\begin{itemize}
    \item \textbf{could have} (ability/possibility): \textit{If I had tried harder, I \textbf{could have} won.}
    \item \textbf{might have} (less certain): \textit{If we had left earlier, we \textbf{might have} caught the train.}
\end{itemize}
\end{tipbox}

\section{Summary Table}

\begin{table}[h]
\centering
\begin{tabular}{|l|l|l|l|l|}
\hline
\textbf{Type} & \textbf{If Clause} & \textbf{Main Clause} & \textbf{Use} & \textbf{Example} \\
\hline
\textbf{Zero} & Present Simple & Present Simple & Facts / Truths & \textit{If you heat ice, it melts.} \\
\hline
\textbf{First} & Present Simple & Will + Verb & Real Possibility & \textit{If it rains, I will stay home.} \\
\hline
\textbf{Second} & Past Simple & Would + Verb & Imaginary Now & \textit{If I were rich, I would travel.} \\
\hline
\textbf{Third} & Past Perfect & Would have + V3 & Imaginary Past & \textit{If I had studied, I would have passed.} \\
\hline
\textbf{Mixed} & Past Perfect & Would + Verb & Past→Present & \textit{If I had studied, I would be a doctor now.} \\
\hline
\end{tabular}
\caption{Conditionals Summary}
\end{table}

\subsection{Visual Timeline of Conditionals}

\begin{center}
\textbf{How "Real" is the Situation?}

\begin{tabular}{ccccc}
\textbf{100\% Real} & & & & \textbf{0\% Real} \\
$\leftarrow$ & & & & $\rightarrow$ \\
Zero & First & & Second & Third \\
(Always true) & (Likely) & & (Unlikely) & (Impossible - past) \\
\end{tabular}
\end{center}

\section{Mixed Conditionals}
Sometimes the time in the "if" clause is different from the time in the "result" clause. This happens when a \textbf{past action affects the present}, or when a \textbf{present situation affected a past result}.

\subsection{Type 1: Past Cause → Present Effect}
This is the most common mixed conditional. A past action (or lack of action) has consequences that we feel \textbf{now}.

\begin{examplebox}[title={Mixed Conditional: Past Cause, Present Effect}]
\textbf{Structure:} If + \textbf{Past Perfect} (3rd), ... \textbf{would} + \textbf{Verb} (2nd)

\begin{itemize}
    \item \textbf{If} I \textbf{had studied} medicine (in the past), I \textbf{would be} a doctor (now).
    \item \trans{Si hubiera estudiado medicina, ahora sería doctor.}
    \item \textit{(Reality: I didn't study medicine, so I'm not a doctor now.)}
\end{itemize}

\begin{itemize}
    \item \textbf{If} you \textbf{hadn't spent} all your money (past), you \textbf{wouldn't be} broke (now).
    \item \trans{Si no hubieras gastado todo tu dinero, no estarías en bancarrota ahora.}
\end{itemize}

\begin{itemize}
    \item \textbf{If} I \textbf{had taken} that job (past), I \textbf{would be living} in New York (now).
    \item \trans{Si hubiera aceptado ese trabajo, estaría viviendo en Nueva York ahora.}
\end{itemize}

\begin{itemize}
    \item \textbf{If} I \textbf{had slept} earlier (past), I \textbf{would not be} so tired (now).
    \item \trans{Si me hubiera acostado más temprano, no estaría tan cansado ahora.}
    \item \textit{(Reality: I didn't sleep earlier, so I am tired now.)}
\end{itemize}
\end{examplebox}

\subsection{Type 2: Present Cause → Past Effect}
This is less common. A permanent present situation affected what happened in the past.

\begin{examplebox}[title={Mixed Conditional: Present Cause, Past Effect}]
\textbf{Structure:} If + \textbf{Past Simple} (2nd), ... \textbf{would have} + \textbf{V3} (3rd)

\begin{itemize}
    \item \textbf{If} I \textbf{spoke} French (general truth about me), I \textbf{would have applied} for that job (in the past).
    \item \trans{Si hablara francés, habría aplicado a ese trabajo.}
    \item \textit{(Reality: I don't speak French, so I didn't apply.)}
\end{itemize}

\begin{itemize}
    \item \textbf{If} I \textbf{weren't} afraid of flying (permanent trait), I \textbf{would have visited} Australia (past).
    \item \trans{Si no tuviera miedo de volar, habría visitado Australia.}
\end{itemize}

\begin{itemize}
    \item \textbf{If} I \textbf{were} more careful (general trait), I \textbf{would not have been} tired yesterday.
    \item \trans{Si fuera más cuidadoso, no habría estado cansado ayer.}
    \item \textit{(Reality: I am not careful, so I was tired yesterday.)}
\end{itemize}
\end{examplebox}

\subsection{Practice: Regular and Mixed Conditionals}

\begin{tcolorbox}[colback=blue!5, colframe=blue!30, title=Conditional Practice Examples]
\textbf{2nd Conditional (Present/Future Hypothetical):}
\begin{itemize}
    \item If I \textbf{went} to sleep earlier, I \textbf{would be} less tired now.
    \item \trans{Si me acostara más temprano, estaría menos cansado ahora.}
\end{itemize}

\textbf{3rd Conditional (Past Hypothetical):}
\begin{itemize}
    \item If I \textbf{had slept} earlier, I \textbf{would not have been} so tired yesterday.
    \item \trans{Si me hubiera acostado más temprano, no habría estado tan cansado ayer.}
\end{itemize}

\textbf{Mixed Conditional (Past → Present):}
\begin{itemize}
    \item If I \textbf{had slept} earlier (past), I \textbf{would not be} so tired (now).
    \item \trans{Si me hubiera acostado más temprano, no estaría tan cansado ahora.}
\end{itemize}

\textbf{Mixed Conditional (Present → Past):}
\begin{itemize}
    \item If I \textbf{were} more careful (general), I \textbf{would not have been} tired yesterday.
    \item \trans{Si fuera más cuidadoso, no habría estado cansado ayer.}
\end{itemize}
\end{tcolorbox}

\section{Alternatives to "If"}
English has several expressions that work like "if" but add different nuances. Learning these will make your English more sophisticated and natural.

\begin{vocabbox}[title=Other Conditional Connectors]

\textbf{1. Unless} = If... not (negative condition)
\begin{itemize}
    \item I won't go \textbf{unless} you go. = I won't go \textbf{if} you \textbf{don't} go.
    \item \trans{No iré a menos que tú vayas.}
    \item We'll be late \textbf{unless} we leave now. = We'll be late \textbf{if} we \textbf{don't} leave now.
    \item \trans{Llegaremos tarde a menos que salgamos ahora.}
\end{itemize}

\textbf{2. As long as / So long as / Provided (that) / Providing (that)} = Only if (strong condition)
\begin{itemize}
    \item You can drive my car \textbf{as long as} you are careful.
    \item \trans{Puedes manejar mi carro siempre y cuando seas cuidadoso.}
    \item I'll help you \textbf{provided that} you help me too.
    \item \trans{Te ayudaré con tal de que tú me ayudes también.}
\end{itemize}

\textbf{3. In case} = Because it might happen (preparation/precaution)
\begin{itemize}
    \item Take an umbrella \textbf{in case} it rains.
    \item \trans{Lleva un paraguas por si llueve.} (Take it now, whether it rains or not).
    \item I'll give you my number \textbf{in case} you need help.
    \item \trans{Te daré mi número por si necesitas ayuda.}
\end{itemize}

\textbf{4. Even if} = The result is the same regardless of the condition
\begin{itemize}
    \item I will go \textbf{even if} it rains. (Rain won't stop me).
    \item \trans{Iré aunque llueva.}
    \item \textbf{Even if} I had money, I wouldn't buy that car.
    \item \trans{Aunque tuviera dinero, no compraría ese carro.}
\end{itemize}

\textbf{5. Whether or not} = In both cases (the condition doesn't matter)
\begin{itemize}
    \item I'm going \textbf{whether or not} you come with me.
    \item \trans{Voy a ir ya sea que vengas o no.}
\end{itemize}

\textbf{6. Suppose / Supposing / What if} = Imagining a hypothetical situation
\begin{itemize}
    \item \textbf{Suppose} you won the lottery. What would you do?
    \item \trans{Supón que ganaras la lotería. ¿Qué harías?}
    \item \textbf{What if} we missed the train?
    \item \trans{¿Qué pasa si perdemos el tren?}
\end{itemize}

\textbf{7. Otherwise} = If not (shows consequence of not doing something)
\begin{itemize}
    \item You should study. \textbf{Otherwise}, you'll fail.
    \item \trans{Deberías estudiar. De lo contrario, reprobarás.}
\end{itemize}
\end{vocabbox}

\section{Common Mistakes to Avoid}

\begin{warningbox}[title=Mistake 1: Using "will" in the If-clause]
\textbf{Why it's wrong:} In First Conditional, the if-clause uses Present Simple to express future meaning. "Will" is redundant.

\begin{itemize}
    \item \incorrect If it \textbf{will rain}, I will stay home.
    \item \correct If it \textbf{rains}, I will stay home.
    \item \incorrect If you \textbf{will come} to the party, I will be happy.
    \item \correct If you \textbf{come} to the party, I will be happy.
\end{itemize}
\end{warningbox}

\begin{warningbox}[title=Mistake 2: Using "would" in the If-clause]
\textbf{Why it's wrong:} In Second Conditional, use Past Simple in the if-clause, not "would."

\begin{itemize}
    \item \incorrect If I \textbf{would have} money, I would travel.
    \item \correct If I \textbf{had} money, I would travel.
    \item \incorrect If I \textbf{would be} you, I would accept.
    \item \correct If I \textbf{were} you, I would accept.
\end{itemize}
\end{warningbox}

\begin{warningbox}[title=Mistake 3: Confusing "In case" with "If"]
\textbf{The difference:}
\begin{itemize}
    \item \textbf{If} = The action happens \textit{after} the condition is true.
    \item \textbf{In case} = The action happens \textit{before}, as a precaution.
\end{itemize}

\begin{itemize}
    \item \textit{I'll call you \textbf{if} I need help.} (I will wait until I need help to call.)
    \item \textit{Take my number \textbf{in case} you need help.} (Take it now, before anything happens.)
\end{itemize}
\end{warningbox}

\begin{warningbox}[title=Mistake 4: Confusing "Unless" with "If"]
\textbf{Remember:} Unless = If... not. Don't use a negative verb after "unless."

\begin{itemize}
    \item \incorrect I won't go \textbf{unless} you \textbf{don't} come. (Double negative!)
    \item \correct I won't go \textbf{unless} you come. (= if you don't come)
    \item \correct I won't go \textbf{if} you \textbf{don't} come.
\end{itemize}
\end{warningbox}

\begin{warningbox}[title=Mistake 5: Using "would have" instead of "had" in Third Conditional If-clause]
\textbf{Why it's wrong:} The if-clause in Third Conditional uses Past Perfect (had + V3), not "would have."

\begin{itemize}
    \item \incorrect If I \textbf{would have known}, I would have helped.
    \item \correct If I \textbf{had known}, I would have helped.
\end{itemize}
\end{warningbox}

\begin{warningbox}[title=Mistake 6: Mixing Conditional Types Incorrectly]
\textbf{Be consistent with time frames:}

\begin{itemize}
    \item \incorrect If I \textbf{study} (1st), I \textbf{would pass} (2nd). — Mixing 1st and 2nd!
    \item \correct If I \textbf{study}, I \textbf{will pass}. (1st Conditional)
    \item \correct If I \textbf{studied}, I \textbf{would pass}. (2nd Conditional)
\end{itemize}
\end{warningbox}

\section{Practice Exercises}

\subsection{Exercise 1: Match the Conditional Type}
Identify if the sentence is Zero, First, Second, or Third conditional.

\begin{enumerate}
    \item If I were a bird, I would fly. (\_\_\_\_\_\_\_\_\_\_)
    \item If you freeze water, it becomes ice. (\_\_\_\_\_\_\_\_\_\_)
    \item If she calls, I will answer. (\_\_\_\_\_\_\_\_\_\_)
    \item If I had known, I would have helped. (\_\_\_\_\_\_\_\_\_\_)
    \item If you press this button, the door opens. (\_\_\_\_\_\_\_\_\_\_)
    \item If I spoke Chinese, I would work in Beijing. (\_\_\_\_\_\_\_\_\_\_)
    \item If we had left earlier, we wouldn't have missed the bus. (\_\_\_\_\_\_\_\_\_\_)
    \item If you study hard, you will pass the exam. (\_\_\_\_\_\_\_\_\_\_)
\end{enumerate}

\subsection{Exercise 2: Fill in the Blanks}
Complete the sentences with the correct verb form.

\begin{enumerate}
    \item (First) If it \underline{\hspace{3cm}} (rain), we will cancel the picnic.
    \item (Second) If I \underline{\hspace{3cm}} (be) you, I would take the job.
    \item (Third) If they \underline{\hspace{3cm}} (leave) earlier, they would have arrived on time.
    \item (Zero) If you mix red and blue, you \underline{\hspace{3cm}} (get) purple.
    \item (Second) If she \underline{\hspace{3cm}} (have) more time, she would learn a new language.
    \item (First) I \underline{\hspace{3cm}} (call) you if I need help.
    \item (Third) If I \underline{\hspace{3cm}} (know) about the meeting, I would have attended.
    \item (Zero) Water \underline{\hspace{3cm}} (freeze) if the temperature drops below 0°C.
\end{enumerate}

\subsection{Exercise 3: Multiple Choice}
Choose the correct option.

\begin{enumerate}
    \item If I \_\_\_\_\_ the lottery, I would buy a mansion.
    \begin{itemize}
        \item a) win \quad b) won \quad c) had won
    \end{itemize}
    \item I will call you if I \_\_\_\_\_ time.
    \begin{itemize}
        \item a) have \quad b) will have \quad c) had
    \end{itemize}
    \item If she had studied, she \_\_\_\_\_ the test.
    \begin{itemize}
        \item a) passes \quad b) would pass \quad c) would have passed
    \end{itemize}
    \item If you \_\_\_\_\_ water to 100°C, it boils.
    \begin{itemize}
        \item a) heat \quad b) heated \quad c) will heat
    \end{itemize}
    \item I \_\_\_\_\_ you if I were you.
    \begin{itemize}
        \item a) wouldn't trust \quad b) won't trust \quad c) hadn't trusted
    \end{itemize}
    \item If I had taken that job, I \_\_\_\_\_ in London now.
    \begin{itemize}
        \item a) will be living \quad b) would be living \quad c) would have lived
    \end{itemize}
\end{enumerate}

\subsection{Exercise 4: Error Correction}
Find and correct the mistake in each sentence.

\begin{enumerate}
    \item If I will see her, I will give her your message. \\
    \textbf{Correction:} \underline{\hspace{10cm}}
    \item If I would have more money, I would travel more. \\
    \textbf{Correction:} \underline{\hspace{10cm}}
    \item If I would have known, I would have come earlier. \\
    \textbf{Correction:} \underline{\hspace{10cm}}
    \item I won't go unless you don't come with me. \\
    \textbf{Correction:} \underline{\hspace{10cm}}
\end{enumerate}

\subsection{Exercise 5: Rewrite the Sentences}
Rewrite the reality as a conditional sentence.

\begin{enumerate}
    \item \textbf{Reality:} I don't have a car, so I take the bus. (Use 2nd Conditional)
    \item \textbf{Conditional:} If I \underline{\hspace{8cm}}, I wouldn't take the bus.
    \vspace{0.3cm}
    \item \textbf{Reality:} I didn't wake up early, so I missed the train. (Use 3rd Conditional)
    \item \textbf{Conditional:} If I \underline{\hspace{8cm}}, I wouldn't have missed the train.
    \vspace{0.3cm}
    \item \textbf{Reality:} She doesn't speak English, so she can't apply for the job. (Use 2nd Conditional)
    \item \textbf{Conditional:} If she \underline{\hspace{8cm}}, she could apply for the job.
    \vspace{0.3cm}
    \item \textbf{Reality:} He didn't study medicine, so he isn't a doctor now. (Use Mixed Conditional)
    \item \textbf{Conditional:} If he \underline{\hspace{8cm}}, he would be a doctor now.
\end{enumerate}

\subsection{Exercise 6: Complete the Conditional Chain}
Complete each conditional type using the situation given.

\textbf{Situation: Winning the lottery}

\begin{enumerate}
    \item \textbf{Zero:} If someone wins the lottery, they \underline{\hspace{5cm}}.
    \item \textbf{First:} If I win the lottery tomorrow, I will \underline{\hspace{5cm}}.
    \item \textbf{Second:} If I won the lottery, I would \underline{\hspace{5cm}}.
    \item \textbf{Third:} If I had won the lottery last year, I would have \underline{\hspace{4cm}}.
\end{enumerate}

\subsection{Exercise 7: Personal Practice}
Complete these sentences about your own life.

\begin{enumerate}
    \item If I have free time this weekend, I will...
    \item If I could travel anywhere in the world, I would go to...
    \item If I had been born in a different country, I would have...
    \item If I were the president of my country, I would...
    \item If I hadn't studied English, I wouldn't be able to...
\end{enumerate}

\section{Answer Key}

\subsection{Exercise 1 Answers}
\begin{enumerate}
    \item Second Conditional
    \item Zero Conditional
    \item First Conditional
    \item Third Conditional
    \item Zero Conditional
    \item Second Conditional
    \item Third Conditional
    \item First Conditional
\end{enumerate}

\subsection{Exercise 2 Answers}
\begin{enumerate}
    \item rains
    \item were
    \item had left
    \item get
    \item had
    \item will call
    \item had known
    \item freezes
\end{enumerate}

\subsection{Exercise 3 Answers}
\begin{enumerate}
    \item b) won
    \item a) have
    \item c) would have passed
    \item a) heat
    \item a) wouldn't trust
    \item b) would be living
\end{enumerate}

\subsection{Exercise 4 Answers}
\begin{enumerate}
    \item If I \textbf{see} her, I will give her your message.
    \item If I \textbf{had} more money, I would travel more.
    \item If I \textbf{had known}, I would have come earlier.
    \item I won't go unless you come with me. (Remove "don't")
\end{enumerate}

\subsection{Exercise 5 Answers}
\begin{enumerate}
    \item If I \textbf{had a car}, I wouldn't take the bus.
    \item If I \textbf{had woken up early}, I wouldn't have missed the train.
    \item If she \textbf{spoke English}, she could apply for the job.
    \item If he \textbf{had studied medicine}, he would be a doctor now.
\end{enumerate}



\section{Online Practice}
Here are some useful websites to practice this topic:
\begin{itemize}
    \item \href{https://learnenglish.britishcouncil.org/grammar/b1-b2-grammar/conditionals-zero-first-second}{British Council LearnEnglish}
    \item \href{https://learnenglish.britishcouncil.org/grammar/b1-b2-grammar/third-conditional}{British Council LearnEnglish}
    \item \href{https://test-english.com/grammar-points/b1/first-second-third-conditionals/}{Test-English}
\end{itemize}
