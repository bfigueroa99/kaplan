\chapter{Phrasal Verbs: Form, Function, and Usage}

\begin{center}
\begin{tabular}{ccc}
\cefrlevel{B2-C1} & \textbf{Study Time:} 3-4 hours & \textbf{Difficulty:} ⭐⭐⭐⭐☆
\end{tabular}
\end{center}

\section{Lesson Objectives}
In this chapter, you will learn:
\begin{itemize}
    \item What phrasal verbs are and how they're formed
    \item Categories of phrasal verbs (transitive, intransitive)
    \item Separable vs. non-separable phrasal verbs
    \item Common phrasal verbs by category
    \item How to use phrasal verbs in context
    \item British English phrasal verb preferences
\end{itemize}

\section{Reading Context}

\begin{readingbox}[title=Dialogue: A Busy Day at Work]
\textbf{Manager:} Good morning! Can you \textbf{take on} the new project?\\
\textbf{Employee:} Sure, but first I need to \textbf{catch up on} my emails.\\
\textbf{Manager:} Okay. The client will \textbf{turn up} at 2 PM. Make sure to \textbf{set up} the meeting room.\\
\textbf{Employee:} No problem. Should I \textbf{put together} a presentation?\\
\textbf{Manager:} Yes, please. And \textbf{point out} any issues you see. Don't \textbf{leave out} important details.\\
\textbf{Employee:} Got it. I'll \textbf{run through} everything before the meeting.\\
\textbf{Manager:} Great. If you \textbf{run into} any problems, just \textbf{get in touch}.
\end{readingbox}

\begin{comparisonbox}[title=🇪🇸 Phrasal Verbs: NO existen en español - Pesadilla para hispanohablantes]
\textbf{⚠️ El español NO tiene phrasal verbs - usamos verbos simples o expresiones}

\textbf{1. Español - verbo simple:}
\begin{itemize}
    \item "Buscar" = Look for
    \item "Cuidar" = Look after
    \item "Rechazar" = Turn down
    \item "Inventar" = Make up
\end{itemize}

\textbf{2. Inglés - verbo + partícula = significado nuevo:}
\begin{itemize}
    \item LOOK = mirar
    \item LOOK UP = buscar información (diccionario)
    \item LOOK FOR = buscar (algo perdido)
    \item LOOK AFTER = cuidar
    \item LOOK INTO = investigar
\end{itemize}

\textbf{3. Problema \#1 - NO son literales:}
\begin{itemize}
    \item "Put off" ≠ "poner fuera" → significa "postponer/aplazar"
    \item "Give up" ≠ "dar arriba" → significa "rendirse/abandonar"
    \item "Take after" ≠ "tomar después" → significa "parecerse a"
\end{itemize}

\textbf{4. Problema \#2 - Muchos significados:}
\begin{itemize}
    \item GET UP = levantarse
    \item GET ON = llevarse bien con alguien
    \item GET OVER = recuperarse de algo
    \item GET AWAY = escaparse
    \item Español usa verbos completamente diferentes
\end{itemize}

\textbf{5. Británico vs Americano:}
\begin{itemize}
    \item UK: "ring up" (llamar por teléfono) - muy británico
    \item US: "call" (más directo)
    \item UK: "pop round" (visitar informalmente) - muy británico
    \item US: "stop by"
\end{itemize}

\textbf{💡 Estrategia para hispanohablantes:}
\begin{itemize}
    \item NO intentes traducir literalmente
    \item Memoriza como expresiones completas
    \item Aprende por contexto, no por partes
    \item Los phrasal verbs son ESENCIALES en inglés informal/hablado
\end{itemize}
\end{comparisonbox}

\section{Grammar Focus: What are Phrasal Verbs?}

\begin{grammarbox}[title=Definition and Structure]
A \textbf{phrasal verb} is a verb combined with a particle (preposition or adverb) that creates a new meaning.

\textbf{Structure:}
\begin{center}
Verb + Particle (Preposition or Adverb)
\end{center}

\textbf{Examples:}
\begin{itemize}
    \item \textbf{look} (verb) + \textbf{up} (particle) = \textbf{look up} (search for information)
    \item \textbf{put} (verb) + \textbf{on} (particle) = \textbf{put on} (wear, apply)
    \item \textbf{turn} (verb) + \textbf{down} (particle) = \textbf{turn down} (reject, reduce volume)
\end{itemize}

\textbf{Important:} The meaning of a phrasal verb is often very different from the base verb.
\begin{itemize}
    \item "Look" = observe with eyes
    \item "Look up" = search for information
    \item "Look after" = care for
\end{itemize}
\end{grammarbox}

\section{Two Types: Separable and Non-Separable}

\subsection{Separable Phrasal Verbs}

\begin{grammarbox}[title=Separable (Transitive) Phrasal Verbs]
A phrasal verb is \textbf{separable} if the particle can be placed between the verb and the object, OR after the object.

\textbf{Pattern:}
\begin{itemize}
    \item V + Obj + Particle: \textbf{put} the book \textbf{down}
    \item V + Particle + Obj: \textbf{put down} the book
\end{itemize}

\textbf{Examples:}
\begin{itemize}
    \item \textbf{Take out} the trash. = \textbf{Take} the trash \textbf{out}. ✅
    \item \textbf{Put on} your jacket. = \textbf{Put} your jacket \textbf{on}. ✅
    \item \textbf{Turn down} the music. = \textbf{Turn} the music \textbf{down}. ✅
\end{itemize}

\textbf{Note:} With pronouns, the particle MUST come after
\begin{itemize}
    \item "Take it out" ✅
    \item \textbf{[INCORRECT]} "Take out it"
\end{itemize}
\end{grammarbox}

\subsection{Non-Separable Phrasal Verbs}

\begin{grammarbox}[title=Non-Separable (Intransitive) Phrasal Verbs]
A phrasal verb is \textbf{non-separable} if the particle CANNOT move to a different position.

\textbf{Pattern:}
\begin{itemize}
    \item V + Particle (+ Object if needed)
\end{itemize}

\textbf{Examples:}
\begin{itemize}
    \item \textbf{get up} (wake and rise from bed) — cannot separate
    \begin{itemize}
        \item "I get up at 7 AM" ✅
        \item \textbf{[INCORRECT]} "I get at 7 AM up" (incorrect)
    \end{itemize}
    \item \textbf{run into} (meet by chance) — cannot separate
    \begin{itemize}
        \item "I ran into an old friend" ✅
        \item \textbf{[INCORRECT]} "I ran an old friend into" (incorrect)
    \end{itemize}
    \item \textbf{look after} (care for) — cannot separate
    \begin{itemize}
        \item "She looks after the children" ✅
        \item \textbf{[INCORRECT]} "She looks the children after" (incorrect)
    \end{itemize}
\end{itemize}
\end{grammarbox}

\section{Common Phrasal Verbs by Category}

\subsection{MOVEMENT and DIRECTION}

\begin{table}[h]
\centering
\begin{tabular}{|l|l|l|}
\hline
\textbf{Phrasal Verb} & \textbf{Meaning} & \textbf{Example} \\
\hline
Get up & Wake and rise & I \textbf{get up} at 6 AM. \\
\hline
Go away & Leave, depart & They \textbf{went away} for the weekend. \\
\hline
Come back & Return & When will you \textbf{come back}? \\
\hline
Turn around & Reverse direction & Please \textbf{turn around}. \\
\hline
Run away & Flee, escape & The dog \textbf{ran away}. \\
\hline
\end{tabular}
\caption{Phrasal verbs of movement}
\end{table}

\subsection{WORK and RESPONSIBILITY}

\begin{table}[h]
\centering
\begin{tabular}{|l|l|l|}
\hline
\textbf{Phrasal Verb} & \textbf{Meaning} & \textbf{Example} \\
\hline
Take on & Accept responsibility & She \textbf{took on} a new project. \\
\hline
Put together & Assemble, create & \textbf{Put together} a plan. \\
\hline
Carry out & Execute, perform & \textbf{Carry out} the instructions. \\
\hline
Set up & Establish, arrange & \textbf{Set up} a meeting. \\
\hline
Catch up on & Get current with & I need to \textbf{catch up on} emails. \\
\hline
\end{tabular}
\caption{Phrasal verbs for work}
\end{table}

\subsection{COMMUNICATION and RELATIONSHIPS}

\begin{table}[h]
\centering
\begin{tabular}{|l|l|l|}
\hline
\textbf{Phrasal Verb} & \textbf{Meaning} & \textbf{Example} \\
\hline
Get in touch & Contact, reach out & \textbf{Get in touch} if you need help. \\
\hline
Point out & Indicate, mention & She \textbf{pointed out} a mistake. \\
\hline
Run through & Go over, review & Let's \textbf{run through} the plan. \\
\hline
Talk about & Discuss & We \textbf{talked about} the project. \\
\hline
Look after & Care for & Can you \textbf{look after} my cat? \\
\hline
\end{tabular}
\caption{Phrasal verbs for communication}
\end{table}

\subsection{CHANGES and PROGRESS}

\begin{table}[h]
\centering
\begin{tabular}{|l|l|l|}
\hline
\textbf{Phrasal Verb} & \textbf{Meaning} & \textbf{Example} \\
\hline
Give up & Stop trying, quit & Don't \textbf{give up}. \\
\hline
Slow down & Reduce speed & \textbf{Slow down}; you're going too fast. \\
\hline
Speed up & Increase pace & \textbf{Speed up}; we're late. \\
\hline
Keep on & Continue & Please \textbf{keep on} trying. \\
\hline
Turn into & Transform, become & The caterpillar \textbf{turned into} a butterfly. \\
\hline
\end{tabular}
\caption{Phrasal verbs for change}
\end{table}

\subsection{COMMON EVERYDAY PHRASAL VERBS}

\begin{table}[h]
\centering
\begin{tabular}{|l|l|l|}
\hline
\textbf{Phrasal Verb} & \textbf{Meaning} & \textbf{Example} \\
\hline
Put on & Wear; apply (makeup, etc.) & \textbf{Put on} your shoes. \\
\hline
Take off & Remove; leave (plane) & \textbf{Take off} your jacket. \\
\hline
Turn on & Switch on & \textbf{Turn on} the light. \\
\hline
Turn off & Switch off & \textbf{Turn off} the TV. \\
\hline
Pick up & Lift; collect & \textbf{Pick up} your phone. \\
\hline
Put down & Lower; stop; lay down & \textbf{Put down} your bag. \\
\hline
Look up & Search for information & \textbf{Look up} the word in the dictionary. \\
\hline
Run into & Meet by chance & I \textbf{ran into} an old friend. \\
\hline
\end{tabular}
\caption{Everyday phrasal verbs}
\end{table}

\section{Practice Exercises}

\subsection{Exercise 1: Separable or Non-Separable?}
Indicate if each phrasal verb is separable (S) or non-separable (NS):

\begin{enumerate}
    \item Look up \underline{\hspace{2cm}}
    \item Turn down \underline{\hspace{2cm}}
    \item Run into \underline{\hspace{2cm}}
    \item Put on \underline{\hspace{2cm}}
    \item Look after \underline{\hspace{2cm}}
\end{enumerate}

\subsection{Exercise 2: Complete the Sentences}
Fill in the blanks with appropriate phrasal verbs:

\begin{enumerate}
    \item She \underline{\hspace{3cm}} her new job with enthusiasm. (start)
    \item Can you \underline{\hspace{3cm}} that problem when you have time? (look/investigate)
    \item The plane will \underline{\hspace{3cm}} at 3 PM. (depart)
    \item I need to \underline{\hspace{3cm}} my emails before the meeting. (become current with)
\end{enumerate}

\subsection{Exercise 3: Rearrange with Pronouns}
Rewrite using pronouns where possible:

\begin{enumerate}
    \item Put on your jacket. \underline{\hspace{4cm}}
    \item Turn down the volume. \underline{\hspace{4cm}}
    \item Take out the trash. \underline{\hspace{4cm}}
    \item Pick up your phone. \underline{\hspace{4cm}}
\end{enumerate}

\subsection{Exercise 4: Matching}
Match phrasal verbs with their meanings:

\begin{enumerate}
    \item Take on \hspace{3cm} a. Investigate, look for
    \item Look up \hspace{3cm} b. Meet unexpectedly
    \item Run into \hspace{3cm} c. Accept (responsibility)
    \item Give up \hspace{3cm} d. Stop trying; quit
\end{enumerate}

\subsection{Exercise 5: Write Your Own Sentences}
Write 5 sentences using different phrasal verbs from this chapter. Include at least 2 separable and 2 non-separable verbs.

\begin{tcolorbox}[colback=white,height=6cm]
\end{tcolorbox}

\section{\textbf{[INCORRECT]} Common Mistakes}

\begin{tcolorbox}[colback=red!5, colframe=red!30, title=Phrasal Verb Errors]

\textbf{Mistake 1:} Separating a non-separable phrasal verb
\begin{itemize}
    \item \textbf{[INCORRECT]} "I looked my friend after." (wrong separation)
    \item ✅ "I \textbf{looked after} my friend." (correct)
\end{itemize}

\textbf{Mistake 2:} Not separating when using pronouns
\begin{itemize}
    \item \textbf{[INCORRECT]} "Put on it" (wrong — must separate)
    \item ✅ "\textbf{Put it on}" (correct)
\end{itemize}

\textbf{Mistake 3:} Using the wrong particle
\begin{itemize}
    \item \textbf{[INCORRECT]} "I looked for up the word"
    \item ✅ "I \textbf{looked up} the word" (or "looked for it")
\end{itemize}

\textbf{Mistake 4:} Confusing similar phrasal verbs
\begin{itemize}
    \item "Look after" = care for (my sister)
    \item "Look for" = search for (my keys)
    \item "Look up" = search for information (a word)
\end{itemize}

\end{tcolorbox}

\section{🇬🇧 British English Notes}

Phrasal verbs are especially common in British English:

\begin{itemize}
    \item British English uses phrasal verbs much more frequently than formal American English
    \item British English preference: "ring up" vs. American "call"
    \item British: "put up with" (tolerate), American: "put up with" (same)
    \item British: "sort out" (resolve), American: "figure out" or "work out"
    \item British: "tidy up" (organize), American: "clean up"
    \item British: "get on with" (make progress), American: "get along with"
\end{itemize}

\section{Key Takeaways}

\begin{itemize}
    \item \textbf{Phrasal verbs} = verb + particle with a new meaning
    \item \textbf{Separable} = particle can move or stay with verb
    \item \textbf{Non-separable} = particle stays with verb
    \item With pronouns, separable verbs MUST separate: "put it on" (not "put on it")
    \item Phrasal verbs are essential for natural, fluent English
    \item British English uses phrasal verbs more frequently than American English
    \item Phrasal verbs are often informal — use single-word verbs for more formal contexts
\end{itemize}

\section{Phrasal Verbs in Social Context}

\subsection{Friendship and Relationship Phrasal Verbs}

\begin{vocabbox}[title=Common Phrasal Verbs for Relationships]
\begin{itemize}
    \item \keyterm{Fall out with} - To have an argument and stop being friends (temporarily or permanently)
    \item \keyterm{Make up} - To become friends again after an argument
    \item \keyterm{Get along/on with} - To have a good relationship with someone
    \item \keyterm{Look up to} - To admire and respect someone
    \item \keyterm{Take after} - To resemble a family member in appearance or character
    \item \keyterm{Bring up} - To raise/educate a child
    \item \keyterm{Grow up} - To become an adult
    \item \keyterm{Split up / Break up} - To end a romantic relationship
    \item \keyterm{Ask out} - To invite someone on a date
    \item \keyterm{Go out (with)} - To date someone
\end{itemize}
\end{vocabbox}

\subsection{Practice: Friendship Expressions}

\begin{tcolorbox}[colback=blue!5, colframe=blue!30, title=Idiomatic Expressions with Phrasal Verbs]
\textbf{1. Falling out and making up:}
\begin{itemize}
    \item "I'm always \textbf{falling out with} my best friend, but usually we \textbf{make up}."
    \item Translation: "I'm always \textit{fighting with} my best friend, but usually we \textit{become friends again}."
\end{itemize}

\textbf{2. Being funny (British expression):}
\begin{itemize}
    \item "My best friend is such a \textbf{good laugh}."
    \item Translation: "My best friend is \textit{so funny}."
    \item Note: "Good laugh" is very British and informal
\end{itemize}

\textbf{3. Being very close:}
\begin{itemize}
    \item "My best friend and I are like \textbf{brother/sisters from another mother}."
    \item Translation: "My best friend and I are \textit{very close like siblings}."
    \item This is an informal, affectionate expression
\end{itemize}
\end{tcolorbox}

\section{Rephrasing and Paraphrasing}

Understanding phrasal verbs is key to \textbf{rephrasing} - expressing the same idea using different words.

\begin{grammarbox}[title=When to Use Rephrasing]
\textbf{Rephrasing} is the act of expressing the same idea using different words or phrases.

\textbf{Use rephrasing to:}
\begin{itemize}
    \item Explain with other words so someone understands better
    \item Avoid repeating the same words
    \item Make your writing more interesting
    \item Simplify complex ideas
    \item Clarify meaning
\end{itemize}

\textbf{Examples:}
\begin{itemize}
    \item Phrasal verb: "Look up" → Rephrase: "Search for information"
    \item Phrasal verb: "Give up" → Rephrase: "Quit" or "Stop trying"
    \item Phrasal verb: "Put off" → Rephrase: "Postpone" or "Delay"
\end{itemize}
\end{grammarbox}

\subsection{Paraphrasing vs. Rephrasing}

\begin{table}[h]
\centering
\begin{tabular}{|l|p{5cm}|p{5cm}|}
\hline
\textbf{Aspect} & \textbf{Rephrasing} & \textbf{Paraphrasing} \\
\hline
Definition & Expressing with different words & Restating entire sentence/passage \\
\hline
Length & Usually shorter & Can be longer or same length \\
\hline
Purpose & Clarify, simplify, avoid repetition & Summarize, avoid plagiarism, understand \\
\hline
Use & Daily conversation, writing & Academic writing, note-taking, literature review \\
\hline
\end{tabular}
\end{table}

\begin{grammarbox}[title=When to Use Paraphrasing]
\textbf{Paraphrasing} is the act of rephrasing or restating a sentence or passage using different words while maintaining the original meaning.

\textbf{Use paraphrasing to:}
\begin{itemize}
    \item Summarize information from sources
    \item Avoid plagiarism in academic writing
    \item Clarify complex ideas for better understanding
    \item Demonstrate understanding of material
    \item Write literature reviews
\end{itemize}

\textbf{Example:}
\begin{itemize}
    \item \textbf{Original:} "The cat sat on the mat while the dog barked loudly."
    \item \textbf{Paraphrase:} "While the dog was barking noisily, the cat remained seated on the mat."
\end{itemize}
\end{grammarbox}

\subsection{Practice Exercise: Rephrasing Phrasal Verbs}

Rephrase these sentences by replacing the phrasal verb with a single-word verb:

\begin{enumerate}
    \item "I need to \textbf{look up} this word in the dictionary."
    \begin{itemize}
        \item Answer: "I need to \textit{search for/find} this word in the dictionary."
    \end{itemize}
    
    \item "She \textbf{gave up} smoking last year."
    \begin{itemize}
        \item Answer: "She \textit{quit/stopped} smoking last year."
    \end{itemize}
    
    \item "They \textbf{put off} the meeting until next week."
    \begin{itemize}
        \item Answer: "They \textit{postponed/delayed} the meeting until next week."
    \end{itemize}
    
    \item "I \textbf{get along with} my colleagues very well."
    \begin{itemize}
        \item Answer: "I \textit{relate to/cooperate with} my colleagues very well."
    \end{itemize}
\end{enumerate}
