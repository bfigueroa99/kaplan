\chapter{Comparatives and Superlatives}

\begin{center}
\begin{tabular}{ccc}
\cefrlevel{A2-B1} & \textbf{Study Time:} 2-3 hours & \textbf{Difficulty:} ⭐⭐⭐☆☆
\end{tabular}
\end{center}

\section{Lesson Objectives}
In this chapter, you will learn:
\begin{itemize}
    \item How to form comparative and superlative adjectives
    \item Rules for short, long, and irregular adjectives
    \item How to compare people, places, and things
\end{itemize}

\section{Reading Context}
\begin{readingbox}[title=Dialogue: Choosing a Vacation Destination]
\textbf{Sarah:} Where should we go for our vacation? I think Paris is \textbf{more romantic than} London.\\
\textbf{Mike:} Maybe, but London is \textbf{cheaper than} Paris right now. And the museums are \textbf{better}.\\
\textbf{Sarah:} True, but the food in Paris is \textbf{the best} in the world!\\
\textbf{Mike:} What about Rome? It's \textbf{hotter than} both London and Paris.\\
\textbf{Sarah:} Rome is beautiful, but it's also \textbf{the most crowded} city in summer.\\
\textbf{Mike:} Okay, let's go to the beach. It's \textbf{the easiest} option.
\end{readingbox}

\section{Grammar Focus: Comparatives}

Comparatives are used to compare \textbf{two} things, people, or places.

\begin{grammarbox}[title=Comparative Structure]
\begin{center}
\Large Subject + verb + \textbf{comparative adjective} + \textbf{than} + object
\end{center}

\textbf{Examples:}
\begin{itemize}
    \item London is \textbf{bigger than} Dublin. \trans{Londres es más grande que Dublín}
    \item This book is \textbf{more interesting than} that one. \trans{Este libro es más interesante que ese}
\end{itemize}
\end{grammarbox}

\section{Grammar Focus: Superlatives}

Superlatives are used to compare \textbf{three or more} things and show the extreme.

\begin{grammarbox}[title=Superlative Structure]
\begin{center}
\Large Subject + verb + \textbf{the} + \textbf{superlative adjective} + (in/of)
\end{center}

\textbf{Examples:}
\begin{itemize}
    \item Tokyo is \textbf{the biggest} city in Japan. \trans{Tokio es la ciudad más grande de Japón}
    \item This is \textbf{the most delicious} pizza. \trans{Esta es la pizza más deliciosa}
\end{itemize}
\end{grammarbox}

\section{Formation Rules Reference}

\begin{table}[h]
\centering
\begin{tabular}{|l|l|l|l|}
\hline
\textbf{Type} & \textbf{Adjective} & \textbf{Comparative} & \textbf{Superlative} \\
\hline
Short (1 syllable) & tall & taller & the tallest \\
\hline
Short ending in -e & nice & nicer & the nicest \\
\hline
Short ending in CVC & big & bigger & the biggest \\
\hline
Ending in -y & happy & happier & the happiest \\
\hline
Long (2+ syllables) & expensive & more expensive & the most expensive \\
\hline
\textbf{Irregular} & \textbf{good} & \textbf{better} & \textbf{the best} \\
\hline
\textbf{Irregular} & \textbf{bad} & \textbf{worse} & \textbf{the worst} \\
\hline
\end{tabular}
\caption{Adjective formation rules}
\end{table}

\section{Practice Exercises}

\subsection{Exercise 1: Write the Comparative and Superlative}

\begin{tabular}{|l|l|l|}
\hline
\textbf{Adjective} & \textbf{Comparative} & \textbf{Superlative} \\
\hline
fast & \underline{\hspace{3cm}} & \underline{\hspace{3cm}} \\
\hline
beautiful & \underline{\hspace{3cm}} & \underline{\hspace{3cm}} \\
\hline
hot & \underline{\hspace{3cm}} & \underline{\hspace{3cm}} \\
\hline
easy & \underline{\hspace{3cm}} & \underline{\hspace{3cm}} \\
\hline
bad & \underline{\hspace{3cm}} & \underline{\hspace{3cm}} \\
\hline
\end{tabular}

\subsection{Exercise 2: Complete the Sentences}

\begin{enumerate}
    \item Mount Everest is \underline{\hspace{2cm}} (high) mountain in the world.
    \item My brother is \underline{\hspace{2cm}} (young) than me.
    \item This restaurant is \underline{\hspace{2cm}} (good) than the one we went to yesterday.
    \item English is \underline{\hspace{2cm}} (easy) than Chinese for Spanish speakers.
\end{enumerate}

\subsection{Exercise 3: Compare Your City}
Write 5 sentences comparing your city to London or another city you know.

\begin{tcolorbox}[colback=white,height=5cm]
% Example: My city is smaller than London, but it's less crowded.
\end{tcolorbox}

\section{Key Takeaways}
\begin{itemize}
    \item Comparatives compare TWO things (use "than").
    \item Superlatives compare THREE or more things (use "the").
    \item Short adjectives: add -er/-est.
    \item Long adjectives: use more/most.
    \item Memorize irregular forms: good-better-best, bad-worse-worst.
\end{itemize}

\section{Online Practice}
\begin{tcolorbox}[colback=blue!5,colframe=blue!40!black,title=Resources to Practice]
Online PracticeReinforce what you have learned with these interactive exercises:
\begin{itemize}
    \item \textbf{Comparatives Exercises:} \url{https://www.perfect-english-grammar.com/comparatives-exercises.html}
    \item \textbf{Superlatives Exercises:} \url{https://www.perfect-english-grammar.com/superlatives-exercises.html}
    \item \textbf{Interactive Practice:} \url{https://test-english.com/grammar-points/a2/comparative-superlative-adjectives/}
    \item \textbf{British Council Practice:} \url{https://learnenglish.britishcouncil.org/grammar/english-grammar-reference/comparative-and-superlative-adjectives}
\end{itemize}
\end{tcolorbox}






