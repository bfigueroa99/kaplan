\chapter{Adverbs: Frequency, Manner, and Comment}

\begin{center}
\begin{tabular}{ccc}
\cefrlevel{A2-B1} & \textbf{Study Time:} 2-3 hours & \textbf{Difficulty:} ⭐⭐⭐☆☆
\end{tabular}
\end{center}

\section{Lesson Objectives}
In this chapter, you will learn:
\begin{itemize}
    \item Adverbs of frequency (always, usually, often, sometimes, rarely, never)
    \item Adverbs of manner (quickly, slowly, carefully, etc.)
    \item Adverbs of comment and viewpoint (frankly, obviously, fortunately, etc.)
    \item Where to place adverbs in sentences
    \item How to use adverbs to express opinions professionally
\end{itemize}

\section{Reading Context}
\begin{readingbox}[title=Dialogue: A Typical Day at the Office]
\textbf{Sarah:} Good morning! You're \textbf{always} early today.\\
\textbf{Tom:} Yes, I \textbf{usually} arrive at 8am, but today I came \textbf{quickly} because I'm excited about the meeting.\\
\textbf{Sarah:} \textbf{Frankly}, I'm nervous about the presentation. Do you think it will go well?\\
\textbf{Tom:} \textbf{Obviously}, you're well-prepared. You work \textbf{very carefully} on every detail.\\
\textbf{Sarah:} Thanks! \textbf{Unfortunately}, I \textbf{sometimes} make mistakes under pressure.\\
\textbf{Tom:} Not really. \textbf{Honestly}, you \textbf{rarely} disappoint anyone. Let's go—we \textbf{definitely} don't want to be late.
\end{readingbox}

\section{Grammar Focus: Adverbs of Frequency}

Adverbs of frequency tell us how often something happens. They can modify the main verb or appear in initial position.

\begin{grammarbox}[title=Adverbs of Frequency]
\textbf{Order from Most to Least Frequent:}
\begin{center}
Always (100\%) → Usually (80-90\%) → Often (50-70\%) → Sometimes (20-50\%) → Rarely (5-20\%) → Never (0\%)
\end{center}

\textbf{Position in Sentence:}
\begin{itemize}
    \item \textbf{With main verbs:} Place BEFORE the main verb
    \begin{itemize}
        \item I \textbf{always} drink coffee. \trans{Siempre bebo café}
        \item She \textbf{usually} works from home. \trans{Ella normalmente trabaja desde casa}
    \end{itemize}
    \item \textbf{With auxiliary verbs:} Place AFTER the auxiliary
    \begin{itemize}
        \item I \textbf{am always} tired. \trans{Siempre estoy cansado}
        \item They \textbf{have never} seen that movie. \trans{Nunca han visto esa película}
    \end{itemize}
    \item \textbf{With "to be":} Place AFTER "to be"
    \begin{itemize}
        \item He \textbf{is usually} late. \trans{Normalmente llega tarde}
        \item You \textbf{are always} right. \trans{Siempre tienes razón}
    \end{itemize}
\end{itemize}

\textbf{Examples:}
\begin{itemize}
    \item I \textbf{never} skip breakfast. \trans{Nunca salto el desayuno}
    \item She \textbf{always} arrives on time. \trans{Ella siempre llega a tiempo}
    \item We \textbf{sometimes} work late. \trans{A veces trabajamos hasta tarde}
\end{itemize}
\end{grammarbox}

\section{Adverbs of Manner}

Adverbs of manner describe HOW an action is performed. Most are formed by adding \textbf{-ly} to an adjective.

\begin{grammarbox}[title=Adverbs of Manner]
\textbf{Formation:}
\begin{itemize}
    \item Adjective: \textbf{quick} → Adverb: \textbf{quickly}
    \item Adjective: \textbf{careful} → Adverb: \textbf{carefully}
    \item Adjective: \textbf{happy} → Adverb: \textbf{happily}
    \item Adjective: \textbf{slow} → Adverb: \textbf{slowly}
\end{itemize}

\textbf{Position:}
\begin{itemize}
    \item Generally placed AFTER the verb or at the end of the sentence
    \item She spoke \textbf{confidently}. \trans{Habló con confianza}
    \item They worked \textbf{efficiently} on the project. \trans{Trabajaron eficientemente en el proyecto}
\end{itemize}

\textbf{Common Adverbs of Manner:}
\begin{itemize}
    \item \textbf{Quickly} - fast, rapidly
    \item \textbf{Slowly} - gradually
    \item \textbf{Carefully} - with attention to detail
    \item \textbf{Accurately} - precisely, correctly
    \item \textbf{Easily} - without difficulty
    \item \textbf{Happily} - with joy
    \item \textbf{Confidently} - with assurance
    \item \textbf{Softly} - quietly, gently
\end{itemize}
\end{grammarbox}

\section{Adverbs of Comment and Viewpoint}

These adverbs express the speaker's opinion or attitude about the entire sentence. They usually come at the BEGINNING, followed by a comma.

\begin{grammarbox}[title=Adverbs of Comment and Viewpoint]
\textbf{Structure:}
\begin{center}
\Large \textbf{Adverb} + , + Sentence
\end{center}

\textbf{Common Adverbs of Comment:}
\begin{itemize}
    \item \textbf{Frankly} - to be honest \trans{Francamente}
    \item \textbf{Obviously} - clearly \trans{Obviamente}
    \item \textbf{Fortunately/Unfortunately} - good/bad luck \trans{Afortunadamente/Desafortunadamente}
    \item \textbf{Surprisingly} - unexpectedly \trans{Sorprendentemente}
    \item \textbf{Honestly} - truthfully \trans{Honestamente}
    \item \textbf{Personally} - in my opinion \trans{Personalmente}
    \item \textbf{Clearly} - obviously \trans{Claramente}
    \item \textbf{Apparently} - seemingly \trans{Aparentemente}
    \item \textbf{Ideally} - in a perfect situation \trans{Idealmente}
    \item \textbf{Certainly} - definitely \trans{Ciertamente}
\end{itemize}

\textbf{Examples:}
\begin{itemize}
    \item \textbf{Frankly}, I don't think this will work. \trans{Francamente, no creo que funcione}
    \item \textbf{Unfortunately}, the meeting was cancelled. \trans{Desafortunadamente, se canceló la reunión}
    \item \textbf{Honestly}, I prefer the other option. \trans{Honestamente, prefiero la otra opción}
    \item \textbf{Obviously}, we need a new strategy. \trans{Obviamente, necesitamos una nueva estrategia}
\end{itemize}
\end{grammarbox}

\section{Adverbs of Degree}

Adverbs of degree tell us the INTENSITY or LEVEL of something.

\begin{table}[h]
\centering
\begin{tabular}{|l|l|l|}
\hline
\textbf{Adverb} & \textbf{Meaning} & \textbf{Example} \\
\hline
Very & Much, to a great degree & The film was \textbf{very} good. \\
\hline
Quite & Fairly, rather & The weather is \textbf{quite} cold. \\
\hline
Rather & Fairly, somewhat & The book is \textbf{rather} long. \\
\hline
Too & More than necessary & This coffee is \textbf{too} hot. \\
\hline
Extremely & To a very great degree & The project was \textbf{extremely} successful. \\
\hline
Hardly & Almost not, scarcely & I \textbf{hardly} know him. \\
\hline
\end{tabular}
\caption{Adverbs of degree with examples}
\end{table}

\section{Practice Exercises}

\subsection{Exercise 1: Fill in the Adverb of Frequency}
Complete the sentences with an appropriate adverb of frequency:

\begin{enumerate}
    \item I \underline{\hspace{3cm}} (never/always) drink coffee in the morning.
    \item She \underline{\hspace{3cm}} (rarely/usually) arrives late for work.
    \item They \underline{\hspace{3cm}} (often/never) go to the cinema on weekends.
    \item He \underline{\hspace{3cm}} (sometimes/always) forgets his keys.
\end{enumerate}

\subsection{Exercise 2: Change Adjectives to Adverbs of Manner}
Convert the adjectives to adverbs and use them in sentences:

\begin{enumerate}
    \item Quick → \underline{\hspace{4cm}}
    \item Careful → \underline{\hspace{4cm}}
    \item Slow → \underline{\hspace{4cm}}
    \item Confident → \underline{\hspace{4cm}}
\end{enumerate}

\subsection{Exercise 3: Complete with Adverbs of Comment}
Fill in the blanks with appropriate adverbs of comment:

\begin{enumerate}
    \item \underline{\hspace{2cm}}, I think we should postpone the meeting. (Frankly/Obviously)
    \item \underline{\hspace{2cm}}, the project was completed on time. (Fortunately/Surprisingly)
    \item \underline{\hspace{2cm}}, I don't agree with that decision. (Honestly/Certainly)
\end{enumerate}

\subsection{Exercise 4: Position the Adverb Correctly}
Place the adverb in the correct position in the sentence:

\begin{enumerate}
    \item I work (carefully) on every project.
    \item She is (always) prepared for meetings.
    \item They have (never) been to London.
    \item He speaks English (fluently).
\end{enumerate}

\subsection{Exercise 5: Professional Communication}
Write a short email (4-5 sentences) about a project update, using at least 5 adverbs (mix of frequency, manner, and comment).

\begin{tcolorbox}[colback=white,height=6cm]
\end{tcolorbox}

\subsection{Common Mistakes}

\begin{tcolorbox}[colback=red!5, colframe=red!30, title=Common Adverb Errors]
\textbf{Mistake 1:} Placing frequency adverbs in wrong position
\begin{itemize}
    \item \textcolor{red}{\textbf{Incorrect:}} ``I drink always coffee''
    \item \textcolor{green!60!black}{\textbf{Correct:}} ``I \textbf{always} drink coffee''
\end{itemize}

\textbf{Mistake 2:} Forgetting the -ly ending
\begin{itemize}
    \item \textcolor{red}{\textbf{Incorrect:}} ``She speaks \textbf{confident}''
    \item \textcolor{green!60!black}{\textbf{Correct:}} ``She speaks \textbf{confidently}''
\end{itemize}

\textbf{Mistake 3:} Wrong position with auxiliaries
\begin{itemize}
    \item \textcolor{red}{\textbf{Incorrect:}} ``They never have been there''
    \item \textcolor{green!60!black}{\textbf{Correct:}} ``They \textbf{have never} been there''
\end{itemize}

\textbf{Mistake 4:} Using adverbs with wrong intensity
\begin{itemize}
    \item \textcolor{red}{\textbf{Incorrect:}} ``I very like this'' (overstated)
    \item \textcolor{green!60!black}{\textbf{Correct:}} ``I \textbf{very much like} this'' or ``I \textbf{like} this \textbf{very much}''
\end{itemize}
\end{tcolorbox}

\section{British English Notes}

In British English, some adverbs have different usage patterns:
\begin{itemize}
    \item \textbf{Quite} in British English often means "completely/rather": "It's quite good" (fairly good)
    \item \textbf{Quite} in American English often means "very": "It's quite good" (very good)
    \item \textbf{Truly} is more formal in British English for "really"
    \item British speakers prefer \textbf{certainly} over "definitely" in formal contexts
\end{itemize}

\section{Key Takeaways}
\begin{itemize}
    \item Frequency adverbs go BEFORE the main verb but AFTER auxiliaries and "to be"
    \item Manner adverbs are usually placed AFTER the verb or at sentence-end
    \item Comment adverbs express opinion and usually come at the BEGINNING with a comma
    \item Most manner adverbs are formed by adding \textbf{-ly} to adjectives
    \item Degree adverbs (very, quite, extremely) modify adjectives or other adverbs
\end{itemize}
