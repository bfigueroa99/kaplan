\section{Fast Notes}


\subsection{relative pronunc and relative clauses}

e.g. who/ whose/ that/ which/ where/ when/ Whom

\textbf{Forms:}
\begin{itemize}
    \item \textbf{For people:} Who, That, Whom
    \item \textbf{For location/places:} Where
    \item \textbf{For things:} That, Which
    \item \textbf{For possessions:} Whose
    \item \textbf{For time:} When
\end{itemize}


\subsubsection{Who vs that vs Whom}

\textbf{Who:} is used as the subject of a sentence.\\
\textbf{That:} can be used as the subject or object of a sentence.\\
\textbf{Whom:} is used as the object of a sentence (formal). 

\textbf{Examples:}
\begin{itemize}
    \item The man \textbf{who} is talking to Sarah is my brother.
    \item The book \textbf{that} you lent me was fascinating.
    \item The woman \textbf{whom} I met at the conference is a renowned scientist.
\end{itemize}

\subsubsection{That vs Which}

\textbf{That:} is used in restrictive clauses (essential information).\\
\textbf{Which:} is used in non-restrictive clauses (additional information, usually set off by commas). \\
\textbf{Examples:}
\begin{itemize}
    \item The car \textbf{that} I bought last year is very reliable.
    \item My car, \textbf{which} I bought last year, is very reliable.
\end{itemize}



-------------------------------------------------------------------------------------

a) which

b) Whose

c) who

d) who

e) where

f) where



