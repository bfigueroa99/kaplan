\chapter{Technology and Communication}

\begin{center}
\begin{tabular}{ccc}
\cefrlevel{B2-C1} & \textbf{Study Time:} 3-4 hours & \textbf{Difficulty:} ⭐⭐⭐⭐☆
\end{tabular}
\end{center}

\section{Lesson Objectives}
In this chapter, you will learn:
\begin{itemize}
    \item Vocabulary related to computers and technology
    \item How to explain technical concepts in simple terms
    \item Communication strategies for checking understanding
    \item Present your interests and abilities
\end{itemize}

\section{Speaking Activity: Your Unique Talent or Interest}

\begin{readingbox}[title=Personal Interests]
Everyone has unique talents, abilities, or interests that make them special. Being able to talk about your passions in English is an important communication skill.

\textbf{Sample Response:}

I'm interested in mountain biking. I started riding when I was 14 years old. My first track in this sport was quite challenging, but it got me hooked. Generally, I ride with my brother now. 

In mountain biking, we talk about different "lines" - these refer to the different paths that you can take on a track. Each line has its own challenges and requires different skills.
\end{readingbox}

\subsection{Discussion Questions}
\begin{enumerate}
    \item What is your unique talent, ability, or interest?
    \item When did you start?
    \item Who do you share this interest with?
    \item What do you enjoy most about it?
\end{enumerate}

\section{Technology Vocabulary}

\begin{spanishbox}[title=Para Hispanohablantes: Technology False Friends]
⚠️ \textbf{Cuidado con estos "False Friends" tecnológicos:}

\begin{itemize}
    \item ❌ "Actually" ≠ "actualmente" → \textbf{Actually} = "en realidad" / "de hecho"
    \item ✓ Currently = actualmente
    \item ❌ "Realize" ≠ "realizar" → \textbf{Realize} = "darse cuenta"
    \item ✓ To carry out = realizar
    \item ❌ "Library" ≠ "librería" → \textbf{Library} (code) = "biblioteca de código"
    \item ✓ Bookshop = librería (tienda)
\end{itemize}

💡 \textbf{Truco:} En español decimos "aplicación móvil", en inglés británico decimos \textbf{mobile app} o simplemente \textbf{app}.
\end{spanishbox}

\subsection{Computers and Software Development}

\begin{vocabbox}[title=Technology and Computing Terms]
\begin{itemize}
    \item \keyterm{Front-end} \trans{Interfaz de usuario / Cliente}
    \begin{itemize}
        \item The part of a website or application that users see and interact with
        \item Example: "The front-end developer designs the user interface."
    \end{itemize}
    \item \keyterm{Back-end} \trans{Servidor / Sistema de fondo}
    \begin{itemize}
        \item The part that works behind the scenes to make everything function
        \item Example: "The back-end handles data processing and storage."
    \end{itemize}
    \item \keyterm{Database} \trans{Base de datos}
    \begin{itemize}
        \item An organized collection of data stored electronically
        \item Example: "User information is stored in the database."
    \end{itemize}
    \item \keyterm{Framework} \trans{Marco de trabajo}
    \begin{itemize}
        \item A set of tools and libraries that help developers create applications more easily
        \item Example: "React is a popular front-end framework."
    \end{itemize}
    \item \keyterm{API (Application Programming Interface)} \trans{Interfaz de programación de aplicaciones}
    \begin{itemize}
        \item A way for different software applications to communicate with each other
        \item Example: "The front-end sends requests to the back-end through an API."
    \end{itemize}
\end{itemize}
\end{vocabbox}

\begin{examplebox}[title=Explaining Technical Concepts]
\textbf{Software Architecture Explanation:}

To explain the architecture of any app's software, there are generally two principal components: the \textbf{front-end} and the \textbf{back-end}, plus a \textbf{database}.

\begin{itemize}
    \item The \textbf{front-end} is the part that users see and interact with
    \item The \textbf{back-end} works behind the scenes to make everything function properly
    \item These two components work together to provide a seamless user experience
\end{itemize}

\textbf{Communication between components:}

The communication between front-end and back-end typically occurs through \textbf{APIs} (Application Programming Interfaces). The front-end sends requests to the back-end server, which processes these requests, interacts with the database if necessary, and then sends back the appropriate responses. This allows for dynamic content updates without needing to reload the entire page.
\end{examplebox}

\subsection{Popular Frameworks}
\begin{itemize}
    \item \textbf{React}: Front-end framework based on JavaScript, owned by Meta (Facebook, Instagram, WhatsApp)
    \item \textbf{Django}: Back-end framework based on Python, owned by Django Software Foundation
\end{itemize}

\section{Communication Skills: Checking Understanding}

When explaining complex ideas or learning new concepts, it's important to check understanding.

\begin{grammarbox}[title=Phrases for Checking Understanding]
\textbf{When you need clarification:}
\begin{itemize}
    \item Could you \textbf{explain} what you mean by that?
    \item Sorry, I'm not sure if I \textbf{understand}. What's [concept]?
    \item Could you \textbf{clarify} that point?
    \item What exactly do you mean by...?
\end{itemize}

\textbf{When checking if others understand you:}
\begin{itemize}
    \item Is everything \textbf{clear} so far?
    \item Does that make sense?
    \item Do you follow me?
    \item Are there any questions?
    \item Do you see what I \textbf{mean}?
\end{itemize}

\textbf{When confirming understanding:}
\begin{itemize}
    \item So, if I \textbf{understand} you correctly...
    \item Let me see if I've got this right...
    \item Just to clarify, you're saying that...
    \item In other words...
\end{itemize}
\end{grammarbox}

\section{Practice Exercises}

\subsection{Exercise 1: Complete the Phrases}
Fill in the blanks with appropriate words:

\begin{enumerate}
    \item Could you \underline{\hspace{3cm}} what you mean by that?
    \item Sorry, I'm not sure if I \underline{\hspace{3cm}}. What's identity theft?
    \item Is everything \underline{\hspace{3cm}} so far?
    \item So, if I \underline{\hspace{3cm}} you correctly...
    \item Do you see what I \underline{\hspace{3cm}}?
\end{enumerate}

\textbf{Answers:} 1. explain, 2. understand/follow, 3. clear, 4. understand, 5. mean

\subsection{Exercise 2: Explain Your Interest}
Write a short paragraph (80-100 words) explaining a hobby, interest, or talent you have. Use some of the communication phrases from this chapter.

\begin{tcolorbox}[colback=white,height=6cm]
% Write your explanation here
\end{tcolorbox}

\section{Globalization and the Modern World}

\subsection{Globalization Vocabulary}

\begin{vocabbox}[title=Key Terms for Globalization]
\begin{itemize}
    \item \keyterm{Trade} \trans{Comercio} - The buying and selling of goods and services between countries
    \item \keyterm{Profit} \trans{Beneficio/Ganancia} - Money gained from business activities
    \item \keyterm{Migration} \trans{Migración} - Movement of people from one region/country to another
    \item \keyterm{Investment} \trans{Inversión} - Money put into a business to make more money
    \item \keyterm{Tax} \trans{Impuesto} - Money paid to the government
    \item \keyterm{Economy} \trans{Economía} - The system of production and consumption in a country
    \item \keyterm{Multinational} \trans{Multinacional} - A company operating in multiple countries
    \item \keyterm{Goods} \trans{Bienes/Productos} - Physical products
    \item \keyterm{Laws} \trans{Leyes} - Rules created by governments
    \item \keyterm{Transform} \trans{Transformar} - To change completely
\end{itemize}
\end{vocabbox}

\subsection{Word Formation: Employment and Development}

\begin{table}[h]
\centering
\begin{tabular}{|l|l|l|}
\hline
\textbf{Base Form} & \textbf{Related Forms} & \textbf{Meaning} \\
\hline
Employ & Employment, Employer & Hire, Job, Person who hires \\
\hline
Develop & Development, Developed & Growth, Process of growth, Advanced \\
\hline
Improve & Improvement, Improved & Make better, Process of making better, Enhanced \\
\hline
Sustain & Sustainability, Sustainable & Maintain, Ability to maintain, Maintainable \\
\hline
\end{tabular}
\end{table}

\subsection{Reading: The History of Globalization}

\begin{readingbox}[title=Globalization Through History]
\textbf{The Silk Road:} One of the earliest examples of globalization was the ancient Silk Road, which connected East Asia with Europe and the Middle East through trade routes.

\textbf{Modern Globalization:} Today's globalization has been driven by:
\begin{itemize}
    \item Improvements in transport (faster ships, airplanes)
    \item Free-market policies (fewer trade barriers)
    \item Technology (internet, communication)
    \item Increased knowledge sharing
\end{itemize}

\textbf{Effects on Different Countries:} Globalization affects countries differently. In poorer countries, it can bring jobs and investment but may also exploit workers. In developed countries, it can lead to job losses in manufacturing but create opportunities in services and technology.

\textbf{Regulations Needed:} Laws are needed for:
\begin{itemize}
    \item Protection of human rights
    \item Regulations about pollution and environmental protection
    \item Cultural preservation
\end{itemize}
\end{readingbox}

\subsection{Writing Task: Globalization Essay}

\textbf{Essay Question:} How has globalization affected your country? Give examples and state your opinion about its effects.

\begin{tcolorbox}[colback=green!5, colframe=green!30, title=Sample Essay: Globalization in Chile]
Globalization has had a significant impact on countries around the world, including my own country. One of the most notable effects of globalization is the increased trade and economic interdependence between nations.

In my country, Chile, globalization has given us access to develop primary industries such as mining, agriculture, and forestry. This has led to increased exports and foreign investment, which has helped to boost the economy and create jobs. For example, Chile is one of the world's largest producers of copper, and globalization has allowed us to export this valuable resource to countries around the world.

However, these industries have had a negative impact on the environment, including deforestation, pollution, and habitat destruction. Additionally, globalization has led to increased competition from foreign companies, which has put pressure on local businesses and workers.

Overall, I believe that globalization has had both positive and negative effects on my country. There are many challenges, where we believe in finding the balance between economic growth and other responsibilities such as environmental protection and social welfare.
\end{tcolorbox}

\section{Ecotourism and Sustainable Travel}

\begin{grammarbox}[title=What is Ecotourism?]
\textbf{Ecotourism} is a form of sustainable travel that focuses on exploring natural environments while minimizing the impact on the ecosystem.

\textbf{Key Principles:}
\begin{itemize}
    \item Responsible travel to natural areas
    \item Minimal environmental impact
    \item Support for local communities
    \item Educational experiences
    \item Conservation of biodiversity
\end{itemize}

It aims to promote conservation, support local communities, and educate travelers about the importance of preserving the environment. Ecotourism often involves activities such as wildlife watching, hiking, and visiting protected areas, allowing tourists to experience nature in a responsible and respectful manner.
\end{grammarbox}

\subsection{Ecotourism Vocabulary}

\begin{vocabbox}[title=Sustainable Tourism Terms]
\begin{itemize}
    \item \keyterm{Economic development} \trans{Desarrollo económico}
    \item \keyterm{Responsible travel} \trans{Viaje responsable}
    \item \keyterm{Wildlife sanctuary} \trans{Santuario de vida silvestre}
    \item \keyterm{Carbon footprint} \trans{Huella de carbono}
    \item \keyterm{Native people} \trans{Pueblos nativos/indígenas}
    \item \keyterm{Cultural heritage} \trans{Patrimonio cultural}
    \item \keyterm{Environmental conservation} \trans{Conservación ambiental}
\end{itemize}
\end{vocabbox}

\subsection{Suffixes and Prefixes for Academic Writing}

\subsubsection{Suffixes}

A suffix is a letter or group of letters added at the end of a word to change its meaning or grammatical function.

\begin{table}[h!]
    \centering
    \begin{tabular}{|l|l|l|}
    \hline
    \textbf{Suffix} & \textbf{Function / Meaning} & \textbf{Example} \\ \hline
    -ness & State or quality (forms nouns) & Happiness (Happy + ness) \\ \hline
    -ly & Manner (forms adverbs) & Quickly (Quick + ly) \\ \hline
    -er / -or & Person who performs an action & Teacher / Actor \\ \hline
    -tion & Action or process & Celebration \\ \hline
    -ful & Full of & Helpful \\ \hline
    -less & Without & Homeless \\ \hline
    \end{tabular}
    \caption{Common Suffixes}
\end{table}

\subsubsection{Prefixes}

A prefix is a letter or group of letters added at the beginning of a word to modify its meaning.

\begin{table}[h!]
    \centering
    \begin{tabular}{|l|l|l|}
    \hline
    \textbf{Prefix} & \textbf{Meaning} & \textbf{Example} \\ 
    \hline
    un- & Not / Opposite & Unhappy (un + happy) \\ 
    \hline
    re- & Again & Redo (re + do) \\ 
    \hline
    dis- & Not / Opposite & Disconnect \\ 
    \hline
    pre- & Before & Preview \\ 
    \hline
    mis- & Wrongly & Misunderstand \\ 
    \hline
    im- & Not & Impossible \\ 
    \hline
    \end{tabular}
    \caption{Common Prefixes}
\end{table}

\subsection{Writing Task: Ecotourism Essay}

\textbf{Essay Title:} Do the disadvantages of ecotourism outweigh the advantages?

\textbf{Use a for-and-against structure:}
\begin{itemize}
    \item Introduction: State the topic
    \item Body Paragraph 1: Advantages
    \item Body Paragraph 2: Disadvantages
    \item Conclusion: Your opinion
\end{itemize}

\begin{tcolorbox}[colback=blue!5, colframe=blue!30, title=Sample Essay: Ecotourism]
In recent years, ecotourism has become popular, and sparked a debate about its advantages and disadvantages. In this article, we will talk about the different arguments.

One of the advantages is the increase in tourism that leads to more profit, and with this, you can improve the facilities and security for local people and the environment. On the other hand, mass tourism can lead to destruction of the environment and disrupt animal behavior.

In summary, there are more benefits than negative aspects. The main advantages include promoting conservation and supporting local economies.
\end{tcolorbox}

\section{Key Takeaways}
\begin{itemize}
    \item Use simple language when explaining technical concepts
    \item Always check understanding when discussing complex topics
    \item Break down explanations into smaller, manageable parts
    \item Use examples to illustrate abstract concepts
    \item Practice active listening and ask for clarification when needed
\end{itemize}


\section{Online Practice}
Here are some useful websites to practice this topic:
\begin{itemize}
    \item \href{https://learnenglish.britishcouncil.org/business-english/vocabulary}{British Council LearnEnglish}
    \item \href{https://www.cambridgeenglish.org/learning-english/activities-for-learners/?skill=reading&level=independent&rows=12}{Cambridge English}
\end{itemize}
