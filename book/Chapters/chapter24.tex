\chapter{Technology and Communication}

\begin{center}
\begin{tabular}{ccc}
\cefrlevel{B2-C1} & \textbf{Study Time:} 3-4 hours & \textbf{Difficulty:} ⭐⭐⭐⭐☆
\end{tabular}
\end{center}

\section{Lesson Objectives}
In this chapter, you will learn:
\begin{itemize}
    \item Vocabulary related to computers and technology
    \item How to explain technical concepts in simple terms
    \item Communication strategies for checking understanding
    \item Present your interests and abilities
\end{itemize}

\section{Speaking Activity: Your Unique Talent or Interest}

\begin{readingbox}[title=Personal Interests]
Everyone has unique talents, abilities, or interests that make them special. Being able to talk about your passions in English is an important communication skill.

\textbf{Sample Response:}

I'm interested in mountain biking. I started riding when I was 14 years old. My first track in this sport was quite challenging, but it got me hooked. Generally, I ride with my brother now. 

In mountain biking, we talk about different "lines" - these refer to the different paths that you can take on a track. Each line has its own challenges and requires different skills.
\end{readingbox}

\subsection{Discussion Questions}
\begin{enumerate}
    \item What is your unique talent, ability, or interest?
    \item When did you start?
    \item Who do you share this interest with?
    \item What do you enjoy most about it?
\end{enumerate}

\section{Technology Vocabulary}

\subsection{Computers and Software Development}

\begin{vocabbox}[title=Technology and Computing Terms]
\begin{itemize}
    \item \keyterm{Front-end} \trans{Interfaz de usuario / Cliente}
    \begin{itemize}
        \item The part of a website or application that users see and interact with
        \item Example: "The front-end developer designs the user interface."
    \end{itemize}
    \item \keyterm{Back-end} \trans{Servidor / Sistema de fondo}
    \begin{itemize}
        \item The part that works behind the scenes to make everything function
        \item Example: "The back-end handles data processing and storage."
    \end{itemize}
    \item \keyterm{Database} \trans{Base de datos}
    \begin{itemize}
        \item An organized collection of data stored electronically
        \item Example: "User information is stored in the database."
    \end{itemize}
    \item \keyterm{Framework} \trans{Marco de trabajo}
    \begin{itemize}
        \item A set of tools and libraries that help developers create applications more easily
        \item Example: "React is a popular front-end framework."
    \end{itemize}
    \item \keyterm{API (Application Programming Interface)} \trans{Interfaz de programación de aplicaciones}
    \begin{itemize}
        \item A way for different software applications to communicate with each other
        \item Example: "The front-end sends requests to the back-end through an API."
    \end{itemize}
\end{itemize}
\end{vocabbox}

\begin{examplebox}[title=Explaining Technical Concepts]
\textbf{Software Architecture Explanation:}

To explain the architecture of any app's software, there are generally two principal components: the \textbf{front-end} and the \textbf{back-end}, plus a \textbf{database}.

\begin{itemize}
    \item The \textbf{front-end} is the part that users see and interact with
    \item The \textbf{back-end} works behind the scenes to make everything function properly
    \item These two components work together to provide a seamless user experience
\end{itemize}

\textbf{Communication between components:}

The communication between front-end and back-end typically occurs through \textbf{APIs} (Application Programming Interfaces). The front-end sends requests to the back-end server, which processes these requests, interacts with the database if necessary, and then sends back the appropriate responses. This allows for dynamic content updates without needing to reload the entire page.
\end{examplebox}

\subsection{Popular Frameworks}
\begin{itemize}
    \item \textbf{React}: Front-end framework based on JavaScript, owned by Meta (Facebook, Instagram, WhatsApp)
    \item \textbf{Django}: Back-end framework based on Python, owned by Django Software Foundation
\end{itemize}

\section{Communication Skills: Checking Understanding}

When explaining complex ideas or learning new concepts, it's important to check understanding.

\begin{grammarbox}[title=Phrases for Checking Understanding]
\textbf{When you need clarification:}
\begin{itemize}
    \item Could you \textbf{explain} what you mean by that?
    \item Sorry, I'm not sure if I \textbf{understand}. What's [concept]?
    \item Could you \textbf{clarify} that point?
    \item What exactly do you mean by...?
\end{itemize}

\textbf{When checking if others understand you:}
\begin{itemize}
    \item Is everything \textbf{clear} so far?
    \item Does that make sense?
    \item Do you follow me?
    \item Are there any questions?
    \item Do you see what I \textbf{mean}?
\end{itemize}

\textbf{When confirming understanding:}
\begin{itemize}
    \item So, if I \textbf{understand} you correctly...
    \item Let me see if I've got this right...
    \item Just to clarify, you're saying that...
    \item In other words...
\end{itemize}
\end{grammarbox}

\section{Practice Exercises}

\subsection{Exercise 1: Complete the Phrases}
Fill in the blanks with appropriate words:

\begin{enumerate}
    \item Could you \underline{\hspace{3cm}} what you mean by that?
    \item Sorry, I'm not sure if I \underline{\hspace{3cm}}. What's identity theft?
    \item Is everything \underline{\hspace{3cm}} so far?
    \item So, if I \underline{\hspace{3cm}} you correctly...
    \item Do you see what I \underline{\hspace{3cm}}?
\end{enumerate}

\textbf{Answers:} 1. explain, 2. understand/follow, 3. clear, 4. understand, 5. mean

\subsection{Exercise 2: Explain Your Interest}
Write a short paragraph (80-100 words) explaining a hobby, interest, or talent you have. Use some of the communication phrases from this chapter.

\begin{tcolorbox}[colback=white,height=6cm]
% Write your explanation here
\end{tcolorbox}

\section{Key Takeaways}
\begin{itemize}
    \item Use simple language when explaining technical concepts
    \item Always check understanding when discussing complex topics
    \item Break down explanations into smaller, manageable parts
    \item Use examples to illustrate abstract concepts
    \item Practice active listening and ask for clarification when needed
\end{itemize}
