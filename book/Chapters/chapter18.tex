\chapter{The Passive Voice}

\section{Lesson Objectives}
In this chapter, you will learn:
\begin{itemize}
    \item What the passive voice is and when to use it
    \item How to form the passive in different tenses
    \item The difference between active and passive voice
    \item When the passive is preferred
    \item The use of "by + agent" in passive sentences
\end{itemize}

\section{Reading Context}
\begin{readingbox}[title=Dialogue: News Report]
\textbf{Reporter:} This historic building \textbf{was built} in 1850.\\
\textbf{Tourist:} Really? Who was it built by?\\
\textbf{Reporter:} It \textbf{was designed} by a famous architect, John Smith. The materials \textbf{were imported} from Italy.\\
\textbf{Tourist:} Is it still used today?\\
\textbf{Reporter:} Yes! It \textbf{is used} as a museum now. Thousands of visitors \textbf{are welcomed} every year.\\
\textbf{Tourist:} That's wonderful. \textbf{Will} it \textbf{be renovated} soon?\\
\textbf{Reporter:} Yes, renovations \textbf{have been scheduled} for next year.
\end{readingbox}

\section{Grammar Focus: Active vs. Passive}

\begin{grammarbox}[title=Understanding the Difference]
\textbf{Active Voice:} The subject DOES the action.
\begin{itemize}
    \item Subject + Verb + Object
    \item \textbf{Shakespeare} (subject) \textbf{wrote} (verb) \textbf{Hamlet} (object).
\end{itemize}

\textbf{Passive Voice:} The subject RECEIVES the action.
\begin{itemize}
    \item Subject + BE + Past Participle (+ by agent)
    \item \textbf{Hamlet} (subject) \textbf{was written} (verb) \textbf{by Shakespeare} (agent).
\end{itemize}

\textbf{Focus Changes:}
\begin{itemize}
    \item Active: Focus on WHO did it (Shakespeare)
    \item Passive: Focus on WHAT happened (Hamlet was written)
\end{itemize}
\end{grammarbox}

\section{When to Use the Passive}

\begin{vocabbox}[title=Use Passive When...]
\textbf{1. The action is more important than who did it:}
\begin{itemize}
    \item English \textbf{is spoken} in many countries. \trans{Se habla inglés...}
\end{itemize}

\textbf{2. We don't know who did it:}
\begin{itemize}
    \item My car \textbf{was stolen} last night. \trans{Mi coche fue robado} (we don't know who)
\end{itemize}

\textbf{3. It's obvious who did it:}
\begin{itemize}
    \item He \textbf{was arrested}. \trans{Fue arrestado} (obviously by police)
\end{itemize}

\textbf{4. We want to be formal or impersonal:}
\begin{itemize}
    \item Smoking \textbf{is not permitted}. \trans{No se permite fumar}
\end{itemize}

\textbf{5. In scientific writing and reports:}
\begin{itemize}
    \item The experiment \textbf{was conducted} in a laboratory.
\end{itemize}
\end{vocabbox}

\section{How to Form the Passive}

\begin{grammarbox}[title=Passive Structure]
\textbf{Formula:} Subject + \textbf{BE} (correct tense) + \textbf{Past Participle (V3)} + (by agent)

The form of BE changes according to the tense!
\end{grammarbox}

\subsection{Present Simple Passive}

\begin{grammarbox}[title=Present Simple Passive]
\textbf{Structure:} am/is/are + past participle

\textbf{Active:} They \textbf{make} cars in Germany.

\textbf{Passive:} Cars \textbf{are made} in Germany.

\textbf{More Examples:}
\begin{itemize}
    \item English \textbf{is spoken} here. \trans{Se habla inglés aquí}
    \item The office \textbf{is cleaned} every day. \trans{La oficina se limpia cada día}
    \item Letters \textbf{are delivered} in the morning. \trans{Las cartas se entregan por la mañana}
\end{itemize}
\end{grammarbox}

\subsection{Past Simple Passive}

\begin{grammarbox}[title=Past Simple Passive]
\textbf{Structure:} was/were + past participle

\textbf{Active:} Shakespeare \textbf{wrote} Hamlet.

\textbf{Passive:} Hamlet \textbf{was written} by Shakespeare.

\textbf{More Examples:}
\begin{itemize}
    \item The house \textbf{was built} in 1950. \trans{La casa fue construida en 1950}
    \item The documents \textbf{were signed} yesterday. \trans{Los documentos fueron firmados ayer}
    \item America \textbf{was discovered} in 1492. \trans{América fue descubierta en 1492}
\end{itemize}
\end{grammarbox}

\subsection{Present Continuous Passive}

\begin{grammarbox}[title=Present Continuous Passive]
\textbf{Structure:} am/is/are + being + past participle

\textbf{Active:} They \textbf{are repairing} the road.

\textbf{Passive:} The road \textbf{is being repaired}.

\textbf{More Examples:}
\begin{itemize}
    \item The project \textbf{is being completed}. \trans{El proyecto se está completando}
    \item New houses \textbf{are being built}. \trans{Se están construyendo casas nuevas}
\end{itemize}
\end{grammarbox}

\subsection{Present Perfect Passive}

\begin{grammarbox}[title=Present Perfect Passive]
\textbf{Structure:} have/has + been + past participle

\textbf{Active:} Someone \textbf{has stolen} my bike.

\textbf{Passive:} My bike \textbf{has been stolen}.

\textbf{More Examples:}
\begin{itemize}
    \item The report \textbf{has been finished}. \trans{El informe ha sido terminado}
    \item The emails \textbf{have been sent}. \trans{Los emails han sido enviados}
\end{itemize}
\end{grammarbox}

\subsection{Future Simple Passive}

\begin{grammarbox}[title=Future Simple Passive]
\textbf{Structure:} will + be + past participle

\textbf{Active:} They \textbf{will complete} the project next week.

\textbf{Passive:} The project \textbf{will be completed} next week.

\textbf{More Examples:}
\begin{itemize}
    \item The meeting \textbf{will be held} tomorrow. \trans{La reunión se realizará mañana}
    \item The results \textbf{will be announced} soon. \trans{Los resultados serán anunciados pronto}
\end{itemize}
\end{grammarbox}

\subsection{Modal Passive}

\begin{grammarbox}[title=Modal Passive]
\textbf{Structure:} modal + be + past participle

\textbf{Examples:}
\begin{itemize}
    \item This \textbf{can be done} easily. \trans{Esto se puede hacer fácilmente}
    \item The form \textbf{must be signed}. \trans{El formulario debe ser firmado}
    \item It \textbf{should be finished} by Friday. \trans{Debería estar terminado para el viernes}
    \item The door \textbf{may be opened}. \trans{La puerta puede ser abierta}
\end{itemize}
\end{grammarbox}

\section{Summary Table of Passive Forms}

\begin{table}[h]
\centering
\begin{tabular}{|l|p{4cm}|p{5cm}|}
\hline
\textbf{Tense} & \textbf{Active} & \textbf{Passive} \\
\hline
Present Simple & They make cars. & Cars \textbf{are made}. \\
\hline
Past Simple & She wrote the book. & The book \textbf{was written}. \\
\hline
Present Continuous & They are fixing it. & It \textbf{is being fixed}. \\
\hline
Present Perfect & He has stolen it. & It \textbf{has been stolen}. \\
\hline
Future Simple & They will build it. & It \textbf{will be built}. \\
\hline
Modal & You can do it. & It \textbf{can be done}. \\
\hline
\end{tabular}
\caption{Passive forms in different tenses}
\end{table}

\section{Using "BY + Agent"}

\begin{grammarbox}[title=When to Include the Agent]
\textbf{Include "by + agent" when the agent is:}
\begin{itemize}
    \item Important or relevant
    \item Famous or specific
    \item Surprising or unexpected
\end{itemize}

\textbf{Examples:}
\begin{itemize}
    \item Hamlet was written \textbf{by Shakespeare}. (famous author - important)
    \item The theory was developed \textbf{by Einstein}. (famous scientist)
    \item My car was repaired \textbf{by my neighbor}. (surprising/specific)
\end{itemize}

\textbf{Omit "by + agent" when:}
\begin{itemize}
    \item It's obvious (arrested = by police)
    \item It's unknown (my bike was stolen)
    \item It's unimportant (English is spoken here)
\end{itemize}
\end{grammarbox}

\section{Passive in Different Contexts}

\begin{examplebox}[title=Formal Writing and Instructions]
\textbf{Signs and Rules:}
\begin{itemize}
    \item Smoking \textbf{is not permitted}. \trans{No se permite fumar}
    \item Credit cards \textbf{are accepted}. \trans{Se aceptan tarjetas}
    \item Dogs \textbf{must be kept} on a leash. \trans{Los perros deben llevarse con correa}
\end{itemize}

\textbf{News Reports:}
\begin{itemize}
    \item A man \textbf{was arrested} yesterday.
    \item The building \textbf{was destroyed} by fire.
    \item New laws \textbf{have been introduced}.
\end{itemize}

\textbf{Processes and Traditions:}
\begin{itemize}
    \item Wine \textbf{is made} from grapes. \trans{El vino se hace de uvas}
    \item Christmas \textbf{is celebrated} in December. \trans{La Navidad se celebra en diciembre}
\end{itemize}
\end{examplebox}

\section{Practice Exercises}

\subsection{Exercise 1: Active to Passive}
Transform these active sentences into passive voice.

\begin{enumerate}
    \item They speak English in Australia.\\
    $\rightarrow$ English \underline{\hspace{7cm}}
    \item Shakespeare wrote Romeo and Juliet.\\
    $\rightarrow$ Romeo and Juliet \underline{\hspace{7cm}}
    \item Someone has stolen my wallet.\\
    $\rightarrow$ My wallet \underline{\hspace{7cm}}
    \item They will finish the project tomorrow.\\
    $\rightarrow$ The project \underline{\hspace{7cm}}
    \item You must sign the form.\\
    $\rightarrow$ The form \underline{\hspace{7cm}}
\end{enumerate}

\subsection{Exercise 2: Choose Active or Passive}
Choose the correct form.

\begin{enumerate}
    \item The letter (delivered / was delivered) this morning.
    \item They (build / are built) houses here.
    \item English (speaks / is spoken) in many countries.
    \item Someone (has broken / has been broken) the window.
    \item The meeting (will hold / will be held) tomorrow.
\end{enumerate}

\subsection{Exercise 3: Complete with the Correct Passive Form}
\begin{enumerate}
    \item The house \underline{\hspace{3cm}} (build) in 1990. (Past Simple)
    \item The report \underline{\hspace{3cm}} (write) right now. (Present Continuous)
    \item The emails \underline{\hspace{3cm}} (send) yesterday. (Past Simple)
    \item The project \underline{\hspace{3cm}} (complete) next week. (Future Simple)
    \item This problem \underline{\hspace{3cm}} (can/solve) easily. (Modal)
\end{enumerate}

\subsection{Exercise 4: Add "by agent" if necessary}
Rewrite in passive voice. Only add "by + agent" if it's important.

\begin{enumerate}
    \item Someone stole my phone. $\rightarrow$ \underline{\hspace{7cm}}
    \item Picasso painted this picture. $\rightarrow$ \underline{\hspace{7cm}}
    \item The police arrested him. $\rightarrow$ \underline{\hspace{7cm}}
    \item People speak Spanish in Mexico. $\rightarrow$ \underline{\hspace{7cm}}
\end{enumerate}

\subsection{Exercise 5: Correct the Errors}
\begin{enumerate}
    \item The house was build in 1950.
    \item English is speaking here.
    \item The report has been wrote.
    \item It will be finish tomorrow.
    \item The car is repairing now.
\end{enumerate}

\subsection{Exercise 6: Writing Task}
Write 5 sentences about your city/country using passive voice (e.g., what language is spoken, what products are made, famous buildings, etc.).

\begin{tcolorbox}[colback=white,height=6cm]
% Example: Spanish is spoken in my country. The Sagrada Familia was designed by Gaudí.
\end{tcolorbox}

\section{Key Takeaways}
\begin{itemize}
    \item \textbf{Passive structure:} BE (correct tense) + past participle (V3)
    \item Use passive when the action is more important than who did it
    \item Use passive when the agent is unknown, obvious, or unimportant
    \item Only include "by + agent" when the agent is important or famous
    \item Remember: The tense is shown in the form of BE
    \item Common in formal writing, news, scientific reports, and instructions
\end{itemize}


\section{Active vs Passive Voice}

\subsection{Understanding the Difference}

Voice refers to the form of a verb that indicates when a subject performs the action or is the receiver of the action.

\begin{description}
    \item[Active Voice:] The subject \textbf{performs} the action. It is direct, clear, and concise.
    \item[Passive Voice:] The subject \textbf{receives} the action. Focus shifts to the action itself or the recipient.
\end{description}

\subsection{Active Voice Structure}

In the active voice, the person or thing doing the action is the subject of the sentence.

\bigskip
\textbf{Formula:}
\[ \text{Subject} + \text{Verb} + \text{Object} \]
\bigskip

\textbf{Examples:}
\begin{itemize}
    \item \textbf{The cat} (subject) \textit{chased} (verb) \textbf{the mouse} (object).
    \item \textbf{She} is \textit{reading} \textbf{a book}.
    \item \textbf{They} will \textit{complete} \textbf{the project}.
\end{itemize}

\subsection{Passive Voice Structure}

In the passive voice, the person or thing executing the action is often less important or unknown. The object of the active sentence becomes the subject of the passive sentence.

\bigskip
\textbf{Formula:}
\[ \text{Object (new Subject)} + \text{Form of ``to be''} + \text{Past Participle} + (\text{by Agent}) \]
\bigskip

\textbf{Examples:}
\begin{itemize}
    \item \textbf{The mouse} (new subject) \textit{was chased} by \textbf{the cat} (agent).
    \item \textbf{A book} \textit{is being read} by \textbf{her}.
    \item \textbf{The project} \textit{will be completed} by \textbf{them}.
\end{itemize}

\subsection{When to Use Passive Voice}

The passive voice is a stylistic choice, often used when:
\begin{enumerate}
    \item \textbf{The focus is on the action/result rather than the doer.} \\
    \textit{Example: ``My bike was stolen.'' (The focus is on the theft, not the thief).}
    
    \item \textbf{The doer is unknown, obvious, or irrelevant.} \\
    \textit{Example: ``Cotton is grown in Egypt.'' (It doesn't matter who specifically grows it).}
    
    \item \textbf{In formal or scientific writing (to remain objective).} \\
    \textit{Example: ``The solution was heated to 100 degrees.''}
\end{enumerate}

\subsection{Tense Transformation Examples}

\begin{center}
\begin{tabular}{|l|l|l|}
\hline
\textbf{Tense} & \textbf{Active Voice} & \textbf{Passive Voice} \\ \hline
Present Simple & The chef cooks the meal. & The meal \textbf{is cooked} by the chef. \\ \hline
Past Simple & The company launched it. & It \textbf{was launched} by the company. \\ \hline
Present Continuous & I am painting the wall. & The wall \textbf{is being painted}. \\ \hline
Future Simple & They will fix the roof. & The roof \textbf{will be fixed}. \\ \hline
Present Perfect & They have finished it. & It \textbf{has been finished}. \\ \hline
\end{tabular}
\end{center}


\subsection{Comprehensive Tense Transformation Table}   


\begin{center}
\begin{tabular}{|l|l|l|}
\hline
\textbf{Tense} & \textbf{Active Voice} & \textbf{Passive Voice} \\ 
\hline
Present Simple & I make a cake. & A cake is made by me. \\
\hline
Present Continuous & I am making a cake. & A cake is being made by me. \\
\hline
Past Simple & I made a cake. & A cake was made by me. \\
\hline
Past Continuous & I was making a cake. & A cake was being made by me. \\
\hline
Present Perfect & I have made a cake. & A cake has been made by me. \\
\hline
Present Perfect Continuous & I have been making a cake. & A cake has been being made by me. \\
\hline
Past Perfect & I had made a cake. & A cake had been made by me. \\
\hline
Future Simple & I will make a cake. & A cake will be made by me. \\
\hline
Future Perfect & I will have made a cake. & A cake will have been made by me. \\
\hline
\end{tabular}
\end{center}

