\chapter{Modal Verbs: Can, Could, May, Might}

\begin{center}
\begin{tabular}{ccc}
\cefrlevel{A2-B1} & \textbf{Study Time:} 2-3 hours & \textbf{Difficulty:} ⭐⭐⭐☆☆
\end{tabular}
\end{center}

\section{Lesson Objectives}
In this chapter, you will learn:
\begin{itemize}
    \item How to use modal verbs to express ability, possibility, and permission
    \item The difference between can, could, may, and might
    \item How to make polite requests
\end{itemize}

\section{Reading Context}
\begin{readingbox}[title=Dialogue: Planning a Party]
\textbf{Alice:} \textbf{Can} you help me organize the office party?\\
\textbf{Bob:} Sure, I \textbf{can} help. When is it?\\
\textbf{Alice:} It \textbf{might} be next Friday, but we haven't decided yet.\\
\textbf{Bob:} We \textbf{could} have it at the new Italian restaurant.\\
\textbf{Alice:} That's a good idea. \textbf{May} I ask you to call them for a reservation?\\
\textbf{Bob:} Of course. \textbf{Could} you send me the number?\\
\textbf{Alice:} Yes. Oh, and it \textbf{may} rain, so we should check if they have indoor seating.
\end{readingbox}

\begin{britishbox}
\textbf{British Politeness with Modals:} British English uses modal verbs for \textbf{extreme politeness}. Instead of \textit{"I want..."}, say \textit{"I'd like..."} or \textit{"I wonder if I could..."} Instead of \textit{"Can you...?"}, use \textit{"Could you possibly...?"} or \textit{"Would you mind...?"} British speakers often use \textbf{might} and \textbf{may} for tentative suggestions: \textit{"It might be worth considering..."} Using \textbf{could} instead of \textbf{can} shows greater respect: \textit{"Could I have a word?"} (more polite than \textit{"Can I...?"}).
\end{britishbox}

\begin{comparisonbox}[title=🇪🇸 "Poder" en Español vs CAN/COULD/MAY en Inglés]
\textbf{⚠️ En español usamos "poder" para todo, pero en inglés hay diferencias:}

\textbf{PODER = habilidad → CAN}
\begin{itemize}
    \item "Puedo hablar inglés" = I \textbf{can} speak English
\end{itemize}

\textbf{PODER = permiso (informal) → CAN}
\begin{itemize}
    \item "¿Puedo ir?" = \textbf{Can} I go?
\end{itemize}

\textbf{PODRÍA = educado → COULD / MAY}
\begin{itemize}
    \item "¿Podría ayudarme?" = \textbf{Could} you help me? (muy británico)
    \item "¿Podría entrar?" = \textbf{May} I come in? (formal)
\end{itemize}

\textbf{PODRÍA = posibilidad → MIGHT / COULD}
\begin{itemize}
    \item "Podría llover" = It \textbf{might} rain / It \textbf{could} rain
\end{itemize}

\textbf{💡 Truco:} En británico, \textbf{cuanto más educado, más indirecto}: Can < Could < May < Might I...?
\end{comparisonbox}

\section{Grammar Focus: Modal Verbs Overview}

Modal verbs are special auxiliary verbs that express ability, possibility, permission, obligation, or advice.

\begin{grammarbox}[title=General Rules]
\begin{itemize}
    \item They do not change form (no -s for third person).
    \item They are followed by the base form of the verb (infinitive without "to").
    \item They do not need auxiliary verbs for questions or negatives.
\end{itemize}
\end{grammarbox}

\section{Key Concepts: Usage Guide}

\begin{vocabbox}[title=Can vs. Could]
\textbf{CAN}
\begin{itemize}
    \item \textbf{Ability (Present):} I \textbf{can} speak English. \trans{Puedo hablar inglés}
    \item \textbf{Permission (Informal):} \textbf{Can} I use your phone? \trans{¿Puedo usar tu teléfono?}
    \item \textbf{Possibility (General):} It \textbf{can} be cold here. \trans{Puede hacer frío aquí}
\end{itemize}

\textbf{COULD}
\begin{itemize}
    \item \textbf{Ability (Past):} I \textbf{could} swim when I was five. \trans{Podía nadar...}
    \item \textbf{Permission (Polite):} \textbf{Could} I leave early? \trans{¿Podría salir temprano?}
    \item \textbf{Possibility (Uncertain):} It \textbf{could} rain. \trans{Podría llover}
\end{itemize}
\end{vocabbox}

\begin{vocabbox}[title=May vs. Might]
\textbf{MAY}
\begin{itemize}
    \item \textbf{Permission (Formal):} \textbf{May} I come in? \trans{¿Puedo entrar? (formal)}
    \item \textbf{Possibility (Likely):} It \textbf{may} rain later. \trans{Puede que llueva}
\end{itemize}

\textbf{MIGHT}
\begin{itemize}
    \item \textbf{Possibility (Less Likely):} I \textbf{might} go to the party. \trans{Podría ir (quizás)}
\end{itemize}
\end{vocabbox}

\section{Politeness Scale}

From least to most polite:
\begin{center}
\Large \textbf{Can} $<$ \textbf{Could} $<$ \textbf{May}
\end{center}

\section{Practice Exercises}

\subsection{Exercise 1: Choose the Correct Modal}

\begin{enumerate}
    \item \underline{\hspace{2cm}} I borrow your pen? (informal request)
    \item She \underline{\hspace{2cm}} speak French when she was a child. (past ability)
    \item It \underline{\hspace{2cm}} snow tonight, but I'm not sure. (uncertain possibility)
    \item \underline{\hspace{2cm}} I leave the room, sir? (formal permission)
\end{enumerate}

\subsection{Exercise 2: Rewrite More Politely}
Transform these sentences to be more polite:

\begin{enumerate}
    \item Can you help me? $\rightarrow$ \underline{\hspace{5cm}}
    \item Can I sit here? $\rightarrow$ \underline{\hspace{5cm}}
    \item Can you explain again? $\rightarrow$ \underline{\hspace{5cm}}
\end{enumerate}

\subsection{Exercise 3: Talk About Abilities}
Write 5 sentences about what you can do now versus what you could do as a child.

\begin{tcolorbox}[colback=white,height=5cm]
% Example: I can drive a car now, but I couldn't when I was 15.
\end{tcolorbox}

\section{Key Takeaways}
\begin{itemize}
    \item \textbf{Can}: present ability, informal permission.
    \item \textbf{Could}: past ability, polite requests.
    \item \textbf{May}: formal permission, likely possibility.
    \item \textbf{Might}: less likely possibility.
    \item Always use the base form of the verb after a modal.
\end{itemize}

\section{Online Practice}
\begin{tcolorbox}[colback=blue!5,colframe=blue!40!black,title=Resources to Practice]
Online PracticeReinforce what you have learned with these interactive exercises:
\begin{itemize}
    \item \textbf{Modal Verbs Exercises:} \url{https://www.perfect-english-grammar.com/modal-verbs-exercises.html}
    \item \textbf{Can/Could/May/Might:} \url{https://test-english.com/grammar-points/a2/can-could-may-might/}
    \item \textbf{Interactive Activities:} \url{https://learnenglish.britishcouncil.org/grammar/english-grammar-reference/modal-verbs}
    \item \textbf{Permission \& Ability Quiz:} \url{https://www.englishpage.com/modals/modalintro.html}
\end{itemize}
\end{tcolorbox}




