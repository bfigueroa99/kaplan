\chapter{Fast Notes to add content after in the book}


\subsubsection{Writting}

\begin{itemize}
    \item Essay title:
    \item Do the disadvantages of ecotourism outweigh the advantages?
    \item Use for and against structure
\end{itemize}


\textbf{Ecotourism}

\textbf{Introduction:} In recent years, ecotourism has become populity, and sparked a debate  advantages and disadvantages. In this article we will talk about the different argument.

\textbf{Argument:} One of the advantages is increase the of mass tourism that leads to more profit, and with this, you can improve the facilities and security to local peoples and the enviroment. In addition, is more easy to promote the conservation of natural areas when more people show interest in visit them.

\textbf{Counterargument:} On the other hand mass tourism leads to destruccion of the enviroment and disrupt animal behavior amung other negative things. Plus, is hard to control the number of tourists that visit a natural area.

\textbf{Conclusion:} In summary, there are more benefits than the negative things. Such as the promoting conservation and suporting local economies. However, it is important to manage ecotourism properly to minimize its impact on the environment.




------------------------------------------------------------------------------------------------------------------------------------------------------------------------------------


\section{Phrasal Verbs}

A phrasal verb is a combination of a verb and one or more particles (prepositions or adverbs) that together create a meaning different from the original verb. Phrasal verbs are commonly used in everyday English and can be separable or inseparable.


\subsection{Common Phrasal Verbs}
\begin{itemize}
    \item \textbf{Look up:} To search for information (e.g., in a dictionary or online).
    \item \textbf{Give up:} To stop trying or to quit.
    \item \textbf{Take off:} To remove clothing or to leave the ground (for airplanes).
    \item \textbf{Run out of:} To have no more of something left.
    \item \textbf{Break down:} To stop functioning (for machines) or to become emotionally upset.
    \item \textbf{Put off:} To postpone or delay.
    \item \textbf{Get along with:} To have a good relationship with someone.
    \item \textbf{Turn on/off:} To activate or deactivate a device.
\end{itemize}

\subsection{When to Use Rephrasal?}

Rephrasal is the act of expressing the same idea using different words or phrases. It is often used to avoid repetition, clarify meaning, or simplify complex ideas.

\begin{itemize}
    \item Explain with others word for someone understand better
    \item Avoid repeating the same Words
    \item Make your writing more interesting
    \item Simplification of complex ideas
    \item Clarify meaning
\end{itemize}

\subsection{When to Use Paraphrasal?}

Paraphrasal is the act of rephrasing or restating a sentence or passage using different words while maintaining the original meaning. It is often used in academic writing, summarizing, and note-taking.

\begin{itemize}
    \item To summarize information
    \item To avoid plagiarism
    \item To clarify complex ideas
    \item To understand better
    \item Literature review
\end{itemize}



------------------------------------------------------------------------------------------------------------------------------------------------------------------------------------

Whereas: indica contraste entre dos ideas o situaciones, similar a "while" o "although". Se utiliza para mostrar que una cosa es diferente o contraria a otra.

