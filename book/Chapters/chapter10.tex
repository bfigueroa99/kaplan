\chapter{Future Simple: Will and Going To}

\section{Lesson Objectives}
\begin{tcolorbox}[colback=green!5, colframe=green!40!black, title={📚 By the end of this chapter, you will be able to:}]
\begin{itemize}
    \item [\checkmark] Form and use Future Simple with "will" for spontaneous decisions and predictions
    \item [\checkmark] Form and use "going to" for plans and intentions
    \item [\checkmark] Distinguish between "will" and "going to" correctly
    \item [\checkmark] Use Present Continuous for fixed future arrangements
    \item [\checkmark] Apply future forms in real communication situations
\end{itemize}
\end{tcolorbox}

\textbf{CEFR Level:} A2-B1

\section{Reading Context}
\begin{readingbox}[title=Dialogue: Making Plans]
\textbf{Sarah:} What are you doing this weekend?\\
\textbf{Tom:} I\textbf{'m meeting} my friend on Saturday morning. We\textbf{'re having} lunch at that new Italian restaurant.\\
\textbf{Sarah:} That sounds nice! What about Sunday?\\
\textbf{Tom:} I\textbf{'m going to visit} my parents. I promised them last week.\\
\textbf{Sarah:} Oh, the weather forecast says it\textbf{'s going to rain}.\\
\textbf{Tom:} Really? Then I\textbf{'ll take} an umbrella. Thanks for telling me!\\
\textbf{Sarah:} No problem. I think I\textbf{'ll stay} home and watch TV if it rains.
\end{readingbox}

\begin{britishbox}
\textbf{British Future Plans and Arrangements:} British people often talk about future plans using these patterns: \textit{"I'm going to..."} (for definite plans), \textit{"I'm thinking of..."} (considering), \textit{"I might..."} (tentative), \textit{"I'll probably..."} (likely but not certain). When making arrangements, British speakers say: \textit{"Shall we meet at...?"} (more common than \textit{"Should we...?"}). Weather discussions are crucial: \textit{"It's going to rain tomorrow"}, \textit{"The forecast says it'll be sunny"}. British people often say \textit{"I'll have a cuppa"} (cup of tea)—a very British future plan!
\end{britishbox}

\begin{comparisonbox}[title=🇪🇸 Futuro en Español vs WILL/GOING TO en Inglés]
\textbf{⚠️ En español tenemos UN tiempo futuro, en inglés VARIOS:}

\textbf{1. Futuro español "Haré" puede ser:}
\begin{itemize}
    \item WILL (decisión espontánea): "El teléfono suena. \textit{Contestaré}" = "I\textbf{'ll answer}"
    \item GOING TO (plan previo): "Mañana \textit{visitaré} a mi madre" = "I\textbf{'m going to visit}"
    \item Present Continuous (cita fija): "Mañana \textit{veré} al doctor" = "I\textbf{'m seeing}"
\end{itemize}

\textbf{2. "Ir a + infinitivo" = GOING TO:}
\begin{itemize}
    \item Español: "Voy a estudiar" = I\textbf{'m going to study}
    \item Español: "Va a llover" (hay nubes) = It\textbf{'s going to rain}
\end{itemize}

\textbf{3. WILL NO se usa para planes previos:}
\begin{itemize}
    \item \textcolor{red}{❌} "Tomorrow I\textbf{'ll visit} my mother" (si ya lo planeaste)
    \item \textcolor{green}{✓} "Tomorrow I\textbf{'m going to visit} my mother"
    \item \textcolor{green}{✓} "The phone's ringing. I\textbf{'ll answer} it" (decisión ahora)
\end{itemize}
\end{comparisonbox}

\section{Grammar Focus: Future Simple with WILL}

The future simple with ``will'' expresses decisions made at the moment of speaking, general predictions, promises, offers, and future facts.

\begin{grammarbox}[title=Future Simple (Will) - Structure]
\textbf{Structure:}
\begin{itemize}
    \item \textbf{Affirmative:} Subject + \textbf{will} + base verb
    \item \textbf{Negative:} Subject + \textbf{will not (won't)} + base verb
    \item \textbf{Interrogative:} \textbf{Will} + subject + base verb?
\end{itemize}

\textbf{Contractions:}
\begin{itemize}
    \item I will = I'll \quad\quad She will = She'll
    \item Will not = Won't
\end{itemize}

\textbf{Examples:}
\begin{itemize}
    \item I\textbf{'ll call} you later. \trans{Te llamaré después} (decision now)
    \item It \textbf{will rain} tomorrow. \trans{Lloverá mañana} (prediction)
    \item We \textbf{won't be} late. \trans{No llegaremos tarde} (negative)
    \item \textbf{Will} you \textbf{help} me? \trans{¿Me ayudarás?} (request)
\end{itemize}
\end{grammarbox}

\subsection{When to Use WILL}

\begin{vocabbox}[title=Use WILL for:]
\textbf{1. Spontaneous decisions (made now):}
\begin{itemize}
    \item The phone's ringing. I\textbf{'ll answer} it. \trans{Yo contesto}
    \item I'm tired. I think I\textbf{'ll go} to bed. \trans{Creo que iré a la cama}
\end{itemize}

\textbf{2. Predictions (based on opinion, not evidence):}
\begin{itemize}
    \item I think it \textbf{will be} sunny tomorrow. \trans{Creo que hará sol}
    \item She \textbf{will probably arrive} late. \trans{Probablemente llegará tarde}
\end{itemize}

\textbf{3. Promises and offers:}
\begin{itemize}
    \item I\textbf{'ll help} you with your homework. \trans{Te ayudaré}
    \item Don't worry, I\textbf{'ll be} there. \trans{No te preocupes, estaré allí}
\end{itemize}

\textbf{4. Threats and warnings:}
\begin{itemize}
    \item If you don't study, you\textbf{'ll fail}. \trans{Si no estudias, reprobarás}
\end{itemize}

\textbf{5. Future facts:}
\begin{itemize}
    \item I \textbf{will be} 30 next year. \trans{Tendré 30 el próximo año}
\end{itemize}
\end{vocabbox}

\section{Grammar Focus: GOING TO}

The future with ``going to'' is used for planned actions or intentions and predictions based on present evidence.

\begin{grammarbox}[title=Going To - Structure]
\textbf{Structure:}
\begin{itemize}
    \item \textbf{Affirmative:} Subject + \textbf{am/is/are} + \textbf{going to} + base verb
    \item \textbf{Negative:} Subject + \textbf{am/is/are not} + \textbf{going to} + base verb
    \item \textbf{Interrogative:} \textbf{Am/Is/Are} + subject + \textbf{going to} + base verb?
\end{itemize}

\textbf{Examples:}
\begin{itemize}
    \item I \textbf{am going to start} a new job next month. \trans{Voy a empezar}
    \item She \textbf{is not going to attend} the meeting. \trans{Ella no va a asistir}
    \item \textbf{Are} they \textbf{going to move} to a new house? \trans{¿Van a mudarse?}
\end{itemize}
\end{grammarbox}

\subsection{When to Use GOING TO}

\begin{vocabbox}[title=Use GOING TO for:]
\textbf{1. Plans and intentions (decided before):}
\begin{itemize}
    \item I\textbf{'m going to visit} my grandparents next weekend. \trans{Voy a visitar} (already planned)
    \item She\textbf{'s going to study} medicine. \trans{Va a estudiar} (her plan)
\end{itemize}

\textbf{2. Predictions based on present evidence:}
\begin{itemize}
    \item Look at those dark clouds! It\textbf{'s going to rain}. \trans{Va a llover} (I can see the clouds)
    \item Be careful! You\textbf{'re going to fall}! \trans{Te vas a caer} (I see you're unsteady)
\end{itemize}
\end{vocabbox}

\section{WILL vs. GOING TO: The Key Difference}

\begin{table}[h]
\centering
\begin{tabular}{|p{5.5cm}|p{5.5cm}|}
\hline
\textbf{WILL} & \textbf{GOING TO} \\
\hline
\multicolumn{2}{|c|}{\textbf{Key Difference}} \\
\hline
Decision made NOW & Decision made BEFORE now \\
\hline
No evidence, opinion & Evidence visible now \\
\hline
\textbf{Examples:} & \textbf{Examples:} \\
\hline
A: The phone's ringing. & I can't go out tonight. \\
B: I\textbf{'ll answer} it. (decides now) & I\textbf{'m going to study}. (planned) \\
\hline
I think it \textbf{will} be nice. (opinion) & Look at the sky! It\textbf{'s going to} snow! (evidence) \\
\hline
\end{tabular}
\caption{Will vs. Going To comparison}
\end{table}

\begin{examplebox}[title=More Examples - WILL vs GOING TO]
\textbf{Situation 1: Someone asks what you'll do tonight}
\begin{itemize}
    \item If you decide NOW: ``I think I\textbf{'ll watch} TV.'' (WILL - spontaneous)
    \item If you already planned: ``I\textbf{'m going to watch} the new series.'' (GOING TO - planned)
\end{itemize}

\textbf{Situation 2: Looking at the weather}
\begin{itemize}
    \item No evidence: ``I think it \textbf{will rain} tomorrow.'' (WILL - opinion)
    \item See dark clouds: ``Look! It\textbf{'s going to rain}!'' (GOING TO - evidence)
\end{itemize}
\end{examplebox}

\section{Present Continuous for Future Arrangements}

The present continuous is often used to talk about \textbf{fixed plans and arrangements}, usually when a time is specified.

\begin{grammarbox}[title=Present Continuous for Future]
\textbf{Structure:} Subject + \textbf{am/is/are} + \textbf{verb-ing} + time reference

\textbf{Use:} For arrangements with specific times, places, or people

\textbf{Examples:}
\begin{itemize}
    \item I\textbf{'m meeting} the manager at 3 p.m. \trans{Me reúno con} (arranged)
    \item She\textbf{'s flying} to Madrid next Monday. \trans{Vuela a} (booked)
    \item \textbf{Are} you \textbf{having} dinner with them tonight? \trans{¿Cenas con ellos?} (arranged)
\end{itemize}

\textbf{Common with:} meet, see, have (meals), fly, leave, arrive
\end{grammarbox}

\section{Comparing All Three Forms}

\begin{table}[h]
\centering
\begin{tabular}{|l|p{4cm}|p{4cm}|}
\hline
\textbf{Form} & \textbf{When to Use} & \textbf{Example} \\
\hline
\textbf{WILL} & Spontaneous decisions, predictions (opinion) & I'll call you. \\
\hline
\textbf{GOING TO} & Plans (decided before), predictions (evidence) & I'm going to visit Paris. \\
\hline
\textbf{Present Continuous} & Fixed arrangements (time/place set) & I'm meeting Tom at 5. \\
\hline
\end{tabular}
\caption{Three ways to talk about the future}
\end{table}

\section{Time Expressions}

\begin{vocabbox}[title=Common Future Time Expressions]
\begin{itemize}
    \item \textbf{tomorrow} \trans{mañana}
    \item \textbf{next} week/month/year/Monday \trans{próximo/próxima}
    \item \textbf{in} two days/a week \trans{en dos días/una semana}
    \item \textbf{tonight} \trans{esta noche}
    \item \textbf{this} evening/weekend \trans{esta tarde/este fin de semana}
    \item \textbf{soon} \trans{pronto}
    \item \textbf{later} \trans{más tarde}
    \item \textbf{in the future} \trans{en el futuro}
\end{itemize}
\end{vocabbox}

\section{Common Mistakes to Avoid}

\begin{tcolorbox}[colback=red!5, colframe=red!60!black, title={Typical Errors}]
\begin{tabular}{|p{5cm}|p{5cm}|}
\hline
\textbf{❌ Incorrect} & \textbf{✓ Correct} \\
\hline
I \textcolor{red}{will going to} study. & I\textbf{'m going to} study. \\
\hline
I \textcolor{red}{go} tomorrow. & I\textbf{'ll go / 'm going} tomorrow. \\
\hline
I \textcolor{red}{will to} help you. & I\textbf{'ll help} you. \\
\hline
Look! It \textcolor{red}{will} rain! & It\textbf{'s going to} rain! \\
\hline
I'm going \textcolor{red}{visit} London. & I'm going \textbf{to visit} London. \\
\hline
\textcolor{red}{Will} you going? & \textbf{Are} you going? \\
\hline
\end{tabular}
\end{tcolorbox}

\section{Practice Exercises}

\subsection{Exercise 1: Choose WILL or GOING TO}
Complete with the correct form:

\begin{enumerate}
    \item A: Why are you turning on the TV?
    
    B: I \underline{\hspace{3cm}} (watch) the news.
    
    \item It's very cold. I think it \underline{\hspace{3cm}} (snow).
    
    \item A: I don't have any money.
    
    B: Don't worry. I \underline{\hspace{3cm}} (lend) you some.
    
    \item Look at those dark clouds! It \underline{\hspace{3cm}} (rain).
    
    \item I \underline{\hspace{3cm}} (visit) my grandparents next weekend. (I already told them)
    
    \item A: The phone's ringing.
    
    B: I \underline{\hspace{3cm}} (answer) it.
\end{enumerate}

\subsection{Exercise 2: Correct the Mistakes}
Find and correct the errors:

\begin{enumerate}
    \item I will going to the party tomorrow. \underline{\hspace{5cm}}
    \item She is going visit London. \underline{\hspace{5cm}}
    \item Are you will help me? \underline{\hspace{5cm}}
    \item I think it will to rain. \underline{\hspace{5cm}}
    \item Look! He will fall! \underline{\hspace{5cm}}
\end{enumerate}

\subsection{Exercise 3: Complete the Dialogues}
Use WILL, GOING TO, or Present Continuous:

\begin{enumerate}
    \item A: What \underline{\hspace{2cm}} you \underline{\hspace{2cm}} (do) tonight?
    
    B: I \underline{\hspace{3cm}} (see) a movie. I bought the tickets yesterday.
    
    \item A: Oh no! I forgot my wallet!
    
    B: Don't worry. I \underline{\hspace{3cm}} (pay) for lunch.
    
    \item A: Why are you carrying a suitcase?
    
    B: I \underline{\hspace{3cm}} (fly) to Rome tomorrow morning.
\end{enumerate}

\subsection{Exercise 4: Real Communication}
Answer these questions about YOUR plans:

\begin{enumerate}
    \item What are you doing this weekend?
    
    \underline{\hspace{10cm}}
    
    \item What will you do if it rains tomorrow?
    
    \underline{\hspace{10cm}}
    
    \item What are you going to do next year?
    
    \underline{\hspace{10cm}}
\end{enumerate}

\section{Quick Reference Card}

\begin{tcolorbox}[colback=blue!5, colframe=blue!60!black, title={Chapter Summary}]
\textbf{Key Points to Remember:}
\begin{itemize}
    \item \textbf{WILL} = spontaneous decisions, predictions (opinion), promises
    \item \textbf{GOING TO} = plans (decided before), predictions (evidence)
    \item \textbf{Present Continuous} = fixed arrangements (time/place confirmed)
    \item WILL + base verb (NOT: will to)
    \item AM/IS/ARE + going TO + base verb
    \item Use GOING TO when you can SEE evidence
    \item Use WILL when you decide NOW
\end{itemize}
\end{tcolorbox}

\section{Key Takeaways}
\begin{itemize}
    \item \textbf{Will}: Spontaneous decisions and general predictions
    \item \textbf{Going to}: Plans and predictions based on evidence
    \item \textbf{Present Continuous}: Fixed arrangements with specific times
    \item The choice depends on WHEN you made the decision
\end{itemize}

\section{🇬🇧 British English Notes}

\begin{tcolorbox}[colback=kaplanLight, colframe=kaplanPurple, title={British Culture \& Usage}]
\textbf{Shall vs Will:}

In British English, ``shall'' is sometimes used with ``I'' and ``we'' for offers and suggestions:
\begin{itemize}
    \item \textbf{Shall} I help you? (offer)
    \item \textbf{Shall} we go? (suggestion)
\end{itemize}

This is more formal and traditional. Americans typically use ``will'' or ``should.''
\end{tcolorbox}

\section{Online Practice}
\begin{tcolorbox}[colback=blue!5,colframe=blue!40!black,title=📱 Resources to Practice]
Reinforce what you have learned with these interactive exercises:
\begin{itemize}
    \item \textbf{Will vs Going To:} \url{https://test-english.com/grammar-points/a2/will-be-going-to/}
    \item \textbf{Future Simple Practice:} \url{https://www.perfect-english-grammar.com/will-exercise-1.html}
    \item \textbf{Going To Exercises:} \url{https://www.perfect-english-grammar.com/going-to-exercise-1.html}
    \item \textbf{British Council:} \url{https://learnenglish.britishcouncil.org/grammar/intermediate-to-upper-intermediate/talking-about-the-future}
\end{itemize}
\end{tcolorbox}
