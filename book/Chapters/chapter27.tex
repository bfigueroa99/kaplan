\section{Content}

This chapter is under development.

Write a short reflective paragraph, thinking about your team's performance in both conversations today.

Our team performed well during both conversations: every member participated actively, contributed thoughtful ideas, and engaged in constructive (and sometimes playful) debate to defend their views. We listened respectfully, built on one another's points, and left with clear next steps to follow up on our decisions.




\section{Differences Between Remains, Remnants, and Leftovers}

The distinction between \emph{remains}, \emph{remnants}, and \emph{leftovers} lies in their scope and context. \textit{Remains} refer to what persists after something has been consumed or destroyed, typically denoting larger or more substantial portions. \textit{Remnants} are smaller fragments or pieces left from a larger whole, often suggesting limited utility or value. \textit{Leftovers}, in contrast, specifically designate uneaten food remaining after a meal, preserved for future consumption. In essence, remains encompass any residual parts, remnants are minimal fragments, and leftovers are reserved foodstuffs.

The distinction between \emph{remains}, \emph{remnants}, and \emph{leftovers} depends on scale and context. \textit{Remains} denotes what persists after something has been consumed or destroyed and often implies larger or more substantial portions. \textit{Remnants} refers to small fragments left from a larger whole, typically with limited use. \textit{Leftovers} specifically describe uneaten food saved after a meal for later consumption.



\section{New Year Resolutions}

\begin{itemize}
    \item Exercise regularly to improve physical health.
    \item Eat a balanced diet and cut down on junk food.
    \item Read more to expand knowledge and stimulate creativity.
    \item Practice mindfulness or meditation for mental well-being.
    \item Learn a new skill or hobby to broaden interests.
    \item Save money and follow a simple budget.
    \item Spend more quality time with family and friends.
    \item Volunteer in the community to give back.
    \item Reduce screen time and enjoy outdoor activities.
    \item Set achievable goals and track progress during the year.
\end{itemize}



\subsection{Phrasal Verbs for New Year Resolutions}

\begin{itemize}
    \item Cut down on junk food.
    \item Work out regularly.
    \item Pick up a new hobby.
    \item Save up for a goal or purchase.
    \item Spend more time with loved ones.
    \item Give up unhealthy habits.
    \item Take up meditation or mindfulness.
    \item Plan out the year ahead.
    \item Follow through on your goals.
    \item Slow down and rest when needed.
\end{itemize}

\bigskip
\noindent\textbf{Definitions}
\begin{description}
    \item[Cut down on:] to reduce the amount of something (e.g., junk food).
    \item[Work out:] to exercise or train physically.
    \item[Pick up:] to start learning or doing something new (e.g., a hobby).
    \item[Save up:] to accumulate money for a specific purpose.
    \item[Spend more time:] to allocate time for a particular activity or person.
    \item[Give up:] to stop doing something (e.g., a bad habit).
    \item[Take up:] to begin a new activity or hobby (e.g., meditation).
    \item[Plan out:] to organize or arrange something in detail (e.g., the year ahead).
    \item[Follow through:] to complete or carry out a plan or goal.
    \item[Slow down:] to reduce speed or take things more leisurely.
\end{description}

---------------------------------------------------------------------------------------------------------------------

A -> a

B -> b

C -> a

D -> a


---------------------------------------------------------------------------------------------------------------------


keep 

set

steps

off

upset


---------------------------------------------------------------------------------------------------------------------


A -> 8

B -> 7

C -> 4

D -> 2

E -> 6

F -> 5

G -> 1

H -> 3


carry out -> complete something

cut down on -> reduce

get round to -> find time to do something

pay off -> bring good results

map out -> plan in detail

take up -> start something

stick to -> continue with something

follow through on -> finish swhat you started




---------------------------------------------------------------------------------------------------------------------


1 lie and 2 truths

1 resolution = a lie

2 resolutions = truths


The resolutions have to be specific, Each resolution has to use new phrasal verbs


1. I am going to cut down on junk food this year, I am going to stop buy its.

2. I am going to take up cooking this year, I am going to pick up new recipes every week.

3. I get round to play guitar this year, I am going to take up guitar lessons every weekend. 

---------------------------------------------------------------------------------------------------------------------





---------------------------------------------------------------------------------------------------------------------
