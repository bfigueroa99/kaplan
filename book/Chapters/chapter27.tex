\chapter{Narrative Techniques and Creative Writing}

\begin{center}
\begin{tabular}{ccc}
\cefrlevel{B2-C1} & \textbf{Study Time:} 3-4 hours & \textbf{Difficulty:} ⭐⭐⭐⭐☆
\end{tabular}
\end{center}

\section{Lesson Objectives}
In this chapter, you will learn:
\begin{itemize}
    \item How to structure a narrative or story
    \item Techniques for creating engaging stories (sequencing, dialogue, description)
    \item How to develop characters and create vivid descriptions
    \item How to use appropriate tenses for storytelling
    \item How to write reflectively and analyze experiences
    \item Phrasal verbs that are useful in narratives and descriptions
\end{itemize}

\section{Reading Context: Example Story}
\begin{readingbox}[title=Sample Narrative: A Day to Remember]
Last summer, I had an unforgettable experience. I was walking through the city center when I suddenly noticed something unusual—a street musician was playing the most beautiful violin melody I had ever heard. At first, I walked past \textbf{hurriedly}, but the music had captivated me. I stopped and \textbf{turned around} to listen more carefully. 

The musician's eyes were closed, completely lost in the performance. I stood there for several minutes, mesmerized. Other people gradually \textbf{gathered around}, drawn by the same enchanting sound. None of us \textbf{wanted to miss} this extraordinary moment. When the piece finished, we all \textbf{burst out} in spontaneous applause.

From that day on, I realized how important it is to \textbf{slow down} and appreciate beauty in everyday life. I have since \textbf{taken up} the violin myself, hoping to one day create such magical moments for others.
\end{readingbox}

\section{Grammar Focus: Narrative Structure and Tenses}

\begin{spanishbox}[title=⚠️ Para Hispanohablantes: Narrative Techniques]
\textbf{Diferencia clave:} El inglés tiene técnicas narrativas más estructuradas que el español.

\textbf{❌ Error común:}
\begin{itemize}
    \item Español: Uso flexible de tiempos narrativos
    \item ❌ "I walked when suddenly I see a dog" (mezclar tiempos)
\end{itemize}

\textbf{✓ En inglés:}
\begin{itemize}
    \item Past Continuous = escenario (lo que estaba pasando)
    \item Past Simple = acción principal (lo que pasó)
    \item Past Perfect = acción anterior (lo que había pasado)
    \item ✓ "I was walking when I suddenly saw a dog"
\end{itemize}

\textbf{💡 Truco:} Inglés narrativo = como una película con múltiples capas de tiempo.
\end{spanishbox}

\begin{grammarbox}[title=Setting the Scene with Past Continuous]
Use Past Continuous to set the background or context of your story:

\textbf{Structure:} Subject + was/were + -ing

\textbf{Examples:}
\begin{itemize}
    \item I \textbf{was walking} home when it started to rain.
    \item The sun \textbf{was shining} as we arrived at the beach.
    \item She \textbf{was preparing} dinner when he called.
\end{itemize}

\textbf{Function:} Past Continuous provides the "scene" or background action.
\end{grammarbox}

\begin{grammarbox}[title=The Action with Past Simple]
Use Past Simple for the main events or actions in your story:

\textbf{Structure:} Subject + past verb

\textbf{Examples:}
\begin{itemize}
    \item Suddenly, it \textbf{started} to rain heavily.
    \item I \textbf{spotted} an old friend in the crowd.
    \item We \textbf{decided} to go for a walk.
\end{itemize}

\textbf{Function:} Past Simple tells what happened, moving the story forward.
\end{grammarbox}

\begin{grammarbox}[title=Highlighting the Climax with Past Perfect]
Use Past Perfect when one action happened before another in the past:

\textbf{Structure:} Subject + had + past participle

\textbf{Examples:}
\begin{itemize}
    \item By the time I arrived, everyone \textbf{had left}.
    \item I \textbf{had never seen} anything so beautiful before.
    \item She \textbf{had finished} her work before the deadline.
\end{itemize}

\textbf{Function:} Past Perfect shows the sequence of events clearly.
\end{grammarbox}

\section{Narrative Techniques}

\begin{vocabbox}[title=Key Storytelling Techniques]

\textbf{1. Sequencing:} Use time expressions to organize events
\begin{itemize}
    \item First... Then... Next... After that... Finally...
    \item Meanwhile... At the same time... Suddenly... Eventually...
\end{itemize}

\textbf{2. Dialogue:} Include direct speech to bring characters to life
\begin{itemize}
    \item "What are you doing?" she asked.
    \item He replied, "I'm trying to fix this broken vase."
\end{itemize}

\textbf{3. Vivid Description:} Use adjectives and adverbs to create imagery
\begin{itemize}
    \item The old house stood silently at the end of the winding road.
    \item Golden sunlight streamed through the ancient windows.
\end{itemize}

\textbf{4. Sensory Details:} Appeal to the senses (sight, sound, smell, taste, touch)
\begin{itemize}
    \item I could hear the distant sound of waves crashing on the shore.
    \item The air smelled fresh and crisp after the rain.
\end{itemize}

\textbf{5. Character Development:} Show growth or change in characters
\begin{itemize}
    \item At first, he was shy and reserved.
    \item By the end of the journey, he had become confident and brave.
\end{itemize}

\end{vocabbox}

\section{Useful Phrasal Verbs for Storytelling}

\begin{table}[h]
\centering
\begin{tabular}{|l|p{6cm}|l|}
\hline
\textbf{Phrasal Verb} & \textbf{Meaning} & \textbf{Example} \\
\hline
Look back (on) & Remember and think about & He looked back on his childhood. \\
\hline
Turn around & Change direction or reverse & I turned around to see who was calling. \\
\hline
Slow down & Reduce speed; relax & Life was too fast; I needed to slow down. \\
\hline
Speed up & Increase pace & The story speeds up after chapter 3. \\
\hline
Come across & Find or meet unexpectedly & I came across an old photo yesterday. \\
\hline
Bring back & Return something; recall & This song brings back memories. \\
\hline
Take up & Start a hobby or activity & I took up painting last year. \\
\hline
Give up & Stop trying; abandon & Don't give up on your dreams. \\
\hline
Gather around & Come together in a group & People gathered around the fire. \\
\hline
Burst out & Suddenly express or do & We burst out laughing. \\
\hline
Carry on & Continue & The story carries on in the next chapter. \\
\hline
Catch up (with) & Move faster to reach & I ran to catch up with my friends. \\
\hline
\end{tabular}
\caption{Phrasal verbs useful for narratives}
\end{table}

\section{Practice Exercises}

\subsection{Exercise 1: Fill in the Gaps with Correct Tense}
Complete the story with the correct past tense (Past Simple, Past Continuous, or Past Perfect):

\begin{enumerate}
    \item When I \underline{\hspace{3cm}} (arrive) at the party, everyone \underline{\hspace{3cm}} (dance) already.
    \item I \underline{\hspace{3cm}} (see) my old friend at the supermarket yesterday. We \underline{\hspace{3cm}} (not see) each other for five years.
    \item While I \underline{\hspace{3cm}} (do) homework, my phone \underline{\hspace{3cm}} (ring).
\end{enumerate}

\subsection{Exercise 2: Use Phrasal Verbs in Context}
Rewrite the following sentences using the correct phrasal verb:

\begin{enumerate}
    \item I remember my childhood with happiness. (look back on)
    \item We suddenly started laughing. (burst out)
    \item I started learning guitar last month. (take up)
    \item She found an interesting book at the library. (come across)
\end{enumerate}

\subsection{Exercise 3: Write a Short Narrative}
Write a short story (150-200 words) about an interesting experience. Include:
\begin{itemize}
    \item A clear setting using Past Continuous
    \item Main events using Past Simple
    \item At least 3 phrasal verbs
    \item Vivid descriptions and sensory details
    \item A beginning, middle, and end
    \item Reflection on what you learned
\end{itemize}

\begin{tcolorbox}[colback=white,height=10cm]
\end{tcolorbox}

\subsection{Exercise 4: Dialogue in Narrative}
Create a short conversation (6-8 lines) between two characters that could be part of a story. Include:
\begin{itemize}
    \item Clear dialogue tags ("she said," "he asked," etc.)
    \item At least one phrasal verb in the dialogue
    \item Character emotions or reactions
\end{itemize}

\begin{tcolorbox}[colback=white,height=6cm]
\end{tcolorbox}

\section{Reflective Writing}

Reflective writing helps you analyze and think deeply about experiences. Use these structures:

\begin{grammarbox}[title=Reflective Sentence Starters]
\begin{itemize}
    \item I realized that... \trans{Me di cuenta de que...}
    \item This experience taught me... \trans{Esta experiencia me enseñó...}
    \item Looking back, I understand now that... \trans{Mirando hacia atrás, ahora entiendo que...}
    \item From this, I learned that... \trans{De esto, aprendí que...}
    \item What surprised me most was... \trans{Lo que más me sorprendió fue...}
    \item I had never considered that... \trans{Nunca había considerado que...}
    \item The most important lesson was... \trans{La lección más importante fue...}
\end{itemize}
\end{grammarbox}

\subsection{Exercise 5: Reflective Paragraph}
Write a reflective paragraph (100-150 words) about an important personal experience. Include:
\begin{itemize}
    \item What happened (the situation)
    \item How you felt at the time
    \item What you learned from it
    \item How it has changed your perspective
\end{itemize}

\begin{tcolorbox}[colback=white,height=7cm]
\end{tcolorbox}

\section{[INCORRECT] Common Mistakes in Narrative Writing}

\begin{tcolorbox}[colback=red!5, colframe=red!30, title=Watch Out!]
\textbf{Mistake 1:} Mixing tenses inconsistently
\begin{itemize}
    \item [INCORRECT] "I walked home and \textbf{see} my friend."
    \item [CORRECT] "I walked home and \textbf{saw} my friend."
\end{itemize}

\textbf{Mistake 2:} Not using Past Continuous for background events
\begin{itemize}
    \item [INCORRECT] "I worked when he arrived."
    \item [CORRECT] "I \textbf{was working} when he arrived."
\end{itemize}

\textbf{Mistake 3:} Missing punctuation in dialogue
\begin{itemize}
    \item [INCORRECT] 'What time is it' she asked
    \item [CORRECT] "What time is it?" she asked.
\end{itemize}

\textbf{Mistake 4:} Not using enough descriptive language
\begin{itemize}
    \item [INCORRECT] "It was a nice day."
    \item [CORRECT] "Golden sunlight filtered through the trees, warming my face."
\end{itemize}
\end{tcolorbox}

\section{🇬🇧 British English Notes}

In British narrative writing:
\begin{itemize}
    \item British writers often use more sophisticated vocabulary and longer sentences
    \item Dialogue punctuation: Use single quotes (\texttt{'}) instead of double quotes (\texttt{"})
    \begin{itemize}
        \item British: \texttt{'What are you doing?' he asked.}
        \item American: \texttt{"What are you doing?" he asked.}
    \end{itemize}
    \item British English prefers \textbf{travelled, grabbed, stopped} (double consonant before adding -ed)
    \item Common British storytelling phrases:
    \begin{itemize}
        \item "Once upon a time..." (more traditional)
        \item "Years ago..." (common opening)
        \item "As luck would have it..." (narrative device)
    \end{itemize}
\end{itemize}

\section{Key Takeaways}
\begin{itemize}
    \item Use \textbf{Past Continuous} for background/scene-setting
    \item Use \textbf{Past Simple} for main events and actions
    \item Use \textbf{Past Perfect} to show which action happened first
    \item Vivid descriptions, dialogue, and sensory details make stories engaging
    \item Phrasal verbs add natural flow to narrative writing
    \item Reflective writing helps you process and learn from experiences
    \item Consistent tense use is essential for clarity
\end{itemize}


keep 

set

steps

off

upset


---------------------------------------------------------------------------------------------------------------------


A -> 8

B -> 7

C -> 4

D -> 2

E -> 6

F -> 5

G -> 1

H -> 3


carry out -> complete something

cut down on -> reduce

get round to -> find time to do something

pay off -> bring good results

map out -> plan in detail

take up -> start something

stick to -> continue with something

follow through on -> finish swhat you started




---------------------------------------------------------------------------------------------------------------------


1 lie and 2 truths

1 resolution = a lie

2 resolutions = truths


The resolutions have to be specific, Each resolution has to use new phrasal verbs


1. I am going to cut down on junk food this year, I am going to stop buy its.

2. I am going to take up cooking this year, I am going to pick up new recipes every week.

3. I get round to play guitar this year, I am going to take up guitar lessons every weekend. 

---------------------------------------------------------------------------------------------------------------------


\subsection{Per se}
The Latin expression \textit{per se} means "by itself" or "in itself". It is used to indicate that something is considered in isolation, without external factors or additions.

It is used similarly to honestly and personally.


---------------------------------------------------------------------------------------------------------------------





\subsection{if-clauses}


---------------------------------------------------------------------------------------------------------------------


innate -adjetive


bias-noun











\section{Online Practice}
Here are some useful websites to practice this topic:
\begin{itemize}
    \item \href{https://learnenglish.britishcouncil.org/grammar/b1-b2-grammar/past-perfect}{British Council LearnEnglish}
    \item \href{https://test-english.com/grammar-points/b1/narrative-tenses/}{Test-English}
\end{itemize}
