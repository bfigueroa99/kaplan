% ============================================
% CUSTOM MACROS AND ENVIRONMENTS
% ============================================

% Emoji and Special Character Replacements
% Note: \checkmark is already defined by amssymb package
% With XeLaTeX, Unicode characters work natively - no need for DeclareUnicodeCharacter
% However, we need a font that supports emojis
% Using Symbola for better emoji coverage
\newfontfamily\emojifont{Symbola}

% Map specific emojis to the emoji font
% Use \mdseries (medium/normal) and \upshape (upright) to avoid looking for Bold/Italic emoji variants
\newunicodechar{⭐}{{\emojifont\mdseries\upshape\textcolor{orange}{☆}}}
\newunicodechar{☆}{{\emojifont\mdseries\upshape ☆}}
\newunicodechar{☕}{{\emojifont\mdseries\upshape ☕}}
\newunicodechar{💡}{{\emojifont\mdseries\upshape 💡}}
\newunicodechar{❌}{{\emojifont\mdseries\upshape ❌}}
\newunicodechar{✓}{{\emojifont\mdseries\upshape ✓}}
\newunicodechar{🔊}{{\emojifont\mdseries\upshape 🔊}}
\newunicodechar{⚠}{{\emojifont\mdseries\upshape ⚠}}
\newunicodechar{✅}{{\emojifont\mdseries\upshape ✅}}
\newunicodechar{🚫}{{\emojifont\mdseries\upshape 🚫}}
\newunicodechar{👇}{{\emojifont\mdseries\upshape 👇}}
\newunicodechar{👉}{{\emojifont\mdseries\upshape 👉}}
\newunicodechar{👈}{{\emojifont\mdseries\upshape 👈}}
\newunicodechar{👍}{{\emojifont\mdseries\upshape 👍}}
\newunicodechar{👎}{{\emojifont\mdseries\upshape 👎}}
\newunicodechar{🤔}{{\emojifont\mdseries\upshape 🤔}}
\newunicodechar{ˈ}{{\ipafont ˈ}}
\newunicodechar{🇬}{{\emojifont\mdseries\upshape 🇬}}
\newunicodechar{🇧}{{\emojifont\mdseries\upshape 🇧}}
\newunicodechar{🇪}{{\emojifont\mdseries\upshape 🇪}}
\newunicodechar{🇸}{{\emojifont\mdseries\upshape 🇸}}
\newunicodechar{≠}{\ensuremath{\neq}}



\newcommand{\crossmark}{\textbf{[INCORRECT]}}

% IPA Font
\newfontfamily\ipafont{DejaVu Sans}

% Common IPA symbols mapping
\newunicodechar{ɪ}{{\ipafont ɪ}}
\newunicodechar{ɒ}{{\ipafont ɒ}}
\newunicodechar{ʌ}{{\ipafont ʌ}}
\newunicodechar{ː}{{\ipafont ː}}
\newunicodechar{θ}{{\ipafont θ}}
\newunicodechar{ʃ}{{\ipafont ʃ}}
\newunicodechar{ð}{{\ipafont ð}}
\newunicodechar{ə}{{\ipafont ə}}
\newunicodechar{ŋ}{{\ipafont ŋ}}
\newunicodechar{ɜ}{{\ipafont ɜ}}
\newunicodechar{ɔ}{{\ipafont ɔ}}
\newunicodechar{ʊ}{{\ipafont ʊ}}
\newunicodechar{ɑ}{{\ipafont ɑ}}
\newunicodechar{æ}{{\ipafont æ}}
\newunicodechar{ɛ}{{\ipafont ɛ}}

% More Emojis
\newunicodechar{📚}{{\emojifont\mdseries\upshape 📚}}
\newunicodechar{📱}{{\emojifont\mdseries\upshape 📱}}
\newunicodechar{✗}{{\emojifont\mdseries\upshape ✗}}
\newunicodechar{️}{{\emojifont\mdseries\upshape ️}} % Variation Selector 16

% Colors
\definecolor{primaryColor}{RGB}{0, 102, 204}    % Blue
\definecolor{secondaryColor}{RGB}{255, 153, 51} % Orange
\definecolor{accentColor}{RGB}{51, 153, 102}    % Green
\definecolor{lightGray}{RGB}{245, 245, 245}
\definecolor{kaplanPurple}{HTML}{000080}
\definecolor{kaplanLight}{HTML}{EDE7F6}

% Box for Vocabulary
\newtcolorbox{vocabbox}[1][]{
    colback=lightGray,
    colframe=primaryColor,
    fonttitle=\bfseries,
    title=Vocabulary Focus,
    #1
}

% Box for Grammar
\newtcolorbox{grammarbox}[1][]{
    colback=white,
    colframe=secondaryColor,
    fonttitle=\bfseries,
    title=Grammar Point,
    #1
}

% Box for Examples
\newtcolorbox{examplebox}[1][]{
    colback=white,
    colframe=accentColor,
    fonttitle=\bfseries,
    title=Examples,
    #1
}

% Box for Dialogues/Reading
\newtcolorbox{readingbox}[1][]{
    colback=lightGray,
    colframe=black,
    fonttitle=\bfseries,
    title=Reading Context,
    #1
}

% Box for Notes
\newtcolorbox{notebox}[1][]{
    colback=kaplanLight,
    colframe=kaplanPurple,
    fonttitle=\bfseries,
    title=Note,
    #1
}

% Box for Tips
\newtcolorbox{tipbox}[1][]{
    colback=white,
    colframe=accentColor,
    fonttitle=\bfseries,
    title=Tip,
    #1
}

% Box for Warnings
\newtcolorbox{warningbox}[1][]{
    colback=white,
    colframe=red,
    fonttitle=\bfseries,
    title=Warning,
    #1
}

% Command for translations
\newcommand{\trans}[1]{\textit{\color{gray} (#1)}}

% Command for key terms
\newcommand{\keyterm}[1]{\textbf{\color{primaryColor} #1}}

% Command for British English terms
\newcommand{\britishterm}[1]{\textcolor{blue}{\textbf{#1}}}

% Command for American English terms (for comparison)
\newcommand{\americanterm}[1]{\textcolor{red}{\textit{#1}}}

% Command for CEFR levels
\newcommand{\cefrlevel}[1]{\textbf{\color{accentColor} [CEFR: #1]}}

% Command for correct answers
\newcommand{\correct}{\textcolor{green!60!black}{\checkmark}}

% Command for incorrect answers
\newcommand{\incorrect}{\textcolor{red}{\textbf{X}}}

% Icons and symbols (using Unicode with XeLaTeX)
\newcommand{\britishflag}{{\emojifont\mdseries\upshape 🇬🇧}}
\newcommand{\speakicon}{{\emojifont\mdseries\upshape 🗣️}}
\newcommand{\listenicon}{{\emojifont\mdseries\upshape 🎧}}
\newcommand{\writeicon}{{\emojifont\mdseries\upshape ✍️}}
\newcommand{\readicon}{{\emojifont\mdseries\upshape 📖}}
\newcommand{\warningicon}{{\emojifont\mdseries\upshape ⚠️}}
\newcommand{\tipicon}{{\emojifont\mdseries\upshape 💡}}
\newcommand{\keyicon}{{\emojifont\mdseries\upshape 🔑}}
\newcommand{\soundicon}{{\emojifont\mdseries\upshape 🔊}}

% Box for British Culture Notes
\newtcolorbox{britishbox}[1][]{
    colback=blue!5,
    colframe=red!60!blue,
    fonttitle=\bfseries,
    title=\britishflag\ British Culture Note,
    #1
}

% Box for Pronunciation
\newtcolorbox{pronunciationbox}[1][]{
    colback=purple!5,
    colframe=purple!60!black,
    fonttitle=\bfseries,
    title=\soundicon\ Pronunciation,
    #1
}

% Box for Comparisons (British vs American)
\newtcolorbox{comparisonbox}[1][]{
    colback=green!5,
    colframe=green!60!black,
    fonttitle=\bfseries,
    title=British vs American English,
    #1
}

% Box for Spanish Speaker Notes
\newtcolorbox{spanishbox}[1][]{
    colback=yellow!10,
    colframe=red!70!black,
    fonttitle=\bfseries,
    title=For Spanish Speakers,
    #1
}

% ============================================
% Custom Title Page for Kaplan
% ============================================
\newcommand{\makekaplantitle}{%
    \begin{titlepage}
        \centering

        % Logo
        \includegraphics[width=0.28\textwidth]{kaplan_inc_purple_(4)[1]-2999653380.jpg}\\[1.2cm]

        \noindent\rule{\textwidth}{0.4pt}\vspace{0.4cm}


        % Title Block
        {\huge\bfseries\color{kaplanPurple} Kaplan Course Notes}\par\vspace{0.4cm}
        {\Large English Language}

        \noindent\rule{\textwidth}{0.4pt}\vspace{0.4cm}

        \vfill

        % Metadata
        {\Large Benjamin Figueroa Guzman}\par\vspace{0.2cm}
        {\large \textit{Last updated: \today\ at \currenttime}}

        \noindent\rule{\textwidth}{0.4pt}\vspace{0.4cm}
        
    \end{titlepage}%
}
